\documentclass{article}
\usepackage{graphicx}
\usepackage{amsmath}
\usepackage{hyperref}
\usepackage{epigraph} 

\title{Linear Algebra Done Right\\Solutions to Exercises 3.E}
\author{}
\date{}

\providecommand{\abs}[1]{\lvert#1\rvert} \providecommand{\norm}[1]{\lVert#1\rVert}

\begin{document}

\maketitle

\section{The graph of $T$}
\subsection*{Problem statement}
Suppose $T$ is a function from $V$ to $W$. 
The \textbf{graph} of $T$ is the subset of $V\times W$ defined by
\[\text{graph of }T=\{(v,Tv)\in V\times W : v\in V\}.\]
Prove that $T$ is a linear map if and only if the graph of $T$ is a subspace of $V\times W$.

\subsection*{Solution}
\subsubsection*{First Direction}
Suppose $T$ is a linear map. 
To show that the graph of $T$ is a subspace of $V\times W$, we need to show the properties of \textbf{additive identity}, \textbf{closed under addition}, and \textbf{closed under scalar multiplication}.

\textbf{additive identity}: Clearly $0\in V$. 
Thus, via Theorem 3.11 (`Linear maps take $0$ to $0$'), it follows that
\[(0,0)=(0,T(0))\in \text{graph of }T\].

\textbf{closed under addition}: Suppose $v,u\in V$. 
Thus \newline $(v,Tv),(u,Tu)\in \text{graph of }T$. 
Via the definition of addition on $V\times W$ (Definition 3.71), we have 
\[(v,Tv)+(u,Tu)=(v+u,Tv+Tu)=(v+u,T(v+u))\in\text{graph of }T\]
since $v+u\in V$. 
Note that the second equality follows from the \textbf{additivity} of $T$.

\textbf{closed under scalar multiplication}: Suppose $v\in V$ and $\lambda\in \textbf{F}$. 
Via the definition of scalar multiplication on $V\times W$ (Definition 3.71), we have
\[\lambda(v,Tv)=(\lambda v,\lambda Tv)=(\lambda v,T(\lambda v))\in\text{graph of }T\]
since $\lambda v\in V$. 
Note that the second equality follows from the \textbf{homogeneity} of $T$.

\subsubsection*{Second Direction}
Suppose the graph of $T$ is a subspace of $V\times W$. 
To show that $T$ is a linear map, we need to show the properties of \textbf{additivity} and \textbf{homogeneity}.

\textbf{additivity}: Suppose $v,u\in V$. 
Given the graph of $T$ is \textbf{closed under addition}, it follows that $(v,Tv)+(u,Tu)=(v+u,Tv+Tu)\in \text{graph of }T$. 
Following from $v+u\in V$, we also have $(v+u,T(v+u))\in \text{graph of }T$. 
Hence, it necessarily follows that
\[(v+u,Tv+Tu)=(v+u,T(v+u))\]
and $Tv+Tu=T(v+u)$.

\textbf{homogeneity}: Suppose $v\in V$ and $\lambda\in \textbf{F}$. 
Given the graph of $T$ is \textbf{closed under scalar multiplication}, it follows that\newline $\lambda(v,Tv)=(v,\lambda Tv)\in \text{graph of }T$. 
Following from $\lambda v\in V$, we also have $(\lambda v,T(\lambda v))\in \text{graph of }T$. 
Hence, it necessarily follows that
\[(v,\lambda Tv)=(\lambda v,T(\lambda v))\]
and $\lambda Tv=T(\lambda v)$.

\subsubsection*{Notes}
It's kinda neat how the \textbf{graph} of $T$ allows us to connect the properties of subspaces with the properties of linear maps.

\clearpage

\section{The dimensionality of $V_1\times\cdots\times V_m$ and $V_j$}
\subsection*{Problem statement}
Suppose $V_1,\ldots,V_m$ are vector spaces such that $V_1\times\cdots\times V_m$ is finite-dimensional. 
Prove that $V_j$ is finite-dimensional for each $j=1,\ldots,m$.

\subsection*{Solution}
Let $u_1,\ldots,u_n$ be a basis of $V_1\times\cdots\times V_m$. 
For each vector $u_k$, it follows from the definition of the product of vector spaces (Definition 3.71) that we can express $u_k$ as
\[u_k=(v_1,\ldots,v_m)\]
where each $v_j\in V_j$. 
Let $u_{k,j}$ be the a vector in the $j$-th slot of the $u_k$ basis vector. 
Fix $j$ and construct the list $(u_{1,j},\ldots,u_{n,j})$. 
If this list does not span $V_j$, then $u_1,\ldots,u_n$ is not a basis of $V_1\times\cdots\times V_m$. 
Hence, the list $(u_{1,j},\ldots,u_{n,j})$ spans $V_j$ and $V_j$ is finite-dimensional.

\clearpage

\section{Isomorphic products don't imply direct sums}
\subsection*{Problem statement}
Give an example of a vector space $V$ and subspaces $U_1,U_2$ of $V$ such that $U_1\times U_2$ is isomorphic to $U_1+U_2$ but $U_1+U_2$ is not a direct sum.

\subsection*{Solution}
Let $V$ be the vector space $\mathbf{F}^\infty$, $U_1$ be the subspace defined by
\[U_1=\{(x_1,x_2,0,0,\ldots):x_1,x_2\in\mathbf{F}\},\]
and $U_2$ be the subspace defined by 
\[U_2=\{(0,y_1,y_2,\ldots):y_j\in\mathbf{F}\text{ for }j=1,2,\ldots\}.\]
Clearly $U_1+U_2$ is not a direct sum, but to show that $U_1\times U_2$ is isomorphic to $U_1+U_2$ we need to find an isomorphism. 
Define $T\in\mathcal{L}(U_1\times U_2,U_1+U_2)$ by
\[T((x_1,x_2,0,0,\ldots),(0,y_1,y_2,\ldots))=(x_1,x_2,y_1,y_2,\ldots)\]
and define $S\in\mathcal{L}(U_1+U_2,U_1\times U_2)$
\[S(x_1,x_2,x_3,x_4,\ldots)=((x_1,x_2,0,0,\ldots),(0,x_3,x_4,\ldots)).\]
It follows that $ST=I$ on $U_1\times U_2$ and $TS=I$ on $U_1+U_2$. 
Hence, $T$ is an isomorphism from $U_1\times U_2$ onto $U_1+U_2$.

\clearpage

\section{$\mathcal{L}(V_1\times\cdots\times V_m,W)$ and $\mathcal{L}(V_1,W)\times\cdots\times\mathcal{L}(V_m,W)$}
\subsection*{Problem statement}
Suppose $V_1,\ldots,V_m$ are vector spaces. 
Prove that $\mathcal{L}(V_1\times\cdots\times V_m,W)$ and $\mathcal{L}(V_1,W)\times\cdots\times\mathcal{L}(V_m,W)$ are isomorphic vector spaces.

\subsection*{Solution}
Via Theorem 3.61 (`$\operatorname{dim}\mathcal{L}(V,W)=(\operatorname{dim}V)(\operatorname{dim}W)$') and Theorem 3.76 (`Dimension of a product is the sum of dimensions'), we can write the dimension of $\mathcal{L}(V_1\times\cdots\times V_m,W)$ as
\begin{align*}
    \operatorname{dim}\mathcal{L}(V_1\times\cdots\times V_m,W)&=(\operatorname{dim}V_1\times\cdots\times V_m)(\operatorname{dim}W)\\
    &=(\operatorname{dim}V_1+\cdots+\operatorname{dim}V_m)(\operatorname{dim}W)\\
    &=(\operatorname{dim}V_1)(\operatorname{dim}W)+\cdots+(\operatorname{dim}V_m)(\operatorname{dim}W)
\end{align*}
and the dimension of $\mathcal{L}(V_1,W)\times\cdots\times\mathcal{L}(V_m,W)$ as
\begin{align*}
    \mathcal{L}(V_1,W)\times\cdots\times\mathcal{L}(V_m,W)&=\operatorname{dim}\mathcal{L}(V_1,W)+\cdots+\operatorname{dim}\mathcal{L}(V_m,W)\\
    &=(\operatorname{dim}V_1)(\operatorname{dim}W)+\cdots+(\operatorname{dim}V_m)(\operatorname{dim}W).
\end{align*}
Hence, $\mathcal{L}(V_1\times\cdots\times V_m,W)$ and $\mathcal{L}(V_1,W)\times\cdots\times\mathcal{L}(V_m,W)$ have the same dimension. 
Thus, via Theorem 3.59 (`Dimension shows whether vector spaces are isomorphic'), they are isomorphic vector spaces. 

\clearpage

\section{$\mathcal{L}(V,W_1\times\cdots\times W_m)$ and $\mathcal{L}(V,W_1)\times\cdots\times\mathcal{L}(V,W_m)$}
\subsection*{Problem statement}
Suppose $W_1,\ldots,W_m$ are vector spaces. 
Prove that $\mathcal{L}(V,W_1\times\cdots\times W_m)$ and $\mathcal{L}(V,W_1)\times\cdots\times\mathcal{L}(V,W_m)$ are isomorphic vector spaces.

\subsection*{Solution}
Via Theorem 3.61 (`$\operatorname{dim}\mathcal{L}(V,W)=(\operatorname{dim}V)(\operatorname{dim}W)$') and Theorem 3.76 (`Dimension of a product is the sum of dimensions'), we can write the dimension of $\mathcal{L}(V,W_1\times\cdots\times W_m)$ as
\begin{align*}
    \mathcal{L}(V,W_1\times\cdots\times W_m)&=(\operatorname{dim}V)(\operatorname{dim}W_1\times\cdots\times W_m)\\
    &=(\operatorname{dim}V)(\operatorname{dim}W_1+\cdots+\operatorname{dim}W_m)\\
    &=(\operatorname{dim}V)(\operatorname{dim}W_1)+\cdots+(\operatorname{dim}V)(\operatorname{dim}W_m)
\end{align*}
and the dimension of $\mathcal{L}(V,W_1)\times\cdots\times\mathcal{L}(V,W_m)$ as
\begin{align*}
    \mathcal{L}(V,W_1)\times\cdots\times\mathcal{L}(V,W_m)&=\operatorname{dim}\mathcal{L}(V,W_1)+\cdots+\operatorname{dim}\mathcal{L}(V,W_m)\\
    &=(\operatorname{dim}V)(\operatorname{dim}W_1)+\cdots+(\operatorname{dim}V)(\operatorname{dim}W_m).
\end{align*}
Hence, $\mathcal{L}(V,W_1\times\cdots\times W_m)$ and $\mathcal{L}(V,W_1)\times\cdots\times\mathcal{L}(V,W_m)$ have the same dimension. 
Thus, via Theorem 3.59 (`Dimension shows whether vector spaces are isomorphic'), they are isomorphic vector spaces. 

\clearpage

\section{$V^n$ and $\mathcal{L}(\mathbf{F}^n,V)$ are isomorphic}
\subsection*{Problem statement}
For $n$ a positive integer, define $V^n$ by
\[V^n=V\times\cdots\times V\Big\}(n\text{ times}).\]
Prove that $V^n$ and $\mathcal{L}(\mathbf{F}^n,V)$ are isomorphic vector spaces.

\subsection*{Solution}
Via Theorem 3.76 (`Dimension of a product is the sum of dimensions'), the dimension of $V^n$ is
\begin{align*}
    \operatorname{dim}V^n&=\operatorname{dim}V+\cdots+\operatorname{dim}V\Big\}(n\text{ times})\\
    &=n(\operatorname{dim}V).
\end{align*}
Via Theorem 3.61 (`$\operatorname{dim}\mathcal{L}(V,W)=(\operatorname{dim}V)(\operatorname{dim}W)$'), the dimension of $\mathcal{L}(\mathbf{F}^n,V)$ is
\[\operatorname{dim}\mathcal{L}(\mathbf{F}^n,V)=(\operatorname{dim}\mathbf{F}^n)(\operatorname{dim}V)=n(\operatorname{dim}V).\]
Hence, $V^n$ and $\mathcal{L}(\mathbf{F}^n,V)$ have the same dimension. 
Thus, via Theorem 3.59 (`Dimension shows whether vector spaces are isomorphic'), they are isomorphic vector spaces. 

\clearpage

\section{If $v+U=x+W$, then $U+W$}
\subsection*{Problem statement}
Suppose $v,x$ are vectors in $V$ and $U,W$ are subspaces of $V$ such that\newline $v+U=x+W$. 
Prove that $U=W$.

\subsection*{Solution}
Given $W$ is a subspace, it follows that $0\in W$ and there exists some vector $u\in U$ such that
\[v+u=x+0.\]
Rearranging terms, we have $v-x=-u$ which implies $v-x\in U$. 
Via theorem 3.85 (`Two affine subsets parallel to $U$ are equal or disjoint'), it follows that $v-x\in U$ is equivalent to $v+U=x+U$. 
Hence, we have 
\[x+W=v+U=x+U,\]
which implies that $U=W$.

\clearpage

\section{A property of affine subsets}
\subsection*{Problem statement}
Prove that a nonempty subset $A$ of $V$ is an affine subset of $V$ if and only if $\lambda v+(1-\lambda)w\in A$ for all $v,w\in A$ and all $\lambda\in\mathbf{F}$.

\subsection*{Solution}
\subsubsection*{First Direction}
Suppose $A$ of $V$ is an affine subset. 
It follows from Definition 3.79 (`$v+U$') and Definition 3.81 (`affine subset, parallel') that we can express $A$ as 
\[A=u+U\]
for some vector $u\in V$ and some subspace $U$ of $V$. 
Consider two vectors $v,w\in A$. 
We can represent these vectors as 
\[v=u+u_1\;\;\;\text{and}\;\;\;w=u+u_2\]
for some $u_1,u_2\in U$. 
For all $\lambda\in\mathbf{F}$, we can write
\begin{align*}
    \lambda v+(1-\lambda)w &= \lambda u+\lambda u_1+u+u_2-\lambda u -\lambda u_2\\
    &= u+\lambda u_1+(1-\lambda)u_2\in u+U=A
\end{align*}
where $\lambda u_1+(1-\lambda)u_2\in U$ since $U$ is a subspace and is \textbf{closed under addition}.

\subsubsection*{Second Direction}
Suppose $A$ is a nonempty subset of $V$ such that $\lambda v+(1-\lambda)w\in A$ for all $v,w\in A$ and all $\lambda\in\mathbf{F}$. 
Given $A$ is nonempty, we can choose a vector $v\in A$ and construct the set
\[U=\{u-v:u\in A\}.\]
If we can show that $U$ is a subspace of $V$, then it follows that $A$ is an affine subset since $A=v+U$.

\textbf{additive identity}: Clearly $0\in U$ since $0=v-v\in U$.

\textbf{closed under addition}: Consider two vectors $u_1,u_2\in U$. 
Our construction of $U$ implies there exist $w_1,w_2\in A$ such that
\[u_1=w_1-v\;\;\;\text{and}\;\;\;u_2=w_2-v.\]
To show that $u_1+u_2\in U$, we need to find a vector $u\in A$ such \newline that $u_1+u_2=u-v$. 
It follows that
\[u_1+u_2=w_1+w_2-v-v\]
and we need to show $w_1+w_2-v\in A$.

We have $\frac{1}{2}(w_1+w_2)\in A$ since it is of the for $\lambda w_1+(1-\lambda)w_2$ where $\lambda=\frac{1}{2}$. 
Considering the vectors $\frac{1}{2}(w_1+w_2)$ and $v$ and the scalar $\lambda=2$, it follows that
\[2(\frac{1}{2}(w_1+w_2))+(1-2)v=w_1+w_2-v\in A\]
and $u_1+u_2=w_1+w_2-v-v\in U$. 
Hence, $U$ is \textbf{closed under addition}.

\textbf{closed under scalar multiplication}: Suppose $u\in U$ and $\lambda\in\mathbf{F}$. 
Via our construction of $U$, there exists some $w\in A$ such that $u=w-v$. 
To show that $\lambda u\in U$, we can write
\[\lambda u=\lambda w - \lambda v=\lambda w - (\lambda -1)v-v=\lambda w+(1-\lambda)v-v\in U\]
where $\lambda w+(1-\lambda)v\in A$. Hence, $U$ is \textbf{closed under scalar multiplication}.

\clearpage

\section{Intersection of two affine subsets}
\subsection*{Problem statement}
Suppose $A_1$ and $A_2$ are affine subsets of $V$. 
Prove that the intersection $A_1\cap A_2$ is either an affine subset of $V$ or the empty set.

\subsection*{Solution}
Let's tackle each case separately.

\subsubsection*{Intersection equals the empty set}
To prove that the intersection $A_1\cap A_2$ could be the empty set, we can find an example. 
Suppose $U=\{(x,2x)\in\mathbf{R}^2:x\in\mathbf{R}\}$, $A_1=(1,2)+U$, and $A_2=(-1,2)+U$. 
Geometrically, $A_1$ and $A_2$ are parallel lines in $\mathbf{R}^2$. 
Since these lines are parallel, it follows that $A_1\cap A_2=\emptyset$.

\subsubsection*{Intersection does not equal the empty set}
Suppose $A_1\cap A_2\neq\emptyset$. 
Thus there exists a vector $v\in V$ such that $v\in A_1\cap A_2$. 
Define the set $U$ by 
\[U=\{u-v:u\in A_1\cap A_2\}\]
In a similar manner as our solution to Exercise 3.E(8), if we can show that $U$ is a subspace of $V$, then it follows that $A_1\cap A_2$ is an affine subset. 

\textbf{additive identity}: Clearly $0\in U$ since $0=v-v\in U$.

\textbf{closed under addition}: Consider two vectors $u_1,u_2\in U$. 
Our construction of $U$ implies there exist $w_1,w_2\in A_1\cap A_2$ such that
\[u_1=w_1-v\;\;\;\text{and}\;\;\;u_2=w_2-v.\]
To show that $u_1+u_2\in U$, we need to find a vector $u\in  A_1\cap A_2$ such \newline that $u_1+u_2=u-v$. 
It follows that
\[u_1+u_2=w_1+w_2-v-v\]
and we need to show $w_1+w_2-v\in  A_1\cap A_2$. 
Via our reasoning in our solution to Exercise 3.E(8), it follows that $w_1+w_2-v\in A_1$ and $w_1+w_2-v\in A_2$ since $A_1$ and $A_2$ are affine subsets. 
Hence, we have $w_1+w_2-v\in  A_1\cap A_2$ and $u_1+u_2\in U$. 
Therefore,  $U$ is \textbf{closed under addition}.

\textbf{closed under scalar multiplication}: Suppose $u\in U$ and $\lambda\in\mathbf{F}$. 
Via our construction of $U$, there exists some $w\in A_1\cap A_2$ such that $u=w-v$. 
To show that $\lambda u\in U$, we can write
\[\lambda u=\lambda w - \lambda v=\lambda w - (\lambda -1)v-v=\lambda w+(1-\lambda)v-v\in U\]
where $\lambda w+(1-\lambda)v\in A_1\cap A_2$ via our reasoning in Exercise 3.E(8). 
Hence, $U$ is \textbf{closed under scalar multiplication}.

\clearpage

\section{Intersection of any collection of affine subsets}
\subsection*{Problem statement}
Prove that the intersection of every collection of affine subsets of $V$ is either an affine subset of $V$ or the empty set.

\subsection*{Solution}
Suppose $A_1,A_2,\ldots,A_n$ is a collection of affine subsets. 
By arranging the intersection of the collection as 
\[\big(\ldots(A_1\cap A_2)\cap\ldots\big)\cap A_n\]
we can iteratively apply our result from Exercise 3.E(9) to show that the intersection must be an affine subset of $V$ or the empty set.

\end{document}