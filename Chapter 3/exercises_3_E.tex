\documentclass{article}
\usepackage{graphicx}
\usepackage{amsmath}
\usepackage{hyperref}
\usepackage{epigraph} 

\title{Linear Algebra Done Right\\Solutions to Exercises 3.E}
\author{}
\date{}

\providecommand{\abs}[1]{\lvert#1\rvert} \providecommand{\norm}[1]{\lVert#1\rVert}

\begin{document}

\maketitle

\section{The graph of $T$}
\subsection*{Problem statement}
Suppose $T$ is a function from $V$ to $W$. 
The \textbf{graph} of $T$ is the subset of $V\times W$ defined by
\[\text{graph of }T=\{(v,Tv)\in V\times W : v\in V\}.\]
Prove that $T$ is a linear map if and only if the graph of $T$ is a subspace of $V\times W$.

\subsection*{Solution}
\subsubsection*{First Direction}
Suppose $T$ is a linear map. 
To show that the graph of $T$ is a subspace of $V\times W$, we need to show the properties of \textbf{additive identity}, \textbf{closed under addition}, and \textbf{closed under scalar multiplication}.

\textbf{additive identity}: Clearly $0\in V$. 
Thus, via Theorem 3.11 (`Linear maps take $0$ to $0$'), it follows that
\[(0,0)=(0,T(0))\in \text{graph of }T\].

\textbf{closed under addition}: Suppose $v,u\in V$. 
Thus \newline $(v,Tv),(u,Tu)\in \text{graph of }T$. 
Via the definition of addition on $V\times W$ (Definition 3.71), we have 
\[(v,Tv)+(u,Tu)=(v+u,Tv+Tu)=(v+u,T(v+u))\in\text{graph of }T\]
since $v+u\in V$. 
Note that the second equality follows from the \textbf{additivity} of $T$.

\textbf{closed under scalar multiplication}: Suppose $v\in V$ and $\lambda\in \textbf{F}$. 
Via the definition of scalar multiplication on $V\times W$ (Definition 3.71), we have
\[\lambda(v,Tv)=(\lambda v,\lambda Tv)=(\lambda v,T(\lambda v))\in\text{graph of }T\]
since $\lambda v\in V$. 
Note that the second equality follows from the \textbf{homogeneity} of $T$.

\subsubsection*{Second Direction}
Suppose the graph of $T$ is a subspace of $V\times W$. 
To show that $T$ is a linear map, we need to show the properties of \textbf{additivity} and \textbf{homogeneity}.

\textbf{additivity}: Suppose $v,u\in V$. 
Given the graph of $T$ is \textbf{closed under addition}, it follows that $(v,Tv)+(u,Tu)=(v+u,Tv+Tu)\in \text{graph of }T$. 
Following from $v+u\in V$, we also have $(v+u,T(v+u))\in \text{graph of }T$. 
Hence, it necessarily follows that
\[(v+u,Tv+Tu)=(v+u,T(v+u))\]
and $Tv+Tu=T(v+u)$.

\textbf{homogeneity}: Suppose $v\in V$ and $\lambda\in \textbf{F}$. 
Given the graph of $T$ is \textbf{closed under scalar multiplication}, it follows that\newline $\lambda(v,Tv)=(v,\lambda Tv)\in \text{graph of }T$. 
Following from $\lambda v\in V$, we also have $(\lambda v,T(\lambda v))\in \text{graph of }T$. 
Hence, it necessarily follows that
\[(v,\lambda Tv)=(\lambda v,T(\lambda v))\]
and $\lambda Tv=T(\lambda v)$.


\end{document}