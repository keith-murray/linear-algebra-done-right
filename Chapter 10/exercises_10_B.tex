\documentclass{article}
\usepackage{graphicx}
\usepackage{amsmath}
\usepackage{hyperref}
\usepackage{epigraph}
\usepackage{csquotes}
\usepackage{mathtools}

\title{Linear Algebra Done Right\\Solutions to Exercises 10.B}
\author{}
\date{}

\begin{document}

\maketitle

\section{If $T$ has no eigenvalues, then $\det T>0$}
\subsection*{Problem statement}
Suppose $V$ is a real vector space. 
Suppose $T\in\mathcal{L}(V)$ has no eigenvalues. 
Prove that $\det T>0$.

\subsection*{Solution}
$T$ having no eigenvalues in $V$ implies all eigenvalues of $T_{\mathbf{C}}$ are nonreal because real eigenvalues of $T_{\mathbf{C}}$ would correspond to real eigenvalues of $T$. 
Suppose $\lambda_1,\ldots,\lambda_m$ are the nonreal eigenvalues of $T_{\mathbf{C}}$ and the determinant is, via Definition 10.20 (`determinant of an operator'), 
\[\det T=\lambda_1\cdots\lambda_m.\] 
Following from Theorem 9.16 (`Nonreal eigenvalues of $T_{\mathbf{C}}$ come in pairs') and Theorem 9.17 (`Multiplicity of $\lambda$ equals multiplicity of $\bar{\lambda}$'), all eigenvalues of $T_{\mathbf{C}}$ can be paired with its complex conjugate and we can rearrange the terms in the determinant of $T$ to be\footnote{
Note the change of $m$ to $n$. 
$T$ having no eigenvalues and Theorem 9.19 imply that $m$ is even and $m=2n$.}
\[\det T=\lambda_1\bar{\lambda_1}\cdots\lambda_n\bar{\lambda_n}.\]
Finally, since Theorem 4.5 (`Properties of complex numbers') states $z\bar{z}=|z|^2$, we can write
\[\det T=|\lambda_1|^2\cdots|\lambda_n|^2.\]
Since all of the terms in $\det T$ are positive, it follows that $\det T>0$.

Note that $\det T\neq0$ since that would require $0$ to be an eigenvalue of $T_{\mathbf{C}}$, which would imply $T$ to have an eigenvalue.

\clearpage

\section{$\det T<0$ implies $T$ has two eigenvalues}
\subsection*{Problem statement}
Suppose $V$ is a real vector space with even dimension and $T\in\mathcal{L}(V)$. 
Suppose $\det T<0$. 
Prove that $T$ has at least two distinct eigenvalues.

\subsection*{Solution}
Following from Exercise 10.B(1), if $T$ has no eigenvalues, then $\det T>0$. 
This contradicts the assumption that $\det T<0$. 
Hence, $T$ must have eigenvalues. 

Now we have two cases to consider: one in which $T_{\mathbf{C}}$ has no nonreal eigenvalues and one in which $T_{\mathbf{C}}$ has some nonreal eigenvalues.

\subsubsection*{All eigenvalues of $T_{\mathbf{C}}$ are real}
Suppose all eigenvalues of $T_{\mathbf{C}}$ are real. 
If there is only one distinct real eigenvalue $\lambda$, then its multiplicity equals the dimension of $V$, which is even and implies that
\[\det T=\lambda^{\dim V}>0.\]
Hence, there must be at least two distinct eigenvalues where the combined multiplicities of the negative eigenvalues is odd.

\subsubsection*{Some eigenvalues of $T_{\mathbf{C}}$ are nonreal}
Following from Theorem 9.16 (`Nonreal eigenvalues of $T_{\mathbf{C}}$ come in pairs') and Theorem 9.17 (`Multiplicity of $\lambda$ equals multiplicity of $\bar{\lambda}$'), nonreal eigenvalues of $T_{\mathbf{C}}$ come in pairs with equal multiplicity. 
Thus, the combined multiplicities of the real eigenvalues must be even. 

Following the logic from the first case, there must be at least two distinct eigenvalues where the combined multiplicities of the negative eigenvalues is odd.

\subsubsection*{Conclusion}
Therefore, in all cases, $T$ has at least two distinct eigenvalues.


\end{document}