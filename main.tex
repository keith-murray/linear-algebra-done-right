\documentclass{article}
\usepackage{graphicx}
\usepackage{amsmath}
\usepackage{hyperref}
\usepackage{epigraph} 

\title{Linear Algebra Done Right\\Solutions Guide}
\author{}
\date{}

\begin{document}

\maketitle

\epigraph{Algebra is but written geometry and geometry is but figured algebra.}{\textit{Sophie Germain}}

\section{Purpose}
There are many interpretations of mathematics, but I prefer the interpretation of math as a language. Learning a language involves reading, so it follows that you should read your textbooks, but it also involves conversing and writing, so you should talk with others about the math you learn and attempt to solve exercises included in your textbook.

I'm writing the solutions to \textit{Linear Algebra Done Right} mainly as an exercise in my ability to write math, but also as a resource for others to read math. Perhaps by providing the solutions to these exercises I am depriving the reader of the chance for them to write math, and thus the chance for them to learn math. But in the age of the internet, solutions to math problems are easy to come by. If the reader is intent on learning math, then they must exercise the fortitude necessary to prevent themselves from looking at these answers. But once you have the answers, sources like these provide an invaluable resource to assess if you came to the correct answer.

Languages are tools for conversation. So these answers are one-half of that conversation.

\end{document}