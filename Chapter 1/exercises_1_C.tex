\documentclass{article}
\usepackage{graphicx}
\usepackage{amsmath}
\usepackage{hyperref}
\usepackage{epigraph} 

\title{Linear Algebra Done Right\\Solutions to Exercises 1.B}
\author{}
\date{}

\begin{document}

\maketitle

\section{Set or subspace?}
\subsection*{Problem statement}
For each of the following subset of $\mathbf{F}^3$, determine whether it is a subspace of $\mathbf{F}^3$:
\begin{itemize}
  \item[(a)] $\{(x_1,x_2,x_3)\in\mathbf{F}^3:x_1+2x_2+3x_3=0\}$;
  \item[(b)] $\{(x_1,x_2,x_3)\in\mathbf{F}^3:x_1+2x_2+3x_3=4\}$;
  \item[(c)] $\{(x_1,x_2,x_3)\in\mathbf{F}^3:x_1x_2x_3=0\}$;
  \item[(d)] $\{(x_1,x_2,x_3)\in\mathbf{F}^3:x_1=5x_3\}$.
\end{itemize}

\subsection*{Solution}
\subsubsection*{a}
To determine if the subset $\{(x_1,x_2,x_3)\in\mathbf{F}^3:x_1+2x_2+3x_3=0\}$ is a subspace of $\mathbf{F}^3$, we can assess if the subset satisfies the properties of a subspace: \textbf{additive identity}, \textbf{closed under addition}, and \textbf{closed under scalar multiplication}.

\textbf{additive identity}: Given that $0+2\cdot0+3\cdot0=0$, we have \newline $(0,0,0)\in\{(x_1,x_2,x_3)\in\mathbf{F}^3:x_1+2x_2+3x_3=0\}$.

\textbf{closed under scalar multiplication}: Suppose\newline $(y_1,y_2,y_3)\in\{(x_1,x_2,x_3)\in\mathbf{F}^3:x_1+2x_2+3x_3=0\}$ and $\lambda\in\mathbf{F}$. 
It follows that 
\begin{align*}
    y_1+2y_2+3y_3&=0=\frac{1}{\lambda}\cdot0\\
    \lambda(y_1+2y_2+3y_3)&=0\\
    \lambda y_1+2(\lambda y_2)+3(\lambda y_3)&=0.
\end{align*}
Hence, we have $\lambda(y_1,y_2,y_3)\in\{(x_1,x_2,x_3)\in\mathbf{F}^3:x_1+2x_2+3x_3=0\}$.

\textbf{closed under addition}: Suppose\newline $(x_1,x_2,x_3),(y_1,y_2,y_3)\in\{(x_1,x_2,x_3)\in\mathbf{F}^3:x_1+2x_2+3x_3=0\}$. 
It follows that 
\[x_1+2x_2+3x_3=0\;\;\;\text{and}\;\;\;y_1+2y_2+3y_3=0.\]
Adding the two equations above, we have
\begin{align*}
    x_1+2x_2+3x_3+y_1+2y_2+3y_3&=0\\
    (x_1+y_1)+2(x_2+y_2)+3(x_3+y_3)&=0.
\end{align*}
Hence, we have\newline $(x_1,x_2,x_3)+(y_1,y_2,y_3)\in\{(x_1,x_2,x_3)\in\mathbf{F}^3:x_1+2x_2+3x_3=0\}$.

\end{document}