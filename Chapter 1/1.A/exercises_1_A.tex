\documentclass{article}
\usepackage{graphicx}
\usepackage{amsmath}
\usepackage{hyperref}
\usepackage{epigraph} 

\title{Linear Algebra Done Right\\Solutions to Exercises 1.A}
\author{}
\date{}

\begin{document}

\maketitle

\section{Inverse of a complex number}
\subsection*{Problem statement}
Suppose $a$ and $b$ are real numbers, not both $0$. Find real numbers $c$ and $d$ such that 
\[1/(a+bi)=c+di\]

\subsection*{Solution}
While we weren't given the definition of the complex conjugate\footnote{We'll have to wait for Chapter 4 to get Definition 4.3.} in Chapter 1.A, we can find $c$ and $d$ by multiplying $1/(a+bi)$ by its complex conjugate. 
Hence we have
\begin{equation*}
    \frac{1}{a+bi}\frac{a-bi}{a-bi}=\frac{a-bi}{(a^2+b^2)+(-ab+ab)i}=\frac{a-bi}{a^2+b^2}=\frac{a}{a^2+b^2}+\frac{-b}{a^2+b^2}i
\end{equation*}
where the first equality comes from the definition of addition and multiplication on $\textbf{C}$ (Definition 1.1). Thus $c=a/(a^2+b^2)$ and $d=-b/(a^2+b^2)$.

\clearpage

\section{Cube root of $1$}
\subsection*{Problem statement}
Show that 
\[\frac{-1+\sqrt{3}i}{2}\]
is a cube root of $1$ (meaning that is cube equals $1$).

\subsection*{Solution}
\begin{align*}
    \left( \frac{-1+\sqrt{3}i}{2}\right)^3&=\left( \frac{-1+\sqrt{3}i}{2}\right)\left( \frac{-1+\sqrt{3}i}{2}\right) \left( \frac{-1+\sqrt{3}i}{2} \right)\\
    &=\frac{1}{8}(-1+\sqrt{3}i)(-1+\sqrt{3}i)(-1+\sqrt{3}i)\\
    &=\frac{1}{8}((1-3)+(-2\sqrt{3})i)(-1+\sqrt{3}i)\\
    &=\frac{1}{8}(-2-2\sqrt{3}i)(-1+\sqrt{3}i)\\
    &=\frac{1}{8}((2+6)+(-2\sqrt{3}+2\sqrt{3})i)\\
    &=\frac{1}{8}(8+0i)\\
    &=1
\end{align*}

\clearpage

\section{Square roots of $i$}
\subsection*{Problem statement}
Find two distinct square roots of $i$.

\subsection*{Solution}
The square roots of $i$ concern $\frac{1}{\sqrt{2}}$.

\begin{align*}
    (\frac{1}{\sqrt{2}}+\frac{1}{\sqrt{2}}i)^2&=(\frac{1}{\sqrt{2}}+\frac{1}{\sqrt{2}}i)(\frac{1}{\sqrt{2}}+\frac{1}{\sqrt{2}}i)\\
    &=(\frac{1}{2}-\frac{1}{2})+(\frac{1}{2}+\frac{1}{2})i\\
    &=0+1i\\
    &=i
\end{align*}

\begin{align*}
    (-\frac{1}{\sqrt{2}}-\frac{1}{\sqrt{2}}i)^2&=(-\frac{1}{\sqrt{2}}-\frac{1}{\sqrt{2}}i)(-\frac{1}{\sqrt{2}}-\frac{1}{\sqrt{2}}i)\\
    &=(\frac{1}{2}-\frac{1}{2})+(\frac{1}{2}+\frac{1}{2})i\\
    &=0+1i\\
    &=i
\end{align*}

\clearpage

\renewcommand{\thesection}{11}
\section{System of equations to find $\lambda\in\textbf{C}$}
\subsection*{Problem statement}
Explain why there does not exist $\lambda\in\textbf{C}$ such that
\[\lambda(2-3i,5+4i,-6+7i)=(12-5i,7+22i,-32-9i).\]

\subsection*{Solution}
We can write $\lambda\in\textbf{C}$ as $\lambda=a+bi$. 
Now we can frame our problem into a series of linear equations. 
We can write 
\begin{gather*} 
    (a+bi)(2-3i)=12-5i\\
    (2a+3b)+(-3a+2b)i=12-5i\\
    a=-\frac{3}{2}b+6\;\;\;\text{and}\;\;\;a=\frac{2}{3}b+\frac{5}{3}\\
    -\frac{3}{2}b+6=\frac{2}{3}b+\frac{5}{3}\\
    -\frac{3}{2}b-\frac{2}{3}b=\frac{5}{3}-6\\
    -\frac{15}{6}b=-\frac{13}{3}\\
    b=\frac{26}{15}
\end{gather*}
and
\begin{gather*}
    (a+bi)(5+4i)=7+22i\\
    (5a-4b)+(4a+5b)i=7+22i\\
    a=\frac{4}{5}b+\frac{7}{5}\;\;\;\text{and}\;\;\;a=-\frac{5}{4}b+\frac{11}{2}\\
    \frac{4}{5}b+\frac{5}{4}b=-\frac{7}{5}+\frac{11}{2}\\
    \frac{41}{20}b=\frac{41}{10}\\
    b=2.
\end{gather*}
Hence, the systems of linear equations does not have a solution and no $\lambda\in\mathbf{C}$ exists.

\end{document}