\documentclass{article}
\usepackage{graphicx}
\usepackage{amsmath}
\usepackage{hyperref}
\usepackage{epigraph} 

\title{Linear Algebra Done Right\\Solutions to Exercises 1.C}
\author{}
\date{}

\begin{document}

\maketitle

\section{Set or subspace?}
\subsection*{Problem statement}
For each of the following subset of $\mathbf{F}^3$, determine whether it is a subspace of $\mathbf{F}^3$:
\begin{itemize}
  \item[(a)] $\{(x_1,x_2,x_3)\in\mathbf{F}^3:x_1+2x_2+3x_3=0\}$;
  \item[(b)] $\{(x_1,x_2,x_3)\in\mathbf{F}^3:x_1+2x_2+3x_3=4\}$;
  \item[(c)] $\{(x_1,x_2,x_3)\in\mathbf{F}^3:x_1x_2x_3=0\}$;
  \item[(d)] $\{(x_1,x_2,x_3)\in\mathbf{F}^3:x_1=5x_3\}$.
\end{itemize}

\subsection*{Solution}
\subsubsection*{a}
To determine if the subset $\{(x_1,x_2,x_3)\in\mathbf{F}^3:x_1+2x_2+3x_3=0\}$ is a subspace of $\mathbf{F}^3$, we can assess if the subset satisfies the properties of a subspace: \textbf{additive identity}, \textbf{closed under addition}, and \textbf{closed under scalar multiplication}.

\textbf{additive identity}: Given that $0+2\cdot0+3\cdot0=0$, we have \newline $(0,0,0)\in\{(x_1,x_2,x_3)\in\mathbf{F}^3:x_1+2x_2+3x_3=0\}$.

\textbf{closed under scalar multiplication}: Suppose\newline $(y_1,y_2,y_3)\in\{(x_1,x_2,x_3)\in\mathbf{F}^3:x_1+2x_2+3x_3=0\}$ and $\lambda\in\mathbf{F}$. 
It follows that 
\begin{align*}
    y_1+2y_2+3y_3&=0=\frac{1}{\lambda}\cdot0\\
    \lambda(y_1+2y_2+3y_3)&=0\\
    \lambda y_1+2(\lambda y_2)+3(\lambda y_3)&=0.
\end{align*}
Hence, we have $\lambda(y_1,y_2,y_3)\in\{(x_1,x_2,x_3)\in\mathbf{F}^3:x_1+2x_2+3x_3=0\}$.

\textbf{closed under addition}: Suppose\newline $(x_1,x_2,x_3),(y_1,y_2,y_3)\in\{(x_1,x_2,x_3)\in\mathbf{F}^3:x_1+2x_2+3x_3=0\}$. 
It follows that 
\[x_1+2x_2+3x_3=0\;\;\;\text{and}\;\;\;y_1+2y_2+3y_3=0.\]
Adding the two equations above, we have
\begin{align*}
    x_1+2x_2+3x_3+y_1+2y_2+3y_3&=0\\
    (x_1+y_1)+2(x_2+y_2)+3(x_3+y_3)&=0.
\end{align*}
Hence, we have\newline $(x_1,x_2,x_3)+(y_1,y_2,y_3)\in\{(x_1,x_2,x_3)\in\mathbf{F}^3:x_1+2x_2+3x_3=0\}$.

Since the subset $\{(x_1,x_2,x_3)\in\mathbf{F}^3:x_1+2x_2+3x_3=0\}$ satisfies all the conditions of a subspace, via Theorem 1.34 (`Conditions of a subspace'), it is a subspace of $\mathbf{F}^3$.

\subsubsection*{b}
No, the subset $\{(x_1,x_2,x_3)\in\mathbf{F}^3:x_1+2x_2+3x_3=4\}$  is not a subspace of $\textbf{F}^3$ because it does not contain the \textbf{additive identity}. 
This follows from the observation that
\[0+2\cdot 0 + 3\cdot 0\neq 4.\]

\subsubsection*{c}
No, the subset $\{(x_1,x_2,x_3)\in\mathbf{F}^3:x_1x_2x_3=0\}$ is not a subspace of $\textbf{F}^3$ because it is not \textbf{closed under addition}. 
We can show this by noting that the vectors $(0,1,1)$ and $(1,0,1)$ are members of the subset, but $(0,1,1)+(1,0,1)$ is not a member since
\[(0,1,1)+(1,0,1)=(1,1,2)\;\;\;\text{and}\;\;\;1\cdot1\cdot2\neq0.\]

\subsubsection*{d}
Yes, the subset $\{(x_1,x_2,x_3)\in\mathbf{F}^3:x_1=5x_3\}$ is a subspace of $\textbf{F}^3$. 
This follows from the observation that the following set 
\[\{(x_1,x_2,x_3)\in\mathbf{F}^3:x_1+0x_2-5x_3=0\}\]
is an equivalent formulation of $\{(x_1,x_2,x_3)\in\mathbf{F}^3:x_1=5x_3\}$ is a subspace of $\textbf{F}^3$ and is of a similar form to the subspace $\{(x_1,x_2,x_3)\in\mathbf{F}^3:x_1+2x_2+3x_3=0\}$, which we showed was a subspace of $\textbf{F}^3$. 
A verification that the subset $\{(x_1,x_2,x_3)\in\mathbf{F}^3:x_1+0x_2-5x_3=0\}$ satisfies all the conditions of a subspace is near identical to the reasoning in part \textbf{a}.

\clearpage

\renewcommand{\thesection}{8}
\section{Set closed under multiplication not subspace}
\subsection*{Problem statement}
Give an example of a nonempty subset $U$ of $\mathbf{R}^2$ such that $U$ is closed under scalar multiplication, but $U$ is not a subspace of $\mathbf{R}^2$.

\subsection*{Solution}
The subset 
\[U=\{(x_1,x_2)\in\textbf{R}^2:x_1x_2=0\}\]
is \textbf{closed under scalar multiplication} but is not \textbf{closed under addition}; thus, it is not a subspace. 

To verify that $U$ is \textbf{closed under scalar multiplication}, for $\lambda\in\textbf{R}$ and $(x_1,x_2)\in U$, it follows that for $\lambda(x_1,x_2)=(\lambda x_1,\lambda x_2)$ and we can write
\[(\lambda x_1)(\lambda x_2)=\lambda^2(x_1x_2)=\lambda^2(0)=0\]
since $x_1x_2=0$.

To verify that $U$ is not \textbf{closed under addition}, it follows that $(1,0),(0,1)\in U$ but $(1,0)+(0,1)=(1,1)$ and $(1,1)\notin U$.

\clearpage

\renewcommand{\thesection}{10}
\section{Intersection of subspaces is a subspace}
\subsection*{Problem statement}
Suppose $U_1$ and $U_2$ are subspaces of $V$. 
Prove that the intersection $U_1\cap U_2$ is a subspace of $V$.

\subsection*{Solution}
Via Theorem 1.34 (‘Conditions of a subspace’), we need to show that $U_1\cap U_2$ satisfies the conditions of \textbf{additive identity}, \textbf{closed under scalar multiplication}, and \textbf{closed under addition}.

\textbf{additive identity}: Given $U_1$ and $U_2$ are subspaces, it follows that $0\in U_1$ and $0\in U_2$. 
Hence, we have $0\in U_1\cap U_2$.

\textbf{closed under scalar multiplication}: Suppose $v\in U_1\cap U_2$ and $\lambda\in\mathbf{F}$. 
Since $U_1$ and $U_2$ are \textbf{closed under scalar multiplication}, it follows that $\lambda v\in U_1$ and $\lambda v\in U2$. 
Hence, we have $\lambda v\in U_1\cap U_2$.

\textbf{closed under addition}: Suppose $v,u\in U_1\cap U_2$. 
Since $U_1$ and $U_2$ are \textbf{closed under addition}, it follows that $v+u\in U_1$ and $v+u\in U2$. 
Hence, we have $v+u\in U_1\cap U_2$.

Therefore, since $U_1\cap U_2$ satisfies all the conditions of a subspace, it follows that $ U_1\cap U_2$ is a subspace $V$.

\clearpage

\renewcommand{\thesection}{11}
\section{Intersection of every collection of subspaces}
\subsection*{Problem statement}
Prove that the intersection of every collection of subspaces of $V$ is a subspace of $V$.

\subsection*{Solution}
Suppose $U_1,U_2,\ldots,U_n$ is a collection of subspaces of $V$. 
By arranging the intersection of the collection as 
\[\big(\ldots(U_1\cap U_2)\cap\ldots\big)\cap U_n\]
we can iteratively apply our result from Exercise 1.C(10) to show that the intersection is a subspace of $V$.

\clearpage

\renewcommand{\thesection}{21}
\section{Find $W$ such that $\mathbf{F}^5=U\oplus W$}
\subsection*{Problem statement}
Suppose 
\[U=\{ (x,y,x+y,x-y,2x)\in \mathbf{F}^5:x,y\in\mathbf{F}\}.\]
Find a subspace $W$ of $\mathbf{F}^5$ such that $\mathbf{F}^5=U\oplus W$.

\subsection*{Solution}
Define the subset $W$ of $\mathbf{F}^5$ as
\[W=\{(0,0,z,u,v)\in\mathbf{F}^5:z,u,v\in\mathbf{F}\}.\]
To prove that $\mathbf{F}^5=U\oplus W$, we need to show that $W$ is a subspace of $\mathbf{F}^5$, $\mathbf{F}^5=U + W$, and $U+W$ is a direct sum.

It obviously follows from the construction of $W$ that $W$ is a subspace of $\mathbf{F}^5$.

To prove $\mathbf{F}^5=U + W$, we need to show $U + W\subset \mathbf{F}^5$ and $\mathbf{F}^5\subset U+W$. 
Suppose $(x,y,x+y,x-y,2x)\in U$ and $(0,0,z,u,v)\in W$. 
It follows that
\begin{multline*}
(x,y,x+y,x-y,2x)+(0,0,z,u,v)\\
=(x,y,x+y+z,x-y+u,2x+v)\in\mathbf{F}^5,
\end{multline*}
showing $U + W\subset \mathbf{F}^5$. 
Suppose $(x,y,z,u,v)\in\mathbf{F}^5$. 
We can construct $(x,y,z,u,v)$ from $U+W$ by choosing vectors $(x,y,x+y,x-y,2x)\in U$ and $(0,0,z-x-y,u-x+y,v-2x)\in W$ and writing
\[(x,y,x+y,x-y,2x)+(0,0,z-x-y,u-x+y,v-2x)=(x,y,z,u,v).\]
Hence, we've shown $U + W\subset \mathbf{F}^5$,[] and it follows that $\mathbf{F}^5=U + W$.

To prove $U+W$ is a direct sum, it follows from Theorem 1.45 (`Direct sum of two subspaces') that we need only show $U\cap W=\{0\}$. 
This result follows from the observation that all vectors in $W$ have a $0$ in the $1^{\text{st}}$ and $2^{\text{nd}}$ coordinates. 
Hence, only vectors with a $0$ in the $1^{\text{st}}$ and $2^{\text{nd}}$ coordinates in $U$ can be members of $U\cap W$. 
Only one vector in $U$ has a $0$ in the $1^{\text{st}}$ and $2^{\text{nd}}$ coordinates, namely the $0$ vector. Thus, it follows that $U\cap W=\{0\}$.

\end{document}