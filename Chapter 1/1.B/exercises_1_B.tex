\documentclass{article}
\usepackage{graphicx}
\usepackage{amsmath}
\usepackage{hyperref}
\usepackage{epigraph} 

\title{Linear Algebra Done Right\\Solutions to Exercises 1.B}
\author{}
\date{}

\begin{document}

\maketitle

\section{Prove that $-(-v)=v$}
\subsection*{Problem statement}
Prove that $-(-v)=v$ for every $v\in V$.

\subsection*{Solution}
Via the properties of \textbf{additive inverse} and \textbf{commutativity} (Definition 1.19), we have
\[-v+(-(-v))=0\;\;\;\text{and}\;\;\;v+(-v)=-v+v=0\]
for every $v\in V$.
But via Theorem 1.26 (`Unique additive inverse'), every element in a vector space has a unique additive inverse. 
Hence, it necessarily follows that $-(-v)=v$ for every $v\in V$.

\clearpage

\renewcommand{\thesection}{4}
\section{Why is the empty set not a vector space?}
\subsection*{Problem statement}
The empty set is not a vector space. 
The empty set fails to satisfy only one of the requirements listed in 1.19. 
Which one?

\subsection*{Solution}
The empty set fails to satisfy that property of \textbf{additive identity} (Definition 1.19). 
The property stipulates that an element exists, but the empty set has no elements.

\clearpage

\renewcommand{\thesection}{5}
\section{Replacing the additive inverse condition}
\subsection*{Problem statement}
Show that in the definition of a vector space (1.19), the additive inverse condition can be replaced with the condition that
\[0v=0\text{ for all }v\in V.\]
Here the $0$ on the left side is the number $0$, and the $0$ on the right side is the additive identity of $V$.

\subsection*{Solution}
We can replace the \textbf{additive inverse} condition for vector spaces (Definition 1.19) with the condition that 
\[0v=0\text{ for all }v\in V,\]
the \textbf{additive inverse} condition for fields (Theorem 1.3), and the \textbf{distributive properties} condition for vector spaces (Definition 1.19). 
Via the \textbf{additive inverse} condition for fields (Theorem 1.3), we can state $1+(-1)=0$. 
Via the condition that 
\[0v=0\text{ for all }v\in V,\]
we can substitute $0$ with $1+(-1)$ to get
\[(1+(-1))v=0\text{ for all }v\in V.\]
Via the the \textbf{distributive properties} condition for vector spaces (Definition 1.19), we now have 
\[(v+(-1)v=0\text{ for all }v\in V.\]
This result is identical to the \textbf{additive inverse} condition. 
Hence, we have deduced the \textbf{additive inverse} condition from our condition that
\[0v=0\text{ for all }v\in V.\]

\clearpage

\renewcommand{\thesection}{6}
\section{Is $\mathbf{R}\cup\{\infty\}\cup\{-\infty\}$ a vector space?}
\subsection*{Problem statement}
Let $\infty$ and $-\infty$ denote two distinct objects, neither of which is in $\mathbf{R}$. 
Define an addition and scalar multiplication on $\mathbf{R}\cup\{\infty\}\cup\{-\infty\}$ as you could guess from the notation. 
Is $\mathbf{R}\cup\{\infty\}\cup\{-\infty\}$ a vector space over $\mathbf{R}$? 
Explain.

\subsection*{Solution}
No, $\mathbf{R}\cup\{\infty\}\cup\{-\infty\}$ is not a vector space over $\mathbf{R}$ because it does not satisfy the \textbf{distributive properties} (Definition 1.19). 
Consider the expression $(2-1)\infty$. 
We can write 
\[(2-1)\infty=2\infty+(-1)\infty=\infty+(-\infty)=0\]
and
\[(2-1)\infty=(1)\infty=\infty.\]
Thus $\mathbf{R}\cup\{\infty\}\cup\{-\infty\}$ is not a vector space over $\mathbf{R}$.

\end{document}