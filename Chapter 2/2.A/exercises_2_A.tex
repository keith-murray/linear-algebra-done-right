\documentclass{article}
\usepackage{graphicx}
\usepackage{amsmath}
\usepackage{hyperref}
\usepackage{epigraph} 

\title{Linear Algebra Done Right\\Solutions to Exercises 1.A}
\author{}
\date{}

\begin{document}

\maketitle

\section{$v_1-v_2,v_2-v_3,v_3-v_4,v_4$ spans $v$}
\subsection*{Problem statement}
Suppose $v_1,v_2,v_3,v_4$ spans $V$. 
Prove that the list
\[v_1-v_2,v_2-v_3,v_3-v_4,v_4\]
also spans $V$.

\subsection*{Solution}
Given $v_1,v_2,v_3,v_4$ spans $V$, every vector $v\in V$ can be written as 
\[v=a_1v_1+a_2v_2+a_3v_3+a_4v_4\]
for some $a_1,a_2,a_3,a_4\in\mathbf{F}$. 
We can write a vector\newline $u\in\operatorname{span}(v_1-v_2,v_2-v_3,v_3-v_4,v_4)$ as
\begin{align*}
    u&=b_1(v_1-v_2)+b_2(v_2-v_3)+b_3(v_3-v_4)+b_4v_4\\
    &=b_1v_1+(b_2-b_1)v_2+(b_3-b_2)v_3+(b_4-b_3)v_4
\end{align*}
for some $b_1,b_2,b_3,b_4\in\mathbf{F}$. 
To show that $v\in\operatorname{span}(v_1-v_2,v_2-v_3,v_3-v_4,v_4)$, we can set $b_1=a_1,b_2=a_2+a_1,b_3=a_3+a_2+a_1,b_4=a_4+a_3+a_2+a_1$ to write
\begin{align*}
    b_1v_1+&(b_2-b_1)v_2+(b_3-b_2)v_3+(b_4-b_3)v_4=a_1v_1+(a_2+a_1-a_1)v_2\\
    &+(a_3+a_2+a_1-a_2-a_1)v_3+(a_4+a_3+a_2+a_1-a_3-a_2-a_1)v_4\\
    &=a_1v_1+a_2v_2+a_3v_3+a_4v_4\\
    &=v.
\end{align*}
Hence, for any vector $v\in V$, it follows that $v\in\operatorname{span}(v_1-v_2,v_2-v_3,v_3-v_4,v_4)$. 
Therefore, the list
\[v_1-v_2,v_2-v_3,v_3-v_4,v_4\]
spans $V$. 

\clearpage

\renewcommand{\thesection}{5}
\section{Sometimes what $\mathbf{F}$ is matters}
\subsection*{Problem statement}
\begin{enumerate}
    \item[(a)] Show that if we think of $\mathbf{C}$ as a vector space over $\mathbf{R}$, then the list $(1+i,1-i)$ is linearly independent.
    \item[(b)] Show that if we think of $\mathbf{C}$ as a vector space over $\mathbf{C}$, then the list $(1+i,1-i)$ is linearly dependent.
\end{enumerate}

\subsection*{Solution}
\subsubsection*{a}
Via Definition 2.17 (`linearly independent'), we need to show that for
\[\lambda_1(1+i)+\lambda(1-i)=0,\]
then $\lambda_1=\lambda_2=0$. 
Our equation above becomes the two separate equations
\[\lambda_1+\lambda_2=0\;\;\;\text{and}\;\;\;\lambda_1i+\lambda_2(-i)=0\]
that imply $\lambda_1=-\lambda_2$ and $\lambda_1=\lambda_2$. 
Hence, it follows that $\lambda_1=\lambda_2=0$ and $(1+i,1-i)$ is linearly independent (if we restrict $\lambda_1,\lambda_2\in\mathbf{R}$).

\subsubsection*{b}
For the constant $\lambda=0-i\in\mathbf{C}$, we can write
\[\lambda(1+i)=-i-i^2=-i-(-1)=1-i.\]
Therefore, $1-i$ is a scalar multiple of $1+i$ and the list $(1+i,1-i)$ is linearly dependent. 

\clearpage

\renewcommand{\thesection}{6}
\section{$v_1-v_2,v_2-v_3,v_3-v_4,v_4$ is linearly independent}
\subsection*{Problem statement}
Suppose $v_1,v_2,v_3,v_4$ is linearly independent in $V$. 
Prove that the list
\[v_1-v_2,v_2-v_3,v_3-v_4,v_4\]
is also linearly independent.

\subsection*{Solution}
Following our reasoning from Exercise 2.A(1), we can write vectors $v\in\operatorname{span}(v_1-v_2,v_2-v_3,v_3-v_4,v_4)$ as
\begin{align*}
    v&=b_1(v_1-v_2)+b_2(v_2-v_3)+b_3(v_3-v_4)+b_4v_4\\
    &=b_1v_1+(b_2-b_1)v_2+(b_3-b_2)v_3+(b_4-b_3)v_4.
\end{align*}
If we set $v=0$, then we can write
\[0=b_1v_1+(b_2-b_1)v_2+(b_3-b_2)v_3+(b_4-b_3)v_4.\]
Given $v_1,v_2,v_3,v_4$ is linearly independent, it follows that\newline $b_1=0,b_2-b_1=0,b_3-b_2=0,b_4-b_3=0$ and thus, $b_1=b_2=b_3=b_4=0$. 
Therefore, the list
\[v_1-v_2,v_2-v_3,v_3-v_4,v_4\]
is also linearly independent.

\clearpage

\renewcommand{\thesection}{8}
\section{Scaled vectors are linearly independent}
\subsection*{Problem statement}
Prove or give a counterexample: 
If $v_1,v_2,\ldots,v_m$ is a linearly independent list of vectors in $V$ and $\lambda\in\mathbf{F}$ with $\lambda\neq 0$, then $\lambda v_1,\lambda v_2,\ldots,\lambda v_m$ is linearly independent.

\subsection*{Solution}
Let's prove it. 
Given $v_1,v_2,\ldots,v_m$ is linearly independent, then for the equation
\[0=a_1v_1+a_2v_2+\cdots+a_m v_m,\]
we have $a_1=a_2=\cdots=a_m=0$ via Definition 2.17 (`linearly independent'). 
In a similar manner, we can show that the list $\lambda v_1,\lambda v_2,\ldots,\lambda v_m$, where $\lambda\in\mathbf{F}$ with $\lambda\neq 0$, is linearly independent by writing
\[0=a_1(\lambda v_1)+a_2(\lambda v_2)+\cdots+a_m(\lambda v_m0))=\lambda(a_1v_1+a_2v_2+\cdots+a_m v_m)\]
and dividing either side of the equation by $\lambda$ (since $\lambda\neq 0$). 
Hence, the proof that $\lambda v_1,\lambda v_2,\ldots,\lambda v_m$ is linearly independent reduces to $v_1,v_2,\ldots,v_m$ being linearly independent.

\clearpage

\renewcommand{\thesection}{9}
\section{Adding linearly independent lists}
\subsection*{Problem statement}
Prove or give a counterexample: 
If $v_1,\ldots,v_m$ and $w_1,\ldots,w_m$ are linearly independent lists of vectors in $V$, then $v_1+w_1,\ldots,v_m+w_m$ is linearly independent.

\subsection*{Solution}
Let's give a counterexample. 
The lists $(1,0),(0,1)$ and $(0,1),(1,0)$ are both linearly independent. 
But the list $(1,0)+(0,1),(0,1)+(1,0)$ equals $(1,1),(1,1)$, which is clearly linearly dependent.

\clearpage

\renewcommand{\thesection}{11}
\section{Appending vectors to linear independence}
\subsection*{Problem statement}
Suppose $v_1,\ldots,v_m$ is linearly independent in $V$ and $w\in V$. 
Show that $v_1,\ldots,v_m,w$ is linearly independent if and only if 
\[w\notin \operatorname{span}(v_1,\ldots,v_m).\]

\subsection*{Solution}
\subsubsection*{First Direction}
Suppose $v_1,\ldots,v_m,w$ is linearly independent. 
If $w\in\operatorname{span}(v_1,\ldots,v_m)$, then there exist $a_1,\ldots,a_m\in\mathbf{F}$, that are not all zero, such that
\[w=a_1v_1+\cdots+a_mv_m\;\;\;\text{and}\;\;\;0=a_1v_1+\cdots+a_mv_m-w.\]
However, the second equation implies that $v_1,\ldots,v_m,w$ is linearly dependent, which is a contradiction. 
Thus, it follows that 
\[w\notin \operatorname{span}(v_1,\ldots,v_m).\]

\subsubsection*{Second Direction}
Suppose $w\notin \operatorname{span}(v_1,\ldots,v_m)$. 
If $v_1,\ldots,v_m,w$ is linearly dependent, then there exist $a_1,\ldots,a_m,a_{m+1}\in\mathbf{F}$, that are not all zero, such that\footnote{Note that $a_{m+1}\neq 0$ since $v_1,\ldots,v_m,w$ is linearly independent.}
\[0=a_1v_1+\cdots+a_mv_m+a_{m+1}w\;\;\;\text{and}\;\;\;w=-\frac{1}{a_{m+1}}(a_1v_1+\cdots+a_mv_m).\]
However, the second equation implies that $w\in \operatorname{span}(v_1,\ldots,v_m)$, which is a contradiction. 
Thus it follows that $v_1,\ldots,v_m,w$ is linearly independent.

\subsubsection*{Thoughts}
The result in this exercise is a restatement of the linear dependence lemma (Theorem 2.21).

\clearpage

\renewcommand{\thesection}{12}
\section{Six polynomials in $\mathcal{P}_4(\mathbf{F})$}
\subsection*{Problem statement}
Explain why there does not exist a list of six polynomials that is linearly independent in $\mathcal{P}_4(\mathbf{F})$.

\subsection*{Solution}
The list of polynomials $1,x,x^2,x^3,x^5$ spans $\mathcal{P}_4(\mathbf{F})$ but is of length five. 
Hence, via Theorem 2.23 (`Length of linearly independent list $\leq$ length of spanning list'), there cannot exist a list of six polynomials that is linearly independent in $\mathcal{P}_4(\mathbf{F})$.

\clearpage

\renewcommand{\thesection}{13}
\section{Four polynomials in $\mathcal{P}_4(\mathbf{F})$}
\subsection*{Problem statement}
Explain why no list of four polynomials spans $\mathcal{P}_4(\mathbf{F})$.

\subsection*{Solution}
The list of polynomials $1,x,x^2,x^3,x^5$ spans $\mathcal{P}_4(\mathbf{F})$ but is of length five. 
Hence, via Theorem 2.23 (`Length of linearly independent list $\leq$ length of spanning list'), there cannot exist a list of four polynomials that spans in $\mathcal{P}_4(\mathbf{F})$.

\clearpage

\renewcommand{\thesection}{17}
\section{A list of linearly dependent polynomials}
\subsection*{Problem statement}
Suppose $p_0,p_1,\ldots,p_m$ are polynomials in $\mathcal{P}_m(\mathbf{F})$ such that $p_j(2)=0$ for each $j$. 
Prove that $p_0,p_1,\ldots,p_m$ is not linearly independent in $\mathcal{P}_m(\mathbf{F})$.

\subsection*{Solution}
First, let's prove that $p_0,p_1,\ldots,p_m$ does not span $\mathcal{P}_m(\mathbf{F})$. 
Consider the polynomial $1$. 
Clearly $1\in\mathcal{P}_m(\mathbf{F})$, yet $1(2)=1$. 
It follows that $1\notin\operatorname{span}(p_0,p_1,\ldots,p_m)$ since $p_j(2)=0$ for each $j$. 
Thus, $p_0,p_1,\ldots,p_m$ does not span $\mathcal{P}_m(\mathbf{F})$. 

Clearly, the list $p_0,p_1,\ldots,p_m$ has a length of $m+1$. 
We must also note that the list $1,z,\ldots,z^m$ spans $\mathcal{P}_m(\mathbf{F})$ and has a length of $m+1$.

Let's perform a proof by contradiction and use the result from Exercise 2.A(11). 
Suppose $p_0,p_1,\ldots,p_m$ is linearly independent. 
Since\newline $1\notin\operatorname{span}(p_0,p_1,\ldots,p_m)$, we can append $1$ to the list to get $1,p_0,p_1,\ldots,p_m$, a list of length $m+2$ that is linearly independent. 
However, via Theorem 2.23 (`Length of linearly independent list $\leq$ length of spanning list'), the list $1,p_0,p_1,\ldots,p_m$ cannot be linearly independent since its length is greater than the list $1,z,\ldots,z^m$, which spans $\mathcal{P}_m(\mathbf{F})$. 
Therefore, $p_0,p_1,\ldots,p_m$ is not linearly independent in $\mathcal{P}_m(\mathbf{F})$.

\end{document}