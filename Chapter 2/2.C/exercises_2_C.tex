\documentclass{article}
\usepackage{graphicx}
\usepackage{amsmath}
\usepackage{hyperref}
\usepackage{epigraph} 

\title{Linear Algebra Done Right\\Solutions to Exercises 2.C}
\author{}
\date{}

\begin{document}

\maketitle

\section{$\operatorname{dim}U=\operatorname{dim}V$ implies $U=V$}
\subsection*{Problem statement}
Suppose $V$ is finite-dimensional and $U$ is a subspace of $V$ such that\newline $\operatorname{dim}U=\operatorname{dim}V$. 
Prove that $U=V$.

\subsection*{Solution}
By virtue of $U$ being a subspace of $V$, we have $U\subset V$. 
We need to show $V\subset U$. 
Suppose $u_1,\ldots,u_m$ is a basis of $U$. 
It follows that\newline $u_1,\ldots,u_m\in V$ and $u_1,\ldots,u_m$ is linearly independent in $V$. 
Via Theorem 2.39 (`Linearly independent list of the right length is a basis'), $u_1,\ldots,u_m$ is a basis of $V$ given that the list is linearly independent and the right length ($m=\operatorname{dim}U=\operatorname{dim}V$. 
Hence, any vector $v\in V$ can be written as
\[v=a_1u_1+\cdots+a_mu_m\]
for some $a_1,\ldots,a_m\in\mathbf{F}$. 
Thus, $v\in U$ and $V\subset U$. 
Therefore, we have $U=V$.

\clearpage

\section{Possible subspaces of $\mathbf{R}^2$}
\subsection*{Problem statement}
Show that the subspaces of $\mathbf{R}^2$ are precisely $\{0\},\mathbf{R}^2,$ and all the lines in $\mathbf{R}^2$ through the origin.

\subsection*{Solution}
Since $(1,0),(0,1)$ is a basis of $\mathbf{R}^2$, via Definition 2.36 (`dimension'),\newline $\operatorname{dim}\mathbf{R}^2=2$. 
Via Theorem 2.38 (`Dimension of a subspace'), the possible dimensions of subspaces of $\mathbf{R}^2$ are $0,1,2$. 
Let's handle each of these cases separately.

For subspaces of $\mathbf{R}^2$ with a dimension of $0$, the only possible subspace is $\{0\}$. 
Remembering Exercise 2.B(1), the empty list $()$ is a basis of $\{0\}$. 
Thus, $\operatorname{dim}\{0\}=0$.

For subspaces of $\mathbf{R}^2$ with a dimension of $1$, we know that the subspace, let's call it $U$, must contain a vector $v\in\mathbf{R}^2$ such that $v\neq0$. 
Given $U$ is closed under scalar multiplication, all vectors $\lambda v$ where $\lambda\in\mathbf{R}$ are in the subspace, $\lambda v\in U$. 
The collection of these vectors $\{\lambda v:\lambda\in\mathbf{R}\}$ can be geometrically represented as a line through the origin. 
The list $v$ is linearly independent and of length $1$. 
Therefore, $v$ is a basis of $U$ and
\[\{\lambda v:\lambda\in\mathbf{R}\}=\operatorname{span}(v)=U.\]
Hence, all subspaces of $\mathbf{R}^2$ with a dimension of $1$ are lines through the origin. 
Furthermore, all lines in $\mathbf{R}^2$ through the origin can be represented as 
\[\{\lambda v:\lambda\in\mathbf{R}\}\]
for some $v\in V$. 
Therefore, all the lines in $\mathbf{R}^2$ through the origin are subspaces of $\mathbf{R}^2$.

Suppose $U$ is a subspace of $\mathbf{R}^2$ with a dimension of $2$. 
Via Exercise 2.C(1), given $\operatorname{dim}\mathbf{R}^2=\operatorname{dim}U$, it follows that $U=\mathbf{R}^2$.

\clearpage

\renewcommand{\thesection}{4}
\section{Explorations on $U=\{p\in\mathcal{P}_4(\mathbf{F}):p(6)=0\}$}
\subsection*{Problem statement}
\begin{enumerate}
    \item[(a)] Let $U=\{p\in\mathcal{P}_4(\mathbf{F}):p(6)=0\}$. Find a basis of $U$.
    \item[(b)] Extend the basis in part (a) to a basis of $\mathcal{P}_4(\mathbf{F})$.
    \item[(c)] Find a subspace $W$ of $\mathcal{P}_4(\mathbf{F})$ such that $\mathcal{P}_4(\mathbf{F})=U\oplus W$.
\end{enumerate}

\subsection*{Solution}
\subsubsection*{a}
Via Theorem 2.38 (`Dimension of a subspace'), it follows that\newline $\operatorname{dim}U\leq\operatorname{dim}\mathcal{P}_4(\mathbf{F})$.
Given $U$ does not contain all the vectors in $\mathcal{P}_4(\mathbf{F})$, for example $1$ since $1(6)\neq 0$, then a basis of $U$ would not span $\mathcal{P}_4(\mathbf{F})$ and it necessarily follows that $\operatorname{dim}U<\operatorname{dim}\mathcal{P}_4(\mathbf{F})$. 
Hence the dimension of $U$ must be $0,1,2,3,4$ but not $5$.

Consider the list of polynomials $(x-6),(x-6)^2,(x-6)^3,(x-6)^4$. 
The list is clearly linearly independent and each polynomial satisfies the condition that $p(6)=0$. 
Thus, we can state $(x-6),(x-6)^2,(x-6)^3,(x-6)^4\in U$. 
Since the list is length $4$ and linearly independent, Theorem 2.39 (`Linearly independent list of the right length is a basis') tells us the list is a basis of $U$. 

\subsubsection*{b}
We have already noted that $1\notin U$. 
Therefore, via Exercise 2.A(11), the list $(x-6),(x-6)^2,(x-6)^3,(x-6)^4,1$ is linearly independent. 
Via Theorem 2.39, it follows that this list is a basis of $\mathcal{P}_4(\mathbf{F})$.

\subsubsection*{c}
The obvious choice of $W$ is 
\[W=\operatorname{span}(1).\]
Given our reasoning in part (b) concerning our basis of $\mathcal{P}_4(\mathbf{F})$, it follows that $\mathcal{P}_4(\mathbf{F})=U + W$. 
Given our reasoning in part (a), namely that $1\notin U$, it follows that $U\cap W=\{0\}$. 
Thus, via Theorem 1.45 (`Direct sum of two subspaces'), it follows that $\mathcal{P}_4(\mathbf{F})=U\oplus W$.

\clearpage

\renewcommand{\thesection}{9}
\section{Dimension of $v_1+w,\ldots,v_m+w$}
\subsection*{Problem statement}
Suppose $v_1,\dots,v_m$ is linearly independent in $V$ and $w\in V$. 
Prove that 
\[\operatorname{dim}\operatorname{span}(v_1+w,\ldots,v_m+w)\geq m-1.\]

\subsection*{Solution}
We can show $v_j-v_1\in\operatorname{span}(v_1+w,\ldots,v_m+w)$ for $j\in\{2,\ldots,m\}$ since $v_j+w-(v_1+w)=v_j-v_1$. Thus, we have 
\[v_2-v_1,\ldots,v_m-v_1\in\operatorname{span}(v_1+w,\ldots,v_m+w)\]
and it follows that
\[\operatorname{span}(v_2-v_1,\ldots,v_m-v_1)\subset\operatorname{span}(v_1+w,\ldots,v_m+w)\]
which implies
\[\operatorname{dim}\operatorname{span}(v_1+w,\ldots,v_m+w)\geq\operatorname{dim}\operatorname{span}(v_2-v_1,\ldots,v_m-v_1).\]
If we can show that $\operatorname{dim}\operatorname{span}(v_2-v_1,\ldots,v_m-v_1)=m-1$, then the desired result follows.

Clearly, 
\[\operatorname{span}(v_1,\dots,v_m)=\operatorname{span}(v_1,v_2-v_1,\ldots,v_m-v_1)\]
since all $v_j$ can be written as $v_j=v_j-v_1+v_1$. 
Thus, we have
\[m=\operatorname{dim}\operatorname{span}(v_1,\dots,v_m)=\operatorname{dim}\operatorname{span}(v_1,v_2-v_1,\ldots,v_m-v_1)\]
and it follows that the list the list $v_1,v_2-v_1,\ldots,v_m-v_1$ is linearly independent via Theorem 2.42 (`Spanning list of the right length is a basis'). 
Therefore, the list $v_2-v_1,\ldots,v_m-v_1$ is also linearly independent and
\[\operatorname{dim}\operatorname{span}(v_2-v_1,\ldots,v_m-v_1)=m-1,\]
implying 
\[\operatorname{dim}\operatorname{span}(v_1+w,\ldots,v_m+w)\geq m-1,\]
which was to be shown.

\clearpage

\renewcommand{\thesection}{11}
\section{Proving $\mathbf{R}^8=U\oplus W$ with Theorem 2.43}
\subsection*{Problem statement}
Suppose that $U$ and $W$ are subspaces of $\mathbf{R}^8$ such that $\operatorname{dim}U=3$, $\operatorname{dim}W=5$, and $U+W=\mathbf{R}^8$. 
Prove that $\mathbf{R}^8=U\oplus W$.

\subsection*{Solution}
Given $U+W=\mathbf{R}^8$, it follows that 
\[\operatorname{dim}(U+W)=\operatorname{dim}\mathbf{R}^8=8.\]
Via Theorem 2.43 (`Dimension of a sum'), we can write
\[\operatorname{dim}(U\cap W)=\operatorname{dim}U+\operatorname{dim}W-\operatorname{dim}(U+W)=3+5-8=0,\]
which implies $U\cap W=\{0\}$. 
Therefore, via Theorem 1.45 (`Direct sum of two subspaces'), we can state $\mathbf{R}^8=U\oplus W$.

\clearpage

\renewcommand{\thesection}{12}
\section{Proving $U\cap W\neq\{0\}$ with Theorem 2.43}
\subsection*{Problem statement}
Suppose $U$ and $W$ are both five-dimensional subspaces of $\mathbf{R}^9$. 
Prove that\newline $U\cap W\neq\{0\}$.

\subsection*{Solution}
Given $U$ and $W$ are subspaces of $\mathbf{R}^9$, then it necessarily follows that $U+W\subset\mathbf{R}^9$ and
\[9=\dim\mathbf{R}^9\geq \dim(U+W).\]
Via Theorem 2.43 (`Dimension of a sum'), we can write
\[9\geq\operatorname{dim}U+\operatorname{dim}W-\operatorname{dim}(U\cap W)=5+5-\operatorname{dim}(U\cap W)\]
and by rearranging terms, we get
\[\operatorname{dim}(U\cap W)\geq 1.\]
This implies that $U\cap W\neq\{0\}$.

\clearpage

\renewcommand{\thesection}{16}
\section{$\dim U_1\oplus\cdots\oplus U_m = \dim U_1 +\cdots+ \dim U_m$}
\subsection*{Problem statement}
Suppose $U_1,\ldots,U_m$ are finite-dimensional subspaces of $V$ such that $U_1+\cdots + U_m$ is a direct sum. 
Prove that $U_1\oplus\cdots\oplus U_m$ is finite-dimensional and 
\[\dim U_1\oplus\cdots\oplus U_m = \dim U_1 +\cdots+ \dim U_m.\]

\subsection*{Solution}
Let's prove this via an algorithm. 

\subsubsection*{Step 1}
Given $U_1+\cdots + U_m$ is a direct sum, it necessarily follows that $U_1+U_2$ is a direct sum and $U_1\cap U_2=\{0\}$. 
Hence, via Theorem 2.43 (`Dimension of a sum'), we have
\[\dim(U_1\oplus U_2)=\dim U_1+\dim U_2.\]

\subsubsection*{Step j}
Given $U_1+\cdots + U_m$ is a direct sum, it necessarily follows that\newline $(U_1+\cdots+U_j)+U_{j+1}$ is a direct sum and $(U_1+\cdots+U_j)\cap U_{j+1}=\{0\}$. 
Hence, via Theorem 2.43, we have
\begin{align*}
    \dim((U_1\oplus\cdots\oplus U_j)\oplus U_{j+1})&=\dim (U_1\oplus\cdots\oplus U_j) +\dim U_{j+1}\\
    &=(\dim U_1+\cdots+\dim U_j)+\dim U_{j+1}\\
    &=\dim U_1+\cdots+\dim U_{j+1}.
\end{align*}

\subsubsection*{Step m-1}
The algorithm terminates at this step with the desired result
\[\dim(U_1\oplus U_2)=\dim U_1+\dim U_2,\]
which further implies that $U_1\oplus\cdots\oplus U_m$ is finite-dimensional.

\clearpage

\renewcommand{\thesection}{17}
\section{Theorem 2.43 fails for three subspaces}
\subsection*{Problem statement}
You might guess, by analogy with the formula for the number of elements in the union of three subset of a finite set, that if $U_1,U_2,U_3$ are subspaces of a finite-dimensional vector space then
\begin{align*}
    \dim(U_1+&U_2+U_3)\\
    &=\dim U_1+\dim U_2 + \dim U_3\\
    &\;\;\;\;-\dim(U_1\cap U_2)-\dim(U_1\cap U_3)-\dim(U_2\cap U_3)\\
    &\;\;\;\;+\dim(U_1\cap U_2\cap U_3).
\end{align*}
Prove this or give a counterexample.

\subsection*{Solution}
Let's give a counterexample. 

For subspaces $U_1,U_2,U_3$ of $\mathbf{R}^2$, let's define them as
\begin{align*}
    U_1&=\{ (x,0)\in\mathbf{R}^2 :x\in\mathbf{R} \},\\
    U_2&=\{ (y,y)\in\mathbf{R}^2 :y\in\mathbf{R} \},\\
    U_3&=\{ (0,z)\in\mathbf{R}^2 :z\in\mathbf{R} \}.
\end{align*}
Clearly, their individual dimensions are
\[\dim U_1=\dim U_2=\dim U_3=1,\]
the dimensions of their sum is
\[\dim(U_1+U_2+U_3)=\dim \mathbf{R}^2=2,\]
and the dimensions of their intersections are
\[\dim(U_1\cap U_2)=\dim(U_1\cap U_3)=\dim(U_2\cap U_3)=\dim(U_1\cap U_2\cap U_3)=0.\]
Hence, substituting all our values into the formula from the \textbf{problem statement}, we have
\[2=1+1+1-0-0-0+0=3,\]
which is a contradiction.

\end{document}