\documentclass{article}
\usepackage{graphicx}
\usepackage{amsmath}
\usepackage{hyperref}
\usepackage{epigraph} 

\title{Linear Algebra Done Right\\Solutions to Exercises 2.C}
\author{}
\date{}

\begin{document}

\maketitle

\section{$\operatorname{dim}U=\operatorname{dim}V$ implies $U=V$}
\subsection*{Problem statement}
Suppose $V$ is finite-dimensional and $U$ is a subspace of $V$ such that\newline $\operatorname{dim}U=\operatorname{dim}V$. 
Prove that $U=V$.

\subsection*{Solution}
By virtue of $U$ being a subspace of $V$, we have $U\subset V$. 
We need to show $V\subset U$. 
Suppose $u_1,\ldots,u_m$ is a basis of $U$. 
It follows that\newline $u_1,\ldots,u_m\in V$ and $u_1,\ldots,u_m$ is linearly independent in $V$. 
Via Theorem 2.39 (`Linearly independent list of the right length is a basis'), $u_1,\ldots,u_m$ is a basis of $V$ given that the list is linearly independent and the right length ($m=\operatorname{dim}U=\operatorname{dim}V$. 
Hence, any vector $v\in V$ can be written as
\[v=a_1u_1+\cdots+a_mu_m\]
for some $a_1,\ldots,a_m\in\mathbf{F}$. 
Thus, $v\in U$ and $V\subset U$. 
Therefore, we have $U=V$.

\clearpage

\section{Possible subspaces of $\mathbf{R}^2$}
\subsection*{Problem statement}
Show that the subspaces of $\mathbf{R}^2$ are precisely $\{0\},\mathbf{R}^2,$ and all the lines in $\mathbf{R}^2$ through the origin.

\subsection*{Solution}
Since $(1,0),(0,1)$ is a basis of $\mathbf{R}^2$, via Definition 2.36 (`dimension'),\newline $\operatorname{dim}\mathbf{R}^2=2$. 
Via Theorem 2.38 (`Dimension of a subspace'), the possible dimensions of subspaces of $\mathbf{R}^2$ are $0,1,2$. 
Let's handle each of these cases separately.

For subspaces of $\mathbf{R}^2$ with a dimension of $0$, the only possible subspace is $\{0\}$. 
Remembering Exercise 2.B(1), the empty list $()$ is a basis of $\{0\}$. 
Thus, $\operatorname{dim}\{0\}=0$.

For subspaces of $\mathbf{R}^2$ with a dimension of $1$, we know that the subspace, let's call it $U$, must contain a vector $v\in\mathbf{R}^2$ such that $v\neq0$. 
Given $U$ is closed under scalar multiplication, all vectors $\lambda v$ where $\lambda\in\mathbf{R}$ are in the subspace, $\lambda v\in U$. 
The collection of these vectors $\{\lambda v:\lambda\in\mathbf{R}\}$ can be geometrically represented as a line through the origin. 
The list $v$ is linearly independent and of length $1$. 
Therefore, $v$ is a basis of $U$ and
\[\{\lambda v:\lambda\in\mathbf{R}\}=\operatorname{span}(v)=U.\]
Hence, all subspaces of $\mathbf{R}^2$ with a dimension of $1$ are lines through the origin. 
Furthermore, all lines in $\mathbf{R}^2$ through the origin can be represented as 
\[\{\lambda v:\lambda\in\mathbf{R}\}\]
for some $v\in V$. 
Therefore, all the lines in $\mathbf{R}^2$ through the origin are subspaces of $\mathbf{R}^2$.

Suppose $U$ is a subspace of $\mathbf{R}^2$ with a dimension of $2$. 
Via Exercise 2.C(1), given $\operatorname{dim}\mathbf{R}^2=\operatorname{dim}U$, it follows that $U=\mathbf{R}^2$.


\end{document}