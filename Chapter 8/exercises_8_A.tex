\documentclass{article}
\usepackage{graphicx}
\usepackage{amsmath}
\usepackage{hyperref}
\usepackage{epigraph}
\usepackage{csquotes}
\usepackage{mathtools}

\title{Linear Algebra Done Right\\Solutions to Exercises 8.A}
\author{}
\date{}

\begin{document}

\maketitle

\section{Generalized eigenvectors of $T(w,z)=(z,0)$}
\subsection*{Problem statement}
Define $T\in\mathcal{L}(\mathbf{C}^2)$ by
\[T(w,z)=(z,0).\]
Find all generalized eigenvectors of $T$.

\subsection*{Solution}
First, let's find the eigenvalues. We have
\begin{align*}
    (z,0)&=T(w,z)\\
    &=\lambda (w,z)
\end{align*}
implying $z=0$ and $\lambda=0$. Hence our only eigenvalue is $\lambda=0$.

To find the generalized eigenvectors, we can use Theorem 8.11 (`Description of generalized eigenspaces') and find a basis of $\operatorname{null}(T-\lambda I)^{\operatorname{dim}V}$. We have
\[(T-0I)^2(w,z)=T^2(w,z)=(0,0).\]
Hence $(1,0),(0,1)\in\operatorname{null}(T-\lambda I)^{\operatorname{dim}V}=G(\lambda,T)$ where $\lambda=0$. Therefore $(1,0),(0,1)$ are all the generalized eigenvetors of $T$.

If we want to be pedantic, we could use Theorem 8.23 (`A basis of generalized eigenvectors') to say that $T$ has exactly two generalized eigenvectors. Since we found two eigenvectors, then we know that we've found all of them.

\clearpage

\section{Generalized eigenspaces of $T(w,z)=(-z,w)$}
\subsection*{Problem statement}
Define $T\in\mathcal{L}(\mathbf{C}^2)$ by
\[T(w,z)=(-z,w).\]
Find the generalized eigenspaces corresponding to the distinct eigenvalues of $T$.

\subsection*{Solution}
First, let's find the eigenvalues of $T$. Note that $T$ is the same operator in Example 5.8. But for the sake of completeness, we have
\begin{align*}
    (-z,w)&=T(w,z)\\
    &=\lambda (w,z)
\end{align*}
with $-z=\lambda w$ and $w=\lambda z$. Hence it follows that $-z=\lambda^2z$ and $-1=\lambda^2$. Therefore the eigenvalues of $T$ are $\lambda=i$ and $\lambda=-i$.

To find the generalized eigenspaces of $T$, we can use Theorem 8.11 (`Description of generalized eigenspaces') to find $G(\lambda,T)=\operatorname{null}(T-\lambda I)^{\operatorname{dim}V}$.

$\lambda=i$: To compute $\operatorname{null}(T-i I)^{2}$, we have
\begin{align*}
    (T-i I)^{2}(w,z)&=(T^2-2iT-I)(w,z)\\
    &=(-w,-z)+(2iz,-2iw)+(-w,-z)\\
    &=(-2(w-iz),-2(z+iw)).
\end{align*}
It follows that $(x,-ix)\in\operatorname{null}(T-i I)^{2}$ for $x\in\mathbf{C}$ since 
\[(T-i I)^{2}(x,-ix)=(-2(x-i(-ix)),-2(-ix+ix))=(0,0)\]
Hence, we have $G(i,T)=\{(x,-ix):x\in\mathbf{C}\}$.

$\lambda=-i$: To compute $\operatorname{null}(T+i I)^{2}$, we have
\begin{align*}
    (T+i I)^{2}(w,z)&=(T^2+2iT-I)(w,z)\\
    &=(-w,-z)+(-2iz,2iw)+(-w,-z)\\
    &=(-2(w+iz),-2(z-iw)).
\end{align*}
It follows that $(x,ix)\in\operatorname{null}(T+i I)^{2}$ for $x\in\mathbf{C}$ since 
\[(T+i I)^{2}(x,ix)=(-2(x+i^2x),-2(ix-ix))=(0,0)\]
Hence, we have $G(-i,T)=\{(x,ix):x\in\mathbf{C}\}$.

\clearpage

\section{$T$ is invertible. Prove $G(\lambda,T)=G(\frac{1}{\lambda},T^{-1})$.}
\subsection*{Problem statement}
Suppose $T\in\mathcal{L}(V)$ is invertible. Prove that $G(\lambda,T)=G(\frac{1}{\lambda},T^{-1})$ for every $\lambda\in\mathbf{F}$ with $\lambda\neq 0$.

\subsection*{Solution}
Let's consider two separate cases: $\lambda$ is an eigenvalue and $\lambda$ is not an eigenvalue.\footnote{My proof for ``$\lambda$ is an eigenvalue'' also applies to ``$\lambda$ is not an eigenvalue'', but I'm including it to use in Exercise 5.C(4).}

\subsubsection*{$\lambda$ is not an eigenvalue}
If $\lambda$ is not an eigenvalue, then via Theorem 5.6 (`Equivalent conditions to be an eigenvalue'), $T-\lambda I$ is in injective. Hence, we have
\[\{0\}=\operatorname{null}(T-\lambda I)^0=\operatorname{null}(T-\lambda I)^1\]
and via Theorem 8.3 (`Equality in the sequence of null spaces') and Theorem 8.11 (`Description of generalized eigenspaces'), we also have
\[\{0\}=\operatorname{null}(T-\lambda I)^1=\operatorname{null}(T-\lambda I)^{\operatorname{dim}V}=G(\lambda,T).\]

Note that Exercise 5.C(9) allows us to state that if $T$ is invertible, then $E(\lambda,T)=E(\frac{1}{\lambda},T^{-1})$ for every $\lambda\in\mathbf{F}$ with $\lambda\neq 0$. It now follows that 
\[\operatorname{null}(T^{-1}-\frac{1}{\lambda}I)^0=\{0\}=\operatorname{null}(T-\lambda I)^1=\operatorname{null}(T^{-1}-\frac{1}{\lambda}I)^1\]
and, like before, Theorem 8.3 and Theorem 8.11, we have
\[\{0\}=\operatorname{null}(T^{-1}-\lambda I)^1=\operatorname{null}(T^{-1}-\lambda I)^{\operatorname{dim}V}=G(\frac{1}{\lambda},T^{-1}).\]
Hence, when $\lambda$ is not an eigenvalue, then $G(\lambda,T)=G(\frac{1}{\lambda},T^{-1})$.

\subsubsection*{$\lambda$ is an eigenvalue}
Suppose $\lambda$ is an eigenvalue and $v\in G(\lambda,T)$. Thus there is a positive integer $j$ such that $v\in\operatorname{null}(T-\lambda I)^j$ and we can write 
\[(T-\lambda I)^jv=0.\]
Applying $(-\frac{T^{-1}}{\lambda})^{j}$ to both sides, we have
\begin{align*}
    (-\frac{T^{-1}}{\lambda})^{j}(T-\lambda I)^jv&=(-\frac{T^{-1}}{\lambda})^{j}0 \\
    (-\frac{T^{-1}}{\lambda}T+\lambda\frac{T^{-1}}{\lambda}I)^jv&=0\\
    (T^{-1}-\frac{1}{\lambda}I)^jv&=0
\end{align*}
where $T^{-1}T=I$ via the definition of invertible operators. Hence, $v\in\operatorname{null}(T^{-1}-\frac{1}{\lambda}I)^j$ and $v\in G(\frac{1}{\lambda},T^{-1})$. It follows that $G(\lambda,T)\subset G(\frac{1}{\lambda},T^{-1})$.

To complete the equality and show $G(\frac{1}{\lambda},T^{-1})\subset G(\lambda,T)$, we can repeat the process above and instead apply $(-\lambda T)^j$ to both sides.

\clearpage

\section{Suppose $\alpha\neq\beta$. Prove $G(\alpha,T)\cap G(\beta,T)=\{0\}$.}
\subsection*{Problem statement}
Suppose $T\in\mathcal{L}(V)$ and $\alpha,\beta\in\mathbf{F}$ with $\alpha\neq\beta$. Prove that
\[G(\alpha,T)\cap G(\beta,T)=\{0\}.\]

\subsection*{Solution}
Following our proof for \textbf{$\lambda$ is not an eigenvalue} in Exercise 8.A(3), if $\lambda$ is not an eigenvalue of $T$, then $G(\lambda,T)=\{0\}$. Hence, if either $\alpha$ or $\beta$ is not an eigenvalue, then
\[G(\alpha,T)\cap G(\beta,T)=\{0\}.\]

Things only get interesting if both $\alpha$ and $\beta$ are eigenvalues. 

Suppose $\alpha$ and $\beta$ are distinct eigenvalues of $T$ and $v\in G(\alpha,T)\cap G(\beta,T)$. Now let's do a simple proof by contradiction. Suppose $v\neq 0$. It follows that $v$ is a generalized eigenvector corresponding to both $\alpha$ and $\beta$. Hence, the list $v,v$ of generalized eigenvectors corresponding to the list $\alpha,\beta$ of eigenvalues of $T$ is linearly dependent. This is a contradiction of Theorem 8.13 (`Linearly independent generalized eigenvectors'); thus, $v=0$. Therefore, we have
\[G(\alpha,T)\cap G(\beta,T)=\{0\}.\]

\clearpage

\section{If $v\in\operatorname{null}T^m$, then $v,Tv,T^2v,\ldots,T^{m-1}v$ is l.i.}
\subsection*{Problem statement}
Suppose $T\in\mathcal{L}(V)$, $m$ is a positive integer, and $v\in V$ is such that $T^{m-1}v\neq 0$ but $T^mv=0$. Prove that
\[v,Tv,T^2v,\ldots,T^{m-1}v\]
is linearly independent.

\subsection*{Solution}
Suppose $a_0,a_1,\ldots,a_{m-1}\in\mathbf{F}$ are such that 
\[0=a_0v+a_1Tv+a_2T^2v+\cdots+a_{m-1}T^{m-1}v.\]
Now apply $T^{m-1}$ to both sides to get
\begin{align*}
    T^{m-1}(0)&=a_0T^{m-1}v+a_1T^mv+a_2T(T^mv)+\cdots+a_{m-1}T^{m-1}(T^mv)\\
    0&=a_0T^{m-1}v+a_1(0)+a_2T(0)+\cdots+a_{m-1}T^{m-1}(0)\\
    0&=a_0T^{m-1}v+a_1(0)+a_2(0)+\cdots+a_{m-1}(0)\\
    0&=a_0T^{m-1}v
\end{align*}
Since $T^{m-1}v\neq 0$, it follows that $a_0=0$.

Applying $T^{m-2}$ to both sides, the logic above, and our conclusion that $a_0=0$, allows us to deduce $a_1=0$. Continuing in this fashion, $a_j=0$ for each $j\in\{0,1,\ldots,m-1\}$, which implies $v,Tv,T^2v,\ldots,T^{m-1}v$ is linearly independent.

\clearpage

\section{No square root for $T(z_1,z_2,z_3)=(z_2,z_3,0)$}
\subsection*{Problem statement}
Suppose $T\in\mathcal{L}(\mathbf{C}^3)$ is defined by $T(z_1,z_2,z_3)=(z_2,z_3,0)$. Prove that $T$ has no square root. More precisely, prove that there does not exist $S\in\mathcal{L}(\mathbf{C}^3)$ such that $S^2=T$.

\subsection*{Solution}
Let's first note that $T$ is nilpotent since 
\[T^3(z_1,z_2,z_3)=(0,0,0).\]
To prove that $T$ has no square root, we can do a proof by contradiction.

Suppose there exists $S\in\mathcal{L}(\mathbf{C}^3)$ such that $S^2=T$. It follows that since $T$ is nilpotent, $T^3=(S^2)^3=0$ and $S$ is nilpotent. However, Theorem 8.18 (`Nilpotent operator raised to dimension of domain is 0') implies that $S^3=0$, and since 
\[S^4(z_1,z_2,z_3)=T^2(z_1,z_2,z_3)=(z_3,0,0),\] 
we have a contradiction. Hence, there does not exist $S\in\mathcal{L}(\mathbf{C}^3)$ such that $S^2=T$.

\subsection*{Notes}
This result implies that for nilpotent operators $N\in\mathcal{L}(V)$ such that 
\[N^{\operatorname{dim}V-1}\neq0,\]
there does not exist $S\in\mathcal{L}(V)$ such that $S^2=N$.

\clearpage

\section{For nilpotent $N\in\mathcal{L}(V)$, $0$ is only eigenvalue}
\subsection*{Problem statement}
Suppose $N\in\mathcal{L}(V)$ is nilpotent. Prove that $0$ is the only eigenvalue of $N$.

\subsection*{Solution}
The proof follows immediately from Theorem 8.19 (`Matrix of a nilpotent operator'), whereby all nilpotent operators have an upper-triangular matrix with only $0$'s on the diagonal, and Theorem 5.32 (`Determination of eigenvalues from upper-triangular matrix'), whereby the eigenvalues of an operator are precisely the entries on the diagonal of the upper-triangular matrix of that operator.


\clearpage

\section{Set of nilpotent operators is not a subspace}
\subsection*{Problem statement}
Prove or give a counterexample: The set of nilpotent operators on $V$ is a subspace of $\mathcal{L}(V)$.

\subsection*{Solution}
Define the operators $T_1,T_2\in\mathcal{L}(\mathbf{C}^2)$ as 
\[T_1(z_1,z_2)=(z_2,0)\;\;\;\text{and}\;\;\;T_2(z_1,z_2)=(0,z_1).\]
Clearly $T_1$ and $T_2$ are both nilpotent, but we can show $T_1+T_2$ is not nilpotent. For $(z_1,z_2)\in\mathcal{L}(\mathbf{C}^2)$, we can write
\begin{align*}
    (T_1+T_2)^2(z_1,z_2)&=(T_1+T_2)((T_1+T_2)(z_1,z_2))\\
    &=(T_1+T_2)((z_2,0)+(0,z_1))\\
    &=(T_1+T_2)(z_2,z_1)\\
    &=(z_1,0)+(0,z_2)\\
    &=(z_1,z_2).
\end{align*}
Hence, via Theorem 8.18 (`Nilpotent operator raised to dimension of domain is 0'), $T_1+T_2$ is not nilpotent. Thus, the set of nilpotent operators is not a subspace.


\end{document}