\documentclass{article}
\usepackage{graphicx}
\usepackage{amsmath}
\usepackage{amssymb}
\usepackage{hyperref}
\usepackage{epigraph}
\usepackage{csquotes}
\usepackage{mathtools}

\title{Linear Algebra Done Right\\Solutions to Exercises 8.C}
\author{}
\date{}

\begin{document}

\maketitle

\section{Prove $(T-3I)^2(T-5I)^2(T-8I)^2=0$}
\subsection*{Problem statement}
Suppose $T\in\mathcal{L}(\mathbf{C}^4)$ is such that the eigenvalues of $T$ are $3,5,8$. 
Prove that $(T-3I)^2(T-5I)^2(T-8I)^2=0$.

\subsection*{Solution}
Given $3,5,8$ are the only eigenvalues of $T$ and $\operatorname{dim}\mathbf{C}^4=4$, Theorem 8.26 (`Sum of the multiplicities equals $\operatorname{dim}V$') implies that the multiplicity of one of the eigenvalues must be $2$ and the other multiplicities are $1$. 
Hence, the characteristic polynomial must be one of
\begin{align*}
    q(z)&=(z-3)^2(z-5)(z-8),\\
    q(z)&=(z-3)(z-5)^2(z-8),\\
    q(z)&=(z-3)(z-5)(z-8)^2.
\end{align*}

Since the characteristic polynomial is a multiple of the minimal polynomial (Theorem 8.48) and $(z-3)^2(z-5)^2(z-8)^2$ is a polynomial multiple of all three possible characteristic polynomials, it follows that 
\newline
$(z-3)^2(z-5)^2(z-8)^2$ is a polynomial multiple of the minimal polynomial. 
Thus, Theorem 8.46 (`$q(T)=0$ implies $q$ is a multiple of the minimal polynomial') implies that $(T-3I)^2(T-5I)^2(T-8I)^2=0$.

\clearpage

\section{Prove $(T-5I)^{n-1}(T-6I)^{n-1}=0$}
\subsection*{Problem statement}
Suppose $V$ is a complex vector space. 
Suppose $T\in\mathcal{L}(V)$ is such that $5$ and $6$ are eigenvalues of $T$ and that $T$ has no other eigenvalues. 
Prove that 
\newline
$(T-5I)^{n-1}(T-6I)^{n-1}=0$, where $n=\operatorname{dim}V$.

\subsection*{Solution}
This proof will resemble our proof for Exercise 8.C(1). 

Given $5$ and $6$ are the only eigenvalues of $T$ and $\operatorname{dim}V=n$, Theorem 8.26 (`Sum of the multiplicities equals $\operatorname{dim}V$') implies that the multiplicities of eigenvalues $5$ and $6$ must add up to $n$. 
Hence, the characteristic polynomial must be one of 
\begin{gather*} 
    q(z)=(z-5)^{n-1}(z-6)^1,\\
    q(z)=(z-5)^{n-2}(z-6)^2,\\
    \vdots\\
    q(z)=(z-5)^{2}(z-6)^{n-2},\\
    q(z)=(z-5)^{1}(z-6)^{n-1}.
\end{gather*}

Since the characteristic polynomial is a multiple of the minimal polynomial (Theorem 8.48) and $(z-5)^{n-1}(z-6)^{n-1}$ is a polynomial multiple of all possible characteristic polynomials, it follows that $(z-5)^{n-1}(z-6)^{n-1}$ is a polynomial multiple of the minimal polynomial. 
Thus, Theorem 8.46 (`$q(T)=0$ implies $q$ is a multiple of the minimal polynomial') implies that $(T-5I)^{n-1}(T-6I)^{n-1}=0$.

\clearpage

\section{Characteristic polynomial $(z-7)^2(z-8)^2$}
\subsection*{Problem statement}
Give an example of an operator on $\mathbf{C}^4$ whose characteristic polynomial equals $(z-7)^2(z-8)^2$.

\subsection*{Solution}
An operator whose characteristic polynomial equals $(z-7)^2(z-8)^2$ implies that the operator has eigenvalues $7$ and $8$ each with multiplicities of 2. 
Consider the operator $T\in\mathcal{L}(\mathbf{C}^4)$ defined by
\[T(z_1,z_2,z_3,z_4)=(7z_1,7z_2,8z_3,8z_4).\]
The eigenvalues of $T$ are clearly $7$ and $8$, where $7$ has eigenvectors $(1,0,0,0),(0,1,0,0)$ and $8$ has eigenvectors $(0,0,1,0),(0,0,0,1)$. 

These eigenvectors form a basis of $\mathbf{C}^4$, therefore, by Exercise 8.B(5), every generalized eigenvector of $T$ is an eigenvector of $T$. 
It follows that
\[\operatorname{dim}G(7,T)=\operatorname{dim}G (8, T) = 2\]
and the multiplicities of the eigenvalues $7$ and $8$ are 2. 
Hence, the characteristic polynomial of $T$ is $(z-7)^2(z-8)^2$.

\clearpage

\section{Minimal polynomial $(z-1)(z-5)^2$}
\subsection*{Problem statement}
Give an example of an operator on $\mathbf{C}^4$ whose characteristic polynomial equals $(z-1)(z-5)^3$ and whose minimal polynomial equals $(z-1)(z-5)^2$.

\subsection*{Solution}
Consider the operator $T\in\mathcal{L}(\mathbf{C}^4)$ defined by
\begin{equation*}
T(z_1,z_2,z_3,z_4)=(z_1,5z_2+z_3,5z_3,5z_4).
\end{equation*}
The matrix of $T$ with respect to the standard basis is
\begin{equation*}
\begin{pmatrix}
1 & 0 & 0 & 0\\
0 & 5 & 1 & 0\\
0 & 0 & 5 & 0\\
0 & 0 & 0 & 5
\end{pmatrix}.
\end{equation*}
Via Theorem 5.32 (`Determination of eigenvalues from upper-triangular matrix'), the eigenvalues of $T$ are $1$ and $5$. 
To assess their multiplicities, let's note that $(1,0,0,0)$ is an eigenvector corresponding to $1$ and $(0,1,0,0),(0,0,0,1)$ are eigenvectors corresponding to $5$. 

To show that the multiplicity of the eigenvalue $1$ is $1$ and $5$ is $3$, we can show that $(0,0,1,0)$ is a generalized eigenvector via
\[(T-5I)^2(0,0,1,0)=(T-5I)(0,1,0,0)=(0,5,0,0)-(0,5,0,0)=0.\]
Hence, via Theorem 8.21 (`Description of operators on complex vector spaces'), it follows that $\operatorname{dim}G(1,T)=1$ and $\operatorname{dim}G(5,T)=3$. 
Thus, by the definition of characteristic polynomials (Definition 8.34), $(z-1)(z-5)^3$ is the characteristic polynomial of $T$.

To show that the minimal polynomial of $T$ equals $(z-1)(z-5)^2$, we can use a similar method as used in Example 8.50 and Example 8.51 to compute both $(T-I)(T-5I)$ and $(T-1)(T-5)^2$. 
Focusing on $(T-I)(T-5I)$ first, for $(z_1,z_2,z_3,z_4)\in\mathbf{C}^4$ we can write
\begin{align*}
    (T-I)(T-5I)(z_1,z_2,z_3,z_4)=(T-I)(-4z_1,z_3,0,0)=(0,4z_3,0,0)
\end{align*}
and hence, $(T-I)(T-5I)\neq 0$. 
Shifting our attention to $(T-1)(T-5)^2$, for $(z_1,z_2,z_3,z_4)\in\mathbf{C}^4$ we can write
\begin{align*}
    (T-I)(T-5I)^2(z_1,z_2,z_3,z_4)&=(T-I)(T-5I)(-4z_1,z_3,0,0)\\
    &=(T-I)(16z_1,0,0,0)\\
    &=(0,0,0,0)
\end{align*}
and hence, $(T-1)(T-5)^2 =0$. Therefore, the minimal polynomial of $T$ equals $(z-1)(z-5)^2$.

\clearpage

\section{Same characteristic and minimal polynomials}
\subsection*{Problem statement}
Give an example of an operator on $\mathbf{C}^4$ whose characteristic polynomial and minimal polynomial equals $z(z-1)^2(z-3)$.

\subsection*{Solution}
Consider the operator $T\in\mathcal{L}(\mathbf{C}^4)$ defined by
\begin{equation*}
T(z_1,z_2,z_3,z_4)=(0,z_2+z_3,z_3,3z_4).
\end{equation*}
The matrix of $T$ with respect to the standard basis is
\begin{equation*}
\begin{pmatrix}
0 & 0 & 0 & 0\\
0 & 1 & 1 & 0\\
0 & 0 & 1 & 0\\
0 & 0 & 0 & 3
\end{pmatrix}.
\end{equation*}
Via Theorem 5.32 (`Determination of eigenvalues from upper-triangular matrix'), the eigenvalues of $T$ are $0,1,3$. 
To assess their multiplicities, let's note that $(1,0,0,0)$ is an eigenvector corresponding to $0$, $(0,1,0,0)$ is an eigenvector corresponding to $1$, and $(0,0,0,1)$ is an eigenvector corresponding to $3$. 

To show that the multiplicity of the eigenvalue $1$ is $2$, we can show that $(0,0,1,0)$ is a generalized eigenvector via
\[(T-I)^2(0,0,1,0)=(T-I)(0,1,0,0)=(0,1,0,0)-(0,1,0,0)=0.\]
Hence, via Theorem 8.21 (`Description of operators on complex vector spaces'), it follows that $\operatorname{dim}G(0,T)=1$, $\operatorname{dim}G(1,T)=2$, and $\operatorname{dim}G(1,T)=1$. 
Thus, by the definition of characteristic polynomials (Definition 8.34), $z(z-1)^2(z-3)$ is the characteristic polynomial of $T$.

To show that $z(z-1)^2(z-3)$ is also the minimal polynomial, we need only show that $T(T-I)(T-3I)\neq 0$, since it is the only polynomial with degree less than the characteristic polynomial and with only zeros of the eigenvalues of $T$ (following from Theorem 8.49 (`Eigenvalues are the zeros of the minimal polynomial')). 
For $(z_1,z_2,z_3,z_4)\in\mathbf{C}^4$ we can write
\begin{align*}
    T(T-I)(T-3I)(z_1,z_2,z_3,z_4)&=T(T-I)(-3z_1,-2z_2+z_3,-2z_3,0)\\
    &=T(3z_1,-2z_3,0,0)\\
    &=(0,-2z_3,0,0)
\end{align*}
and hence, $T(T-I)(T-3I)\neq 0$. 

\clearpage

\section{Tweak on Exercise 8.C(5)}
\subsection*{Problem statement}
Give an example of an operator on $\mathbf{C}^4$ whose characteristic polynomial equals $z(z-1)^2(z-3)$ and minimal polynomial equals $z(z-1)(z-3)$.

\subsection*{Solution}
We can slightly tweak our solution for Exercise 8.C(5). 
Consider the operator $T\in\mathcal{L}(\mathbf{C}^4)$ defined by
\begin{equation*}
T(z_1,z_2,z_3,z_4)=(0,z_2,z_3,3z_4).
\end{equation*}
The matrix of $T$ with respect to the standard basis is
\begin{equation*}
\begin{pmatrix}
0 & 0 & 0 & 0\\
0 & 1 & 0 & 0\\
0 & 0 & 1 & 0\\
0 & 0 & 0 & 3
\end{pmatrix}.
\end{equation*}
The eigenvalues of $T$ are clearly $0,1,3$, where $0$ has eigenvector $(1,0,0,0)$, $1$ has eigenvectors $(0,1,0,0),(0,0,1,0)$, and $3$ has eigenvectors $(0,0,0,1)$. 

These eigenvectors form a basis of $\mathbf{C}^4$, therefore, by Exercise 8.B(5), every generalized eigenvector of $T$ is an eigenvector of $T$. 
It follows that the multiplicities of the eigenvalues $0,1,3$ are $1,2,1$ respectively. Hence, the characteristic polynomial of $T$ is $z(z-1)^2(z-3)$.

Unlike Exercise 8.C(5), we can show that $T(T-I)(T-3I)=0$. 
For $(z_1,z_2,z_3,z_4)\in\mathbf{C}^4$ we can write
\begin{align*}
    T(T-I)(T-3I)(z_1,z_2,z_3,z_4)&=T(T-I)(-3z_1,-2z_2,-2z_3,0)\\
    &=T(3z_1,0,0,0)\\
    &=(0,0,0,0).
\end{align*}
Hence, the minimal polynomial of $T$ equals $z(z-1)(z-3)$.

\clearpage

\section{Characteristic polynomial of $P^2=P$}
\subsection*{Problem statement}
Suppose $V$ is a complex vector space. Suppose $T\in\mathcal{L}(V)$ is such that \newline
$P^2=P$. 
Prove that the characteristic polynomial of $P$ is $z^m(z-1)^n$, where \newline
$m=\operatorname{dim}\operatorname{null}P$ and $n=\operatorname{dim}\operatorname{range}P$.

\subsection*{Solution}
First, let's show that $0$ and $1$ are the only possible eigenvalues. Suppose $\lambda$ is an eigenvalue of $P$. 
It follows that there exists an eigenvector $v\in V$ such that $P=\lambda v$. Given that $P^2=P$, we can write
\begin{equation}
    \lambda v=Pv=P^2v=\lambda^2v
\end{equation}
implying that $\lambda=0$ or $\lambda=1$.

If we can prove that the multiplicity of $\lambda=0$ is $\operatorname{dim}\operatorname{null}P$, then Theorem 8.26 (`Sum of the multiplicities equals $\operatorname{dim}V$') implies that the multiplicity of $\lambda=1$ is $\operatorname{dim}V-\operatorname{dim}\operatorname{null}P$, which equals $\operatorname{dim}\operatorname{range}P$ via our best friend, the Fundamental Theorem of Linear Maps.

We know that $\operatorname{dim}G(0,P)\geq \operatorname{dim}E(0,P)=\operatorname{dim}\operatorname{null}P$, so suppose that \newline $\operatorname{dim}G(0,P)> \operatorname{dim}\operatorname{null}P$. 
This implies $\operatorname{null}P^2\neq\operatorname{null}P$, hence there exists a vector $v\in\operatorname{range}P$ such that $Pv=0$ and $v\neq 0$. 
However, since $v\in\operatorname{range}P$, there must exists $u\in V$ such that $Pu=v$ and we obtain a contradiction since $Pu=v$ and $P^2u=0$. 
Thus, it follows that $\operatorname{dim}G(0,P)=\operatorname{dim}\operatorname{null}P$.

Therefore, the multiplicity of $\lambda=0$ is $m=\operatorname{dim}\operatorname{null}P$ and the multiplicity of $\lambda=1$ is $n=\operatorname{dim}\operatorname{range}P$. 
Hence the characteristic polynomial of $P$ is $z^m(z-1)^n$.

\clearpage

\section{Nonzero constant term in minimal polynomial}
\subsection*{Problem statement}
Suppose $T\in\mathcal{L(V)}$. Prove that $T$ is invertible if an only if the constant term in the minimal polynomial of $T$ is nonzero.

\subsection*{Solution}
Theorem 8.49 (`Eigenvalues are the zeros of the minimal polynomial') implies that $0$ is an eigenvalue if and only if $z$ is a factor of the minimal polynomial.

\subsubsection{First Direction}
Suppose $T$ is invertible. 
It follows that $0$ is not an eigenvalue because $Tv=0(v)$ only for $v=0$. 
Hence $z$ is not a factor of the minimal polynomial and the constant term of the minimal polynomial is nonzero.

\subsubsection{Second Direction}
Suppose the constant term in the minimal polynomial of $T$ is nonzero. It follows that $z$ is not a factor of the minimal polynomial. Hence $0$ is not an eigenvalue and $\operatorname{null}T=E(0,T)=\{0\}$. 
Thus $T$ is invertible.


\end{document}