\documentclass{article}
\usepackage{graphicx}
\usepackage{amsmath}
\usepackage{amssymb}
\usepackage{hyperref}
\usepackage{epigraph}
\usepackage{csquotes}
\usepackage{mathtools}
\usepackage{mathdots}

\title{Linear Algebra Done Right\\Solutions to Exercises 8.D}
\author{}
\date{}

\begin{document}

\maketitle

\section{Polynomials of Example 8.53}
\subsection*{Problem statement}
Find the characteristic polynomial and the minimal polynomial of the operator $N$ in Example 8.53.

\subsection*{Solution}
For the sake of completeness, the operator $N\in\mathcal{L}(\mathbf{F}^4)$ is defined by
\[N(z_1,z_2,z_3,z_4)=(0,z_1,z_2,z_3).\]
Via Theorem 5.32 (`Determination of eigenvalues from upper-triangular matrix'), $N$ only has an eigenvalue of $0$, and by Theorem 8.26 (`Sum of the multiplicities equals $\operatorname{dim}V$'), it follows that the multiplicity of $0$ is $4$. 
Hence, the characteristic polynomial is $z^4$.

To find the minimal polynomial, let's compute $N^4$, and we can stop when $N^m=0$ for $m\in\{1,2,3,4\}$. For $(z_1,z_2,z_3,z_4)\in\mathbf{F}^4$ we can write
\begin{align*}
    N^4(z_1,z_2,z_3,z_4)&=N^3(0,z_1,z_2,z_3)\\
    &=N^2(0,0,z_1,z_2)\\
    &=N^1(0,0,0,z_1)\\
    &=(0,0,0,0).
\end{align*}
Therefore, the minimal polynomial of $N$ is $z^4$, the same as the characteristic polynomial.

\clearpage

\section{Polynomials of Example 8.54}
\subsection*{Problem statement}
Find the characteristic polynomial and the minimal polynomial of the operator $N$ in Example 8.54.

\subsection*{Solution}
For the sake of completeness, the operator $N\in\mathcal{L}(\mathbf{F}^6)$ is defined by
\[N(z_1,z_2,z_3,z_4,z_5,z_6)=(0,z_1,z_2,0,z_4,0).\]
Via Theorem 5.32 (`Determination of eigenvalues from upper-triangular matrix'), $N$ only has an eigenvalue of $0$, and by Theorem 8.26 (`Sum of the multiplicities equals $\operatorname{dim}V$'), it follows that the multiplicity of $0$ is $6$. 
Hence, the characteristic polynomial is $z^6$.

To find the minimal polynomial, let's compute $N^6$, and we can stop when $N^m=0$ for $m\in\{1,2,3,4,5,6\}$. For $(z_1,z_2,z_3,z_4,z_5,z_6)\in\mathbf{F}^6$ we can write
\begin{align*}
    N^6(z_1,z_2,z_3,z_4,z_5,z_6)&=N^5(0,z_1,z_2,0,z_4,0)\\
    &=N^4(0,0,z_1,0,0,0)\\
    &=N^3(0,0,0,0,0,0)=(0,0,0,0,0,0).
\end{align*}
Therefore, since it took 3 applications of $N$ to equal $0$, the minimal polynomial of $N$ is $z^3$.

\clearpage

\section{Minimal polynomial of nilpotent is $z^{m+1}$}
\subsection*{Problem statement}
Suppose $N\in\mathcal{L}(V)$ is nilpotent. Prove that the minimal polynomial of $N$ is $z^{m+1}$, where $m$ is the length of the longest consecutive string of $1$'s that appears on the line directly above the diagonal in the matrix of $N$ with respect to any Jordan basis for $N$.

\subsection*{Solution}
As we saw in the proof of Theorem 8.60 (`Jordan Form'), a nilpotent operator $N\in\mathcal{L}(V)$ has vectors $v_1,\ldots,v_n\in V$, given by Theorem 8.55 (`Basis corresponding to a nilpotent operator'), such that list $N^{m_j}v_j,\ldots,Nv_j,v_j$ has a matrix of the form
\begin{equation*}
A_j=\begin{pmatrix}
0 & 1 &  & 0\\
 & \ddots & \ddots & \\
 &  & \ddots & 1\\
0 &  &  & 0
\end{pmatrix}.
\end{equation*}
These matrices $A_j$ lie on the diagonal of the block diagonal matrix given by the Jordan basis of $N$.

For $A_j$, the length of consecutive $1$'s appearing on the line directly above the diagonal is $m_j$\footnote{Note that the list $N^{m_j}v_j,\ldots,Nv_j,v_j$ is of length $m_j+1$.}. 
We can also note that $A_j^{m_j+1}=0$ since $N^{m_j+1}v_j=0$.

Now suppose $N^{m_1}v_1,\ldots,Nv_1,v_1,\ldots,N^{m_n}v_n,\ldots,Nv_n,v_n$ is the basis of $N$ given by Theorem 8.55 and Theorem 8.60. 
Further suppose that $k$ is such that $m_k > m_j$
\footnote{Okay it's possible that there is another $m_i$ such that $m_k=m_i$, but the result still holds.} 
where $N^{m_k+1}v_k=0$ and $N^{m_j+1}v_j=0$. 
For every $v\in V$, we can write $v$ as a linear combination of $N^{m_1}v_1,\ldots,Nv_1,v_1,\ldots,N^{m_n}v_n,\ldots,Nv_n,v_n$. 
Applying $N^{m_k}$ to our linear combination, we have 
\[N^{m_k}v=N^{m_k}v_k\]
where all the other terms are $0$ since $m_k > m_j$. 
Applying $N^{m_k+1}$ to our linear combination, it follows that
\[N^{m_k+1}v=N^{m_k+1}v_k=0\]
for all $v\in V$.

Hence, we've shown that $N^{m_k}\neq0$ and $N^{m_k+1}=0$, implying that $z^{m_k+1}$ is the minimal polynomial. In the second paragraph, we showed that $m_k$ is the length of a consecutive string of $1$'s appearing above the diagonal in the matrix of $N$ with respect to the Jordan basis for $N$. Thus, $m_k > m_j$ implies that the consecutive string of $1$'s corresponding to length $m_k$ is the longest consecutive string of $1$'s, which was to be shown.

\clearpage

\section{Matrix of $T$ with reverse Jordan basis}
\subsection*{Problem statement}
Suppose $T\in\mathcal{L}(V)$ and $v_1,\dots,v_n$ is a basis of $V$ that is a Jordan basis for $T$. Describe the matrix of $T$ with respect to the basis $v_n,\ldots,v_1$ obtained by reversing the order of the $v$'s.

\subsection*{Solution}
This feels more like a $\text{\LaTeX}$ exercise than a math problem.

Suppose $v_1,\dots,v_n$ is a basis of $V$ that is a Jordan basis for $T$. The matrix of $T$ with respect to the basis $v_n,\ldots,v_1$ is a square matrix of the form
\begin{equation*}
\begin{pmatrix}
0 & & A_1\\
& \iddots & \\
A_m & & 0
\end{pmatrix}
\end{equation*}
where $A_1,\ldots,A_m$ are square matrices lying along the line from upper right corner to the bottom left corner and all the other entries of the matrix equal $0$. Each $A_j$ is a matrix of the form
\begin{equation*}
\begin{pmatrix}
0 & & 1 &\lambda_j\\
& \iddots & \iddots & \\
1 & \iddots & & \\
\lambda_j & & & 0
\end{pmatrix}
\end{equation*}
where the $\lambda_j$'s lie along the line from upper right corner to the bottom left corner, the $1$'s lie along the line above the line of $\lambda_j$'s, and all the other entries of $A_j$ equal $0$.

\clearpage

\section{Matrix of $T^2$ respect to Jordan basis of $T$}
\subsection*{Problem statement}
Suppose $T\in\mathcal{L}(V)$ and $v_1,\ldots,v_n$ is a basis of $V$ that is a Jordan basis for $T$. Describe the matrix of $T^2$ with respect to this basis.

\subsection*{Solution}
Following from the definition of the Jordan basis (Definition 8.59), the matrix of $T^2$ with respect to the Jordan basis of $T$ is the block diagonal matrix
\begin{equation*}
\begin{pmatrix}
A_1^2 & & 0\\
& \ddots & \\
0 & & A_m^2
\end{pmatrix}
\end{equation*}
where each $A_j^2$ is an upper-triangular matrix of the form
\begin{equation*}
A_j^2=\begin{pmatrix}
\lambda_j^2 & 2\lambda_j & & 0 \\
& \ddots & \ddots & \\
& & \ddots & 2\lambda_j \\
0 & & & \lambda_j^2
\end{pmatrix}.
\end{equation*}

\clearpage

\section{$N^{m_1}v_1,\ldots,N^{m_n}v_n$ is a basis of $\operatorname{null}N$}
\subsection*{Problem statement}
Suppose $N\in\mathcal{L}(V)$ is nilpotent and $v_1,\ldots,v_n$ and $m_1,\ldots,m_n$ are as in 8.55. Prove that $N^{m_1}v_1,\ldots,N^{m_n}v_n$ is a basis of $\operatorname{null}N$.

\subsection*{Solution}
By Theorem 8.55 (`Basis corresponding to a nilpotent operator'), the list of vectors $N^{m_1}v_1,\ldots,N^{m_n}v_n$ are linearly independent and
\[N^{m_1}v_1,\ldots,N^{m_n}v_n\in\operatorname{null}N.\]
To show that $N^{m_1}v_1,\ldots,N^{m_n}v_n$ spans $\operatorname{null}N$, we can write $v\in\operatorname{null}N$ as a linear combination of $N^{m_1}v_1,\ldots,Nv_1,v_1,\ldots,N^{m_n}v_n,\ldots,Nv_n,v_n$. 
Applying $N$ to both sides, we have $Nv=0$, $N(N^{{m_1}}v_1)=\cdots=N(N^{{m_n}}v_n)=0$, but vectors $N(N^{{m_1-1}}v_1),\ldots,N(Nv_1),Nv_1,\ldots,N(N^{m_n-1}v_n),\ldots,N(Nv_n),N(v_n)$ do not equal zero. 
Since vectors \newline
$N(N^{{m_1-1}}v_1),\ldots,N(Nv_1),Nv_1,\ldots,N(N^{m_n-1}v_n),\ldots,N(Nv_n),N(v_n)$ are linearly independent, their coefficients in the linear combination equal to $v$ are $0$, implying that $v\in\operatorname{span}(N^{m_1}v_1,\ldots,N^{m_n}v_n)$.

Hence, $N^{m_1}v_1,\ldots,N^{m_n}v_n$ is a basis of $\operatorname{null}N$


\end{document}