\documentclass{article}
\usepackage{graphicx}
\usepackage{amsmath}
\usepackage{amssymb}
\usepackage{hyperref}
\usepackage{epigraph}
\usepackage{csquotes}
\usepackage{mathtools}
\usepackage{mathdots}

\title{Linear Algebra Done Right\\Solutions to Exercises 9.A}
\author{}
\date{}

\begin{document}

\maketitle

\section{Prove $V_{\mathbf{C}}$ is a complex vector space}
\subsection*{Problem statement}
Prove 9.3.

\subsection*{Solution}
All we need to do is use Definition 9.3 (`Complexification of $V$, $V_{\mathbf{C}}$') to show that $V_{\mathbf{C}}$ satisfies the condition of a vector space in Definition 1.19 (`vector space').

\subsubsection*{commutativity}
For $u_1,v_1,u_2,v_2\in V$, we can write
\begin{multline*}
(u_1+iv_1)+(u_2+iv_2)=(u_1+u_2)+i(v_1+v_2)\\
=(u_2+u_1)+i(v_2+v_1)=(u_2+iv_2)+(u_1+iv_1)
\end{multline*}
where the first and third equalities come from the definition of addition on $V_{\mathbf{C}}$, and the second equality comes from \textbf{commutativity} on $V$.

\subsubsection*{associativity}
For $u_1,v_1,u_2,v_2,u_3,v_3\in V$, we can show additive \textbf{associativity} by writing
\begin{align*}
    ((u_1+iv_1)+(u_2+iv_2))+(u_3+iv_3)&=((u_1+u_2)+i(v_1+v_2))+(u_3+iv_3)\\
    &=((u_1+u_2)+u_3)+i((v_1+v_2)+v_3)\\
    &=(u_1+(u_2+u_3))+i(v_1+(v_2+v_3))\\
    &=(u_1+iv_1)+((u_2+u_3)+i(v_2+v_3))\\
    &=(u_1+iv_1)+((u_2+iv_2)+(u_3+iv_3))
\end{align*}
where the first, second, fourth, and fifth equalities come from the definition of addition on $V_{\mathbf{C}}$, and the third equality comes from \textbf{associativity} on $V$.

For $a_1,b_1,a_2,b_2\in\mathbf{R}$ and $u,v\in V$, we can show multiplicative \textbf{associativity} by writing
\begin{align*}
    ((a_1+b_1i)(a_2+b_2i))(u+iv)&=((a_1a_2-b_1b_2)+(a_1b_2+b_1a_2)i)(u+iv)\\
    &=((a_1a_2-b_1b_2)u-(a_1b_2+b_1a_2)v)\\
    &\qquad+i((a_1a_2-b_1b_2)v+(a_1b_2+b_1a_2)u)\\
    &=(a_1a_2u-b_1b_2u-a_1b_2v-b_1a_2v)\\
    &\qquad+i(a_1a_2v-b_1b_2v+a_1b_2u+b_1a_2u)\\
    &=(a_1(a_2u-b_2v)-b_1(a_2v+b_2u))\\
    &\qquad+i(a_1(a_2v+b_2u)+b_1(a_2u-b_2v))\\
    &=(a_1+b_1i)((a_2u-b_2v)+i(a_2v+b_2u))\\
    &=(a_1+b_1i)((a_2+b_2i)(u+iv))
\end{align*}
where the first, second, fifth, and sixth equalities come from the definition of complex scalar multiplication on $V_{\mathbf{C}}$; and the third and fourth equalities come from the \textbf{distributive properties} on $V$.

\subsubsection*{additive identity}
The \textbf{additive identity} is $0+i0$, which we can verify by writing
\[(u+iv)+(0+i0)=(u+0)+i(v+0)=u+iv\]
for all $u,v\in V$.

\subsubsection*{additive inverse}
For $u,v\in V$, the \textbf{additive inverse} of $u+iv$ is simply $-u-iv$, which we can verify by writing
\[(u+iv)+(-u-iv)=(u-u)+i(v-v)=0+i0.\]

\subsubsection*{multiplicative identity}
The \textbf{multiplicative identity} is simply $1+0i$, which we can verify by writing
\[(1+0i)(u+iv)=(1u-0(v))+i(1v+0(u))=u+iv\]
for all $u,v\in V$.

\subsubsection*{distributive properties}
For $u_1,v_1,u_2,v_2\in V$ and $a,b\in\mathbf{R}$, we can write
\begin{align*}
    (a+bi)((u_1+iv_1)+(u_2+iv_2))&=(a+bi)((u_1+u_2)+i(v_1+v_2))\\
    &=(a(u_1+u_2)-b(v_1+v_2))\\
    &\qquad+i(a(v_1+v_2) + b(u_1+u_2))\\
    &=(au_1+au_2-bv_1-bv_2))\\
    &\qquad+i(av_1+av_2 + bu_1+bu_2))\\
    &=(au_1-bv_1)+i(av_1+bu_1)\\
    &\qquad+(au_2-bv_2)+i(av_2+bu_2)\\
    &=(a+bi)(u_1+iv_1)+(a+bi)(u_2+iv_2).
\end{align*}
For $u,v\in V$ and $a_1,b_1,a_2,b_2\in\mathbf{R}$, we can write
\begin{align*}
    ((a_1+b_1i)+(a_2+b_2i))(u+iv)&=((a_1+a_2)+(b_1+b_2)i)(u+iv)\\
    &=((a_1+a_2)u-(b_1+b_2)v)\\
    &\qquad+i((a_1+a_2)v+(b_1+b_2)u)\\
    &=(a_1u+a_2u-b_1v-b_2v)\\
    &\qquad+i(a_1v+a_2v+b_1u+b_2u)\\
    &=(a_1u-b_1v)+i(a_1v+b_1u)\\
    &\qquad+(a_2u-b_2v)+i(a_2v+b_2u)\\
    &=(a_1+b_1i)(u+iv)+(a_2+b_2i)(u+iv).
\end{align*}

\clearpage

\section{If $T\in\mathcal{L}(V)$, then $T_{\mathbf{C}}\in\mathcal{L}(V_{\mathbf{C}})$}
\subsection*{Problem statement}
Verify that if $V$ is a real vector space and if $T\in\mathcal{L}(V)$, then $T_{\mathbf{C}}\in\mathcal{L}(V_{\mathbf{C}})$.

\clearpage

\section{Linear independence in $V$ and $V_{\mathbf{C}}$}
\subsection*{Problem statement}
Suppose $V$ is a real vector space and $v_1,\ldots,v_m\in V$. 
Prove that $v_1,\ldots,v_m\in V$ is linearly independent in $V_{\mathbf{C}}$ if and only if $v_1,\ldots,v_m\in V$ is linearly independent in $V$.

\clearpage

\section{Spanning list in $V$ and $V_{\mathbf{C}}$}
\subsection*{Problem statement}
Suppose $V$ is a real vector space and $v_1,\ldots,v_m\in V$. 
Prove that $v_1,\ldots,v_m\in V$ spans $V_{\mathbf{C}}$ if and only if $v_1,\ldots,v_m\in V$ spans $V$.

\clearpage

\section{$\mathcal{L}(V_{\mathbf{C}})$ is a vector space over $\mathbf{R}$}
\subsection*{Problem statement}
Suppose that $V$ is a real vector space and $S,T\in\mathcal{L}(V)$. 
Show that \newline
$(S+T)_{\mathbf{C}}=S_{\mathbf{C}}+T_{\mathbf{C}}$ and that $(\lambda T)_{\mathbf{C}}=\lambda T_{\mathbf{C}}$ for every $\lambda\in\mathbf{R}$.

\clearpage

\section{Prove $T_{\mathbf{C}}$ is invertible iff $T$ is invertible}
\subsection*{Problem statement}
Suppose $V$ is a real vector space and $T\in\mathcal{L}(V)$. Prove that $T_{\mathbf{C}}$ is invertible if and only if $T$ is invertible.

\clearpage

\section{Prove $N_{\mathbf{C}}$ is nilpotent iff $N$ is nilpotent}
\subsection*{Problem statement}
Suppose $V$ is a real vector space and $N\in\mathcal{L}(V)$. Prove that $N_{\mathbf{C}}$ is nilpotent if and only if $N$ is nilpotent.

\clearpage

\section{Nonreal eigenvalues of $T_{\mathbf{C}}$}
\subsection*{Problem statement}
Suppose $T\in\mathcal{L}(\mathbf{R}^3)$ and $5,7$ are eigenvalues of $T$. Prove that $T_{\mathbf{C}}$ has no nonreal eigenvalues.

\clearpage

\section{$T\in\mathcal{L}(\mathbf{R}^7)$ such that $T^2+T+I$ is nilpotent}
\subsection*{Problem statement}
Prove there does not exist an operator $T\in\mathcal{L}(\mathbf{R}^7)$ such that $T^2+T+I$ is nilpotent.

\subsection*{Solution}
Suppose there exists some operator $T\in\mathcal{L}(\mathbf{R}^7)$ such that $T^2+T+I$ is nilpotent. Via Theorem 8.18 (`Nilpotent operator raised to dimension of domain is $0$'), it follows that $(T^2+T+I)^7=0$ and for the polynomial $q(z)=(z^2+z+1)^7$ that $q(T)=0$. Via Theorem 8.46 (`$q(T)=0$ implies $q$ is a multiple of the minimal polynomial'), we can infer that $q$ is a polynomial multiple of the minimal polynomial of $T$. Via Theorem 9.19 (`Operator on odd-dimensional vector space has eigenvalue'), it follows that $T$ has a real eigenvalue, and via Theorem 8.49 (`Eigenvalues are the zeros of the minimal polynomial'), that eigenvalue is a zero of the minimal polynomial of $T$.

However, we can see that the polynomial $z^2+z+1$ has no real zeros\footnote{This follows from Theorem 4.16 (`Factorization of a quadratic polynomial') and the simple observation that $1^2\ngeq 4(1)$}. Thus, the polynomial $q(z)=(z^2+z+1)^7$ has no real zeros and cannot be a polynomial multiple of the minimal polynomial of $T$ since the minimal polynomial of $T$ is guaranteed to have a real zero. Therefore, we have a contradiction and it follows that no such operator $T\in\mathcal{L}(\mathbf{R}^7)$ exists such that $T^2+T+I$ is nilpotent.

\clearpage

\section{$T\in\mathcal{L}(\mathbf{C}^7)$ such that $T^2+T+I$ is nilpotent}
\subsection*{Problem statement}
Give an example of an operator $T\in\mathcal{L}(\mathbf{C}^7)$ such that $T^2+T+I$ is nilpotent.

\subsection*{Solution}
Let the operator $T\in\mathcal{L}(\mathbf{C}^7)$ be defined by
\[Tv=\frac{-1+\sqrt{3}i}{2}v.\]
For every $v\in\mathbf{C}^7$, we can write
\begin{align*}
    (T^2+T+I)v&=T^2v+Tv+Iv\\
    &=(\frac{-1+\sqrt{3}i}{2})^2v+\frac{-1+\sqrt{3}i}{2}v+v\\
    &=\frac{-2-2\sqrt{3}i}{4}v+\frac{-2+2\sqrt{3}i}{4}v+v\\
    &=-v+v=0
\end{align*}
and hence, $T^2+T+I=0$. The $0$ operator is clearly nilpotent, so it follows that $T^2+T+I$ is nilpotent.


\end{document}