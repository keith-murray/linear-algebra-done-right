\documentclass{article}
\usepackage{graphicx}
\usepackage{amsmath}
\usepackage{amssymb}
\usepackage{hyperref}
\usepackage{epigraph}
\usepackage{csquotes}
\usepackage{mathtools}
\usepackage{mathdots}

\title{Linear Algebra Done Right\\Solutions to Exercises 9.B}
\author{}
\date{}

\providecommand{\abs}[1]{\lvert#1\rvert} \providecommand{\norm}[1]{\lVert#1\rVert}
\DeclarePairedDelimiter{\innerprod}\langle\rangle
\newcommand\conjinnerp[2][]{\:\overline{\mkern-4mu\innerprod[#1]{#2}\mkern-4mu}\:}

\begin{document}

\maketitle

\section{For isometry $S\in\mathcal{L}(\mathbf{R}^3)$, there exists $S^2x=x$}
\subsection*{Problem statement}
Suppose $S\in\mathcal{L}(\mathbf{R}^3)$ is an isometry. 
Prove that there exists a nonzero vector $x\in\mathbf{R}^3$ such that $S^2x=x$.

\subsection*{Solution}
Via Theorem 9.36 (`Description of isometries when $\mathbf{F}=\mathbf{R}$'), there exists an orthonormal basis of $\mathbf{R}^3$ with respect to which $S$ has a block diagonal matrix and each block in that matrix is either a 1-by-1 matrix containing $1$ or $-1$ or a 2-by-2 matrix of the form given in Theorem 9.36. 
Since $\operatorname{dim}\mathbf{R}^3=3$, it follows that this block diagonal matrix either has one 2-by-2 matrix or does not have a 2-by-2 matrix. 
In both cases, there is at least one 1-by-1 matrix containing $1$ or $-1$ in the block diagonal matrix. 

Suppose that the vector in the orthonormal basis of $\mathbf{R}^3$ corresponding to this 1-by-1 matrix is $e_j$. 
If the entry in this 1-by-1 matrix is $1$, then 
\[S^2e_j=S(1e_j)=Se_j=1e_j=e_j\]
and $e_j$ is the vector we are looking for. 
If the entry in this 1-by-1 matrix is $-1$, then
\[S^2e_j=S(-1e_j)=-1Se_j=-1(-1e_j)=e_j\]
and $e_j$ is, again, the vector we are looking for. 
Thus, for all possible isometries in $\mathcal{L}(\mathbf{R}^3)$, there exists a nonzero vector $x\in\mathbf{R}^3$ such that $S^2x=x$.

\clearpage

\section{Eigenvalues for odd-dimensional isometries}
\subsection*{Problem statement}
Prove that every isometry on an odd-dimensional real inner product space has $1$ or $-1$ as an eigenvalue.

\subsection*{Solution}
Suppose $V$ is an odd-dimensional real inner product space with dimension $n$. 
Suppose $S\in\mathcal{L}(V)$ is an isometry. 
It follows from Theorem 9.36 (`Description of isometries when $\mathbf{F}=\mathbf{R}$') that $S$ has a block diagonal matrix with at most $\frac{n-1}{2}$ number\footnote{
Since $n$ is odd, then $\frac{n-1}{2}$ is an integer
} of 2-by-2 matrices on the diagonal. 
Thus, every isometry on an odd-dimensional real inner product space has at least one 1-by-1 matrix containing $1$ or $-1$ on the diagonal of some block diagonal matrix. 

Suppose $e_j$ is the orthonormal vector corresponding to this 1-by-1 matrix in the block diagonal matrix of $S$. 
If the entry in this 1-by-1 matrix is $1$, then 
\[Se_j=1e_j\]
and $1$ is an eigenvalue of $S$. 
If the entry in this 1-by-1 matrix is $-1$, then
\[Se_j=-1e_j\]
and $-1$ is an eigenvalue of $S$. 
Therefore, every isometry on an odd-dimensional real inner product space has $1$ or $-1$ as an eigenvalue.

\clearpage

\section{Complex inner product on $V_{\mathbf{C}}$}
\subsection*{Problem statement}
Suppose $V$ is a real inner product space. 
Show that 
\[\langle u+iv,x+iy\rangle=\langle u,x\rangle +\langle v,y\rangle + (\langle v,x\rangle - \langle u,y\rangle)i\]
for $u,v,x,y\in V$ defines a complex inner product on $V_{\mathbf{C}}$.

\subsection*{Solution}
To show that the inner product in the \textbf{Problem statement} defines a complex inner product, we have to show that it fulfills the properties in the definition of inner products (Definition 6.3).

\subsubsection*{positivity}
For $u,v\in V$, we can write
\begin{align*}
    \langle u+iv,u+iv\rangle&=\langle u,u\rangle +\langle v,v\rangle + (\langle v,u\rangle - \langle u,v\rangle)i\\
    &=\langle u,u\rangle +\langle v,v\rangle + (\langle v,u\rangle - \langle v,u\rangle)i\\
    &=\langle u,u\rangle +\langle v,v\rangle \geq 0
\end{align*}
where $\innerprod{u,v}=\conjinnerp{v,u}=\innerprod{v,u}$ since $\innerprod{u,v}\in\mathbf{R}$, and $\innerprod{u,u}\geq0$ and $\innerprod{v,v}\geq0$ follow from the property of \textbf{positivity} on the inner product associated with $V$.

\subsubsection*{definiteness}
Suppose there exist $u,v\in V$, such that
\[\langle u+iv,u+iv\rangle=\langle u,u\rangle +\langle v,v\rangle + (\langle v,u\rangle - \langle u,v\rangle)i=\langle u,u\rangle +\langle v,v\rangle=0.\]
Given the property of \textbf{positivity}, it follows that $\langle u,u\rangle=\langle v,v\rangle=0$ and we can use the property of \textbf{definiteness} on the inner product associated with $V$ to infer that $u=v=0$. 
Hence, if $\langle u+iv,u+iv\rangle=0$, then $u=v=0$.

It clearly follows that if $u=v=0$ then $\langle u+iv,u+iv\rangle=0$.

\subsubsection*{additivity in first slot}
For $u_1,v_1,u_2,v_2,x,y\in V$, we can write
\begin{align*}
    \langle (u_1+iv_2)+(u_2+iv_2),x+iy\rangle&=\innerprod{(u_1+u_2)+i(v_1+v_2),x+iy}\\
    &=\langle (u_1+u_2),x\rangle +\langle (v_1+v_2),y\rangle\\
    &\qquad+ (\langle (v_1+v_2),x\rangle - \langle (u_1+u_2),y\rangle)i\\
    &=\innerprod{u_1,x}+\innerprod{v_1,y}+(\innerprod{v_1,x}-\innerprod{u_1,y})i\\
    &\qquad +\innerprod{u_2,x}+\innerprod{v_2,y}+(\innerprod{v_2,x}-\innerprod{u_2,y})i\\
    &=\innerprod{u_1+iv_1,x+iy}+\innerprod{u_2+iv_2,x+iy}
\end{align*}
where the first equality follows from Definition 9.2 (`complexification on $V_{\textbf{C}}$'), and the second and fourth equalities follow from the definition of the complex inner product on $V_{\textbf{C}}$.

\subsubsection*{homogeneity in first slot}
For $u,v,x,y\in V$ and $a,b\in\textbf{R}$, we can write
\begin{align*}
    \langle (a+bi)(u+iv),x+iy\rangle&=\innerprod{(au-bv)+i(av+bu),x+iy}\\
    &=\innerprod{au-bv,x}+\innerprod{av+bu,y}\\
    &\qquad +(\innerprod{av+bu,x}-\innerprod{au-bv,y})i\\
    &=a\innerprod{u,x}-b\innerprod{v,x}+a\innerprod{v,y}+b\innerprod{u,y}\\
    &\qquad +(a\innerprod{v,x}+b\innerprod{u,x}-a\innerprod{u,y}+b\innerprod{v,y})i\\
    &=a(\innerprod{u,x}+\innerprod{v,y})-b(\innerprod{v,x}-\innerprod{u,y})\\
    &\qquad +(a(\innerprod{v,x}-\innerprod{u,y})+b(\innerprod{u,x}+\innerprod{v,y}))i\\
    &=(a+bi)(\innerprod{u,x}+\innerprod{v,y}+(\innerprod{v,x}-\innerprod{u,y})i)\\
    &=(a+bi)\langle u+iv,x+iy\rangle
\end{align*}
where the first equality follows from Definition 9.2, the second and sixth equality follows from the definition of the complex inner product, and the fifth quality follows from Definition 1.1 (`complex numbers').

\subsubsection*{conjugate symmetry}
For $u,v,x,y\in V$, we can write
\begin{align*}
    \innerprod{u+iv,x+iy}&=\innerprod{u,x} +\innerprod{v,y} + (\innerprod{v,x} - \innerprod{u,y})i\\
    &=\innerprod{x,u} +\innerprod{y,v} + (\innerprod{x,v} - \innerprod{y,u})i\\
    &=\innerprod{y,v} + \innerprod{x,u} - (\innerprod{y,u} - \innerprod{x,v})i\\
    &=\overline{\innerprod{y,v} + \innerprod{x,u} + (\innerprod{y,u} - \innerprod{x,v})i}\\
    &=\conjinnerp{x+iy,u+iv}
\end{align*}
where $\innerprod{u,x}=\innerprod{x,u},\innerprod{v,y}=\innerprod{y,v},\dots$ since those inner products are real numbers and the fourth equality follows from the definition of the complex conjugate (Definition 4.3).

\clearpage

\section{If $T\in\mathcal{L}(V)$ is self-adjoint, then $T_{\textbf{C}}$ is too}
\subsection*{Problem statement}
Suppose $V$ is a real inner product space and $T\in\mathcal{L}(V)$ is self-adjoint. 
Show that $T_{\textbf{C}}$ is a self-adjoint operator on the inner product space $V_{\textbf{C}}$ defined by the previous exercise.

\subsection*{Solution}
For $u,v,x,y\in V$, we can write
\begin{align*}
    \innerprod{u+iv,T^*_{\textbf{C}}(x+iy)}&=\innerprod{T_{\textbf{C}}(u+iv),x+iy}\\
    &=\innerprod{Tu+iTv,x+iy}\\
    &=\innerprod{Tu,x} +\innerprod{Tv,y} + (\innerprod{Tv,x} - \innerprod{Tu,y})i\\
    &=\innerprod{u,Tx} +\innerprod{v,Ty} + (\innerprod{v,Tx} - \innerprod{u,Ty})i\\
    &=\innerprod{u+iv,Tx+iTy}\\
    &=\innerprod{u+iv,T_{\textbf{C}}(x+iy)}
\end{align*}
where the first equality comes from the definition of the adjoint (Definition 7.2), the second and sixth equality come from the definition of the complexification of $T$ (Definition 9.5), and the fourth equality comes from $T$ being self-adjoint. 
Therefore, $T_{\textbf{C}}$ is a self-adjoint operator on the inner product space $V_{\textbf{C}}$ defined by Exercise 9.B(3).

\clearpage

\section{Prove Real Spectral Theorem via $T_{\textbf{C}}$}
\subsection*{Problem statement}
Use the previous exercise to give a proof of the Real Spectral Theorem (7.29) via complexification and the Complex Spectral Theorem (7.24).

\subsection*{Solution}
For completeness, the Real Spectral Theorem (Theorem 7.29) states that the following are equivalent for $\textbf{F}=\textbf{R}$ and $T\in\mathcal{L}(V)$:
\begin{enumerate}
  \item[(a)] $T$ is self-adjoint
  \item[(b)] $V$ has an orthonormal basis consisting of eigenvectors of $T$.
  \item[(c)] $T$ has a diagonal matrix with respect to some orthonormal basis of $V$.
\end{enumerate}
As given in Axler's original proof, (b) clearly implies (c) and (c) implies (a) since diagonal matrices are equal to their transpose. 
Thus, in this solution, we need to prove that (a) implies (b) with complexification and the Complex Spectral Theorem (Theorem 7.24).

Suppose (a) holds, so $T\in\mathcal{L}(V)$ is self-adjoint. 
Via Exercise 9.B(4), it follows that $T_{\textbf{C}}\in\mathcal{L}(V_{\textbf{C}})$ is self-adjoint. 
Thus, via the Complex Spectral Theorem, $V_{\textbf{C}}$ has an orthonormal basis consisting of eigenvectors of $T_{\textbf{C}}$. 
Suppose $e_1,\ldots,e_n$ is this orthonormal basis. 
If we can use $e_1,\ldots,e_n$ to construct an orthonormal basis of $V$ consisting of eigenvectors of $T$, then it follows that (b) holds.

Theorem 7.13 (`Eigenvalues of self-adjoint operators are real') tells us that the eigenvector $e_j$ of $T_{\textbf{C}}$ has a corresponding eigenvalue of $\lambda_j$ such that $\lambda_j$ is real. 
Now we can use Theorem 9.12 (`$T_{\textbf{C}}-\lambda I$ and $T_{\textbf{C}}-\bar{\lambda} I$') to claim that for eigenvector $e_j=u_j+iv_j$ of $T_{\textbf{C}}$ corresponding to eigenvalue $\lambda$, there exists another eigenvector $e_k=u_j-iv_j$ also corresponding to $\lambda$ since $\lambda=\bar{\lambda}$. 
To construct an orthonormal basis of $V$ consisting of eigenvectors of $T$, there are three cases to consider: $\textbf{v=0}$, $\textbf{u=0}$, and $\textbf{u}\neq\textbf{v}\neq\textbf{0}$.

\subsubsection*{$\textbf{v=0}$}
Suppose $e_j=u_j+i0$. It follows that $e_j\in V$ and 
\[\lambda_ju_j+i0=\lambda_j(e_j)=T_{\mathbf{C}}(e_j)=Tu_j+iT(0)=Tu_j+i0.\]
Hence, $u_j$ is an eigenvector of $T$ corresponding to the eigenvalue $\lambda_j$. 
This case tells us that if we can \textit{convert} eigenvectors in the list $e_1,\ldots,e_n$ to vectors in $V$, then the \textit{converted} vector is an eigenvector of $T$ with an eigenvalue of $\lambda_j$.

\subsubsection*{$\textbf{u=0}$}
Suppose $e_j=0+iv_j$. Since $e_1,\ldots,e_n$ is orthonormal, we can replace $e_j$ in the list with $-ie_j=v_j+i0$ without affecting the norm of $e_j$ since 
\[\norm{-ie_j}=\abs{-i}\norm{e_j}=\norm{e_j}.\]
The vector $e_j$ remains orthogonal to other vectors in $e_1,\ldots,e_n$ since
\[\innerprod{-ie_j,e_k}=-i\innerprod{e_j,e_k}=-i(0)=0.\]
Via the $\textbf{v=0}$ case, we know that $v_j$ is an eigenvector of $T$ with an eigenvalue of $\lambda_j$.

\subsubsection*{$\textbf{u}\neq\textbf{v}\neq\textbf{0}$}
First, let's show that for eigenvectors $e_j=u_j+iv_j$ and\newline 
$e_k=u_j-iv_j$, it follows that if $u_j\neq 0$ and $v_j\neq0$, then $u_j\neq v_j$. 
Since $e_j,e_k$ are in the orthonormal list $e_1,\ldots,e_n$, it follows that $\innerprod{e_j,e_k}=0$ and we can write
\begin{align*}
    0=\innerprod{e_j,e_k}&=\innerprod{u_j+iv_j,u_j-iv_j}\\
    &=\innerprod{u_j,u_j} +\innerprod{v_j,-v_j} + (\innerprod{v_j,u_j} - \innerprod{u_j,-v_j})i\\
    &=\innerprod{u_j,u_j} -\innerprod{v_j,v_j} + (2\innerprod{v_j,u_j})i.
\end{align*}
Hence it follows that $\norm{u_j}=\norm{v_j}$, $\innerprod{v_j,u_j}=0$, and, via the property of \textbf{definiteness}, $u_j\neq v_j$. 
Happily, we've also shown that $u_j$ and $v_j$ are orthogonal. 
Thus, to replace vectors $e_j=u_j+iv_j$ and $e_k=u_j-iv_j$ in the orthonormal list $e_1,\ldots,e_n$, we can use vectors $\frac{u_j}{\norm{u_j}}+i0$ and $\frac{v_j}{\norm{v_j}}+i0$. 
These vectors are orthogonal to each other, have a norm of $1$, and, since they are a linear combination of $e_j$ and $e_k$, they are orthogonal to all other vectors in $e_1,\ldots,e_n$. 
To show that they are eigenvectors of $T$, for $e_j=u_j+iv_j$ we can write
\[\lambda_ju_j+i(\lambda_jv_j)=\lambda_j(e_j)=T_{\mathbf{C}}(e_j)=Tu_j+iTv_j.\]
Thus, $\frac{u_j}{\norm{u_j}}$ and $\frac{v_j}{\norm{v_j}}$ are eigenvectors of $T$ corresponding to eigenvalue $\lambda_j$.

\subsubsection*{Putting it all together}
Up to this point, we have shown that if (a) holds, then $V_{\textbf{C}}$ has an orthonormal basis $e_1,\ldots,e_n$ consisting of eigenvectors of $T_{\textbf{C}}$. 
Through our cases $\textbf{v=0}$, $\textbf{u=0}$, and $\textbf{u}\neq\textbf{v}\neq\textbf{0}$, we have shown that every vector in $e_1,\ldots,e_n$ can be replaced with a vector in $V$ that is an eigenvector of $T$. 
This new list is also orthonormal and of length $n$. 
Therefore, via Theorem 9.4(b) (`Basis of $V$ is a basis of $V_{\textbf{C}}$'), this new list is a basis of $V$. Hence, $V$ has an orthonormal basis consisting of eigenvectors of $T$. 

\clearpage

\section{If $T$ is not normal, then Theorem 9.30 fails}
\subsection*{Problem statement}
Give an example of an operator $T$ on an inner product space such that $T$ has an invariant subspace whose orthogonal complement is not invariant under $T$.

\subsection*{Solution}
Let $T\in\mathcal{L}(\textbf{R}^2)$ be defined by 
\[T(x,y)=(x+y,0).\]
The subspace $U=\operatorname{span}((1,0))$ is clearly invariant under $T$ (in fact, $(1,0)$ is an eigenvector of $T$). 
However, the orthogonal complement $U^\bot=\operatorname{span}((0,1))$ is not invariant under $T$ since $T(0,1)=(1,0)$ and thus, $T(0,1)\notin U^\bot$.

\end{document}