\documentclass{article}
\usepackage{graphicx}
\usepackage{amsmath}
\usepackage{amssymb}
\usepackage{hyperref}
\usepackage{epigraph}
\usepackage{csquotes}
\usepackage{mathtools}
\usepackage{mathdots}

\title{Linear Algebra Done Right\\Solutions to Exercises 9.A}
\author{}
\date{}

\begin{document}

\maketitle

\section{Prove $V_{\mathbf{C}}$ is a complex vector space}
\subsection*{Problem statement}
Prove 9.3.

\subsection*{Solution}
All we need to do is use Definition 9.3 (`Complexification of $V$, $V_{\mathbf{C}}$') to show that $V_{\mathbf{C}}$ satisfies the condition of a vector space in Definition 1.19 (`vector space').

\subsubsection*{commutativity}
For $u_1,v_1,u_2,v_2\in V$, we can write
\begin{multline*}
(u_1+iv_1)+(u_2+iv_2)=(u_1+u_2)+i(v_1+v_2)\\
=(u_2+u_1)+i(v_2+v_1)=(u_2+iv_2)+(u_1+iv_1)
\end{multline*}
where the first and third equalities come from the definition of addition on $V_{\mathbf{C}}$, and the second equality comes from \textbf{commutativity} on $V$.

\subsubsection*{associativity}
For $u_1,v_1,u_2,v_2,u_3,v_3\in V$, we can show additive \textbf{associativity} by writing
\begin{align*}
    ((u_1+iv_1)+(u_2+iv_2))+(u_3+iv_3)&=((u_1+u_2)+i(v_1+v_2))+(u_3+iv_3)\\
    &=((u_1+u_2)+u_3)+i((v_1+v_2)+v_3)\\
    &=(u_1+(u_2+u_3))+i(v_1+(v_2+v_3))\\
    &=(u_1+iv_1)+((u_2+u_3)+i(v_2+v_3))\\
    &=(u_1+iv_1)+((u_2+iv_2)+(u_3+iv_3))
\end{align*}
where the first, second, fourth, and fifth equalities come from the definition of addition on $V_{\mathbf{C}}$, and the third equality comes from \textbf{associativity} on $V$.

For $a_1,b_1,a_2,b_2\in\mathbf{R}$ and $u,v\in V$, we can show multiplicative \textbf{associativity} by writing
\begin{align*}
    ((a_1+b_1i)(a_2+b_2i))(u+iv)&=((a_1a_2-b_1b_2)+(a_1b_2+b_1a_2)i)(u+iv)\\
    &=((a_1a_2-b_1b_2)u-(a_1b_2+b_1a_2)v)\\
    &\qquad+i((a_1a_2-b_1b_2)v+(a_1b_2+b_1a_2)u)\\
    &=(a_1a_2u-b_1b_2u-a_1b_2v-b_1a_2v)\\
    &\qquad+i(a_1a_2v-b_1b_2v+a_1b_2u+b_1a_2u)\\
    &=(a_1(a_2u-b_2v)-b_1(a_2v+b_2u))\\
    &\qquad+i(a_1(a_2v+b_2u)+b_1(a_2u-b_2v))\\
    &=(a_1+b_1i)((a_2u-b_2v)+i(a_2v+b_2u))\\
    &=(a_1+b_1i)((a_2+b_2i)(u+iv))
\end{align*}
where the first, second, fifth, and sixth equalities come from the definition of complex scalar multiplication on $V_{\mathbf{C}}$; and the third and fourth equalities come from the \textbf{distributive properties} on $V$.

\subsubsection*{additive identity}
The \textbf{additive identity} is $0+i0$, which we can verify by writing
\[(u+iv)+(0+i0)=(u+0)+i(v+0)=u+iv\]
for all $u,v\in V$.

\subsubsection*{additive inverse}
For $u,v\in V$, the \textbf{additive inverse} of $u+iv$ is simply $-u-iv$, which we can verify by writing
\[(u+iv)+(-u-iv)=(u-u)+i(v-v)=0+i0.\]

\subsubsection*{multiplicative identity}
The \textbf{multiplicative identity} is simply $1+0i$, which we can verify by writing
\[(1+0i)(u+iv)=(1u-0(v))+i(1v+0(u))=u+iv\]
for all $u,v\in V$.

\subsubsection*{distributive properties}
For $u_1,v_1,u_2,v_2\in V$ and $a,b\in\mathbf{R}$, we can write
\begin{align*}
    (a+bi)((u_1+iv_1)+(u_2+iv_2))&=(a+bi)((u_1+u_2)+i(v_1+v_2))\\
    &=(a(u_1+u_2)-b(v_1+v_2))\\
    &\qquad+i(a(v_1+v_2) + b(u_1+u_2))\\
    &=(au_1+au_2-bv_1-bv_2))\\
    &\qquad+i(av_1+av_2 + bu_1+bu_2))\\
    &=(au_1-bv_1)+i(av_1+bu_1)\\
    &\qquad+(au_2-bv_2)+i(av_2+bu_2)\\
    &=(a+bi)(u_1+iv_1)+(a+bi)(u_2+iv_2).
\end{align*}
For $u,v\in V$ and $a_1,b_1,a_2,b_2\in\mathbf{R}$, we can write
\begin{align*}
    ((a_1+b_1i)+(a_2+b_2i))(u+iv)&=((a_1+a_2)+(b_1+b_2)i)(u+iv)\\
    &=((a_1+a_2)u-(b_1+b_2)v)\\
    &\qquad+i((a_1+a_2)v+(b_1+b_2)u)\\
    &=(a_1u+a_2u-b_1v-b_2v)\\
    &\qquad+i(a_1v+a_2v+b_1u+b_2u)\\
    &=(a_1u-b_1v)+i(a_1v+b_1u)\\
    &\qquad+(a_2u-b_2v)+i(a_2v+b_2u)\\
    &=(a_1+b_1i)(u+iv)+(a_2+b_2i)(u+iv).
\end{align*}

\clearpage

\section{If $T\in\mathcal{L}(V)$, then $T_{\mathbf{C}}\in\mathcal{L}(V_{\mathbf{C}})$}
\subsection*{Problem statement}
Verify that if $V$ is a real vector space and if $T\in\mathcal{L}(V)$, then $T_{\mathbf{C}}\in\mathcal{L}(V_{\mathbf{C}})$.

\subsection*{Solution}
To verify that $T_{\mathbf{C}}\in\mathcal{L}(V_{\mathbf{C}})$, we must demonstrate the properties of \textbf{additivity} and \textbf{homogeneity}.

\subsubsection*{additivity}
For $u_1,v_1,u_2,v_2\in V$, we can write
\begin{align*}
    T_{\mathbf{C}}((u_1+iv_1)+(u_2+iv_2))&=T_{\mathbf{C}}((u_1+u_2)+i(v_1+v_2))\\
    &=T(u_1+u_2)+iT(v_1+v_2)\\
    &=Tu_1+iTv_1+Tu_2+iTv_2\\
    &=T_{\mathbf{C}}(u_1+iv_1)+T_{\mathbf{C}}(u_2+iv_2)
\end{align*}
where the first and fourth equalities come from the definition of $T_{\mathbf{C}}$ (Definition 9.5).

\subsubsection*{homogeneity}
For $a,b\in\mathbf{R}$ and $u,v\in V$, we can write
\begin{align*}
    T_{\mathbf{C}}((a+bi)(u+iv))&=T_{\mathbf{C}}((au-bv)+i(av+bu))\\
    &=T(au-bv)+iT(av+bu)\\
    &=aTu-bTv+aiTv+biTu\\
    &=(a+bi)(Tu+iTv)\\
    &=(a+bi)(T_{\mathbf{C}}(u+iv))
\end{align*}
where the first and fourth equality come from the definition of complex scalar multiplication on $V_{\mathbf{C}}$ (Definition 9.2), and the second and fifth equalities come from the definition of $T_{\mathbf{C}}$ (Definition 9.5).

\clearpage

\section{Linear independence in $V$ and $V_{\mathbf{C}}$}
\subsection*{Problem statement}
Suppose $V$ is a real vector space and $v_1,\ldots,v_m\in V$. 
Prove that $v_1,\ldots,v_m\in V$ is linearly independent in $V_{\mathbf{C}}$ if and only if $v_1,\ldots,v_m\in V$ is linearly independent in $V$.

\subsection*{Solution}
\subsubsection*{First Direction}
Suppose $v_1,\ldots,v_m\in V$ is linearly independent in $V_{\mathbf{C}}$. 
This implies that for $\lambda_1,\ldots,\lambda_m\in\mathbf{C}$ and 
\[\lambda_1v_1+\cdots+\lambda_mv_m=0,\]
then $\lambda_1=\cdots=\lambda_m=0+0i$. 
Since any scalar $\lambda\in\mathbf{C}$ can be constructed from $a,b\in\mathbf{R}$ (Definition 1.1), it follows that for $a_1,\ldots,a_m\in\mathbf{R}$ and
\[a_1v_1+\cdots+a_mv_m=0,\]
then $a_1=\cdots=a_m=0$. 
Hence, $v_1,\ldots,v_m$ is linearly independent in $V$.

\subsubsection*{Second Direction}
Suppose $v_1,\ldots,v_m\in V$ is linearly independent in $V$. 
Suppose $\lambda_1,\ldots,\lambda_m\in\mathbf{C}$ and 
\[\lambda_1v_1+\cdots+\lambda_mv_m=0.\]
Then the equation above implies that
\[(\operatorname{Re}\lambda_1)v_1+\cdots+(\operatorname{Re}\lambda_m)v_m=0\;\;\;\text{and}\;\;\;(\operatorname{Im}\lambda_1)v_1+\cdots+(\operatorname{Im}\lambda_m)v_m=0.\]
Since $\operatorname{Re}\lambda_j,\operatorname{Im}\lambda_j\in\mathbf{R}$ (via Definition 4.2) and $v_1,\ldots,v_m$ is linearly independent in $V$, it follows that $\operatorname{Re}\lambda_1=\cdots=\operatorname{Re}\lambda_m=0$ and $\operatorname{Im}\lambda_1=\cdots=\operatorname{Im}\lambda_m=0$. 
Thus we have $\lambda_1=\cdots=\lambda_m=0$. 
Hence, $v_1,\ldots,v_m$ is linearly independent in $V_{\mathbf{C}}$.


\clearpage

\section{Spanning list in $V$ and $V_{\mathbf{C}}$}
\subsection*{Problem statement}
Suppose $V$ is a real vector space and $v_1,\ldots,v_m\in V$. 
Prove that $v_1,\ldots,v_m\in V$ spans $V_{\mathbf{C}}$ if and only if $v_1,\ldots,v_m\in V$ spans $V$.

\subsection*{Solution}
\subsubsection*{First Direction}
Suppose $v_1,\ldots,v_m\in V$ spans $V_{\mathbf{C}}$. 
Via Theorem 2.31 (`Spanning list contains a basis'), the list $v_1,\ldots,v_m$ can be reduced to a basis of $V_{\mathbf{C}}$. 
Suppose that this reduced list is of length $n$, implying that $\operatorname{dim}V_{\mathbf{C}}=n$ and, following from Theorem 9.4(b) (`Basis of $V$ is a basis of $V_{\mathbf{C}}$'), $\operatorname{dim}V=n$.

This basis of $V_{\mathbf{C}}$ is linearly independent and hence, following from Exercise 9.A(3), is linearly independent in $V$. 
Since it is a linearly independent list of length $n=\operatorname{dim}V$, Theorem 2.39 (`Linearly independent list of the right length is a basis') implies that this list is a basis of $V$.

Putting it all together, we have proven that $v_1,\ldots,v_m$ contains a list that is a basis of $V$. 
Therefore, $v_1,\ldots,v_m$ spans $V$.

\subsubsection*{Second Direction}
Suppose $v_1,\ldots,v_m\in V$ spans $V$. 
Theorem 2.31 implies that $v_1,\ldots,v_m$ can be reduced to a basis of $V$. 
Following Theorem 9.4(a), this basis of $V$ is a basis of $V_{\mathbf{C}}$. 
Hence, $v_1,\ldots,v_m$ contains a list that is a basis of $V_{\mathbf{C}}$. 
Therefore, $v_1,\ldots,v_m$ spans $V_{\mathbf{C}}$. 

\clearpage

\section{Complexification is a linear map}
\subsection*{Problem statement}
Suppose that $V$ is a real vector space and $S,T\in\mathcal{L}(V)$. 
Show that \newline
$(S+T)_{\mathbf{C}}=S_{\mathbf{C}}+T_{\mathbf{C}}$ and that $(\lambda T)_{\mathbf{C}}=\lambda T_{\mathbf{C}}$ for every $\lambda\in\mathbf{R}$.

\subsection*{Solution}
\subsubsection*{additivity}
To show that $(S+T)_{\mathbf{C}}=S_{\mathbf{C}}+T_{\mathbf{C}}$, we can write
\begin{align*}
    (S+T)_{\mathbf{C}}(u+iv)&=(S+T)(u)+i(S+T)(v)\\
    &=Su+iSv+Tu+iTv\\
    &=S_{\mathbf{C}}(u+iv)+T_{\mathbf{C}}(u+iv)\\
    &=(S_{\mathbf{C}}+T_{\mathbf{C}})(u+iv)
\end{align*}
for $v,u\in\mathbf{R}$.

\subsubsection*{homogeneity}
To show that $(\lambda T)_{\mathbf{C}}=\lambda T_{\mathbf{C}}$ for every $\lambda\in\mathbf{R}$, we can write
\begin{align*}
    (\lambda T)_{\mathbf{C}}(u+iv)&=(\lambda T)(u)+i(\lambda T)(v)\\
    &=\lambda (Tu)+\lambda(iTv)\\
    &=\lambda(Tu+iTv)\\
    &=\lambda T_{\mathbf{C}}(u+iv)
\end{align*}
for $v,u\in\mathbf{R}$.

\subsection*{Notes}
This result allows us to prove Lemma \ref{lemma:complexification_isomorphic} (`Complexification is isomorphism from $\mathcal{L}(V)$ to $\mathcal{L}(V_{\mathbf{C}})$').

\clearpage

\section{Prove $T_{\mathbf{C}}$ is invertible iff $T$ is invertible}
\subsection*{Problem statement}
Suppose $V$ is a real vector space and $T\in\mathcal{L}(V)$. Prove that $T_{\mathbf{C}}$ is invertible if and only if $T$ is invertible.

\subsection*{Solution}
\subsubsection{Lemma: Complexification is isomorphism from $\mathcal{L}(V)$ to $\mathcal{L}(V_{\mathbf{C}})$}\label{lemma:complexification_isomorphic}
Via Theorem 9.4(b) (`Basis of $V$ is a basis of $V_{\mathbf{C}}$'), we know that $\operatorname{dim}V=\operatorname{dim}V_{\mathbf{C}}$. 
Hence, Theorem 3.61 (`$\operatorname{dim}\mathcal{L}(V,W)=(\operatorname{dim}V)(\operatorname{dim}W)$') implies
\begin{equation*}
    \operatorname{dim}\mathcal{L}(V)=(\operatorname{dim}V)(\operatorname{dim}V)=(\operatorname{dim}V_{\mathbf{C}})(\operatorname{dim}V_{\mathbf{C}})=\operatorname{dim}\mathcal{L}(V_{\mathbf{C}}).
\end{equation*}
Thus, Theorem 3.59 (`Dimension shows whether vector spaces are isomorphic') tells us that $\mathcal{L}(V)$ and $\mathcal{L}(V_{\mathbf{C}})$ are isomorphic since they have the same dimension.

It is now clear to see that the \textbf{complexification} of $T$ is an isomorphism from $\mathcal{L}(V)$ to $\mathcal{L}(V_{\mathbf{C}})$. 
This follows from the observation that the \textbf{complexification} of operators is injective, $T_{\mathbf{C}}=0$ iff $T=0$, and surjective, the \textbf{complexification} of a basis of operators in $\mathcal{L}(V)$ yields a linearly independent list of operators in $\mathcal{L}(V_{\mathbf{C}})$ of length $\operatorname{dim}\mathcal{L}(V_{\mathbf{C}})$\footnote{
This could use a more rigorous proof, but that's a labor for another time
}. 
Therefore, for every operator $S\in\mathcal{L}(V_{\mathbf{C}})$, there exists a unique operator $R\in\mathcal{L}(V)$ such that $R_{\mathbf{C}}=S$.

\subsubsection*{First Direction}
Suppose $T_{\mathbf{C}}$ is invertible. 
Thus there exists a unique inverse $S\in\mathcal{L}(V_{\mathbf{C}})$ such that $ST_{\mathbf{C}}=I$ and $T_{\mathbf{C}}S=I$. 
From Lemma \ref{lemma:complexification_isomorphic}, there exists $R\in\mathcal{L}(V)$ such that $R_{\mathbf{C}}=S$. 
For $u,v\in V$, we can write
\begin{align*}
    u+iv&=I(u+iv)=ST_{\mathbf{C}}(u+iv)\\
    &=R_{\mathbf{C}}(Tu+iTv)=RTu+iRTv
\end{align*}
implying that $RT=I$ for all vectors in $v$. 
Since $R,T$ are operators, Exercise 3.D(10) (`$ST=I$ iff $TS=I$') allows us to state that $TR=I$. 
Hence, $T$ has an inverse and $T$ is invertible.

\subsubsection*{Second Direction}
Suppose $T$ is invertible. Thus there exists a unique inverse $R\in V$ such that $RT=I$ and $TR=I$. 
We will show that $R_{\mathbf{C}}$ is the inverse of $T_{\mathbf{C}}$. 

For $u,v\in V$, we can write 
\begin{equation*}
    R_{\mathbf{C}}T_{\mathbf{C}}(u+iv)=R_{\mathbf{C}}(Tu+iTv)=RTu+iRTv=u+iv
\end{equation*}
and
\begin{equation*}
    T_{\mathbf{C}}R_{\mathbf{C}}(u+iv)=T_{\mathbf{C}}(Ru+iRv)=TRu+iTRv=u+iv.
\end{equation*}
Hence, $T_{\mathbf{C}}R_{\mathbf{C}}=I$ and $R_{\mathbf{C}}T_{\mathbf{C}}=I$. Therefore, $T_{\mathbf{C}}$ is invertible.

\clearpage

\section{Prove $N_{\mathbf{C}}$ is nilpotent iff $N$ is nilpotent}
\subsection*{Problem statement}
Suppose $V$ is a real vector space and $N\in\mathcal{L}(V)$. Prove that $N_{\mathbf{C}}$ is nilpotent if and only if $N$ is nilpotent.

\subsection*{Solution}
\subsubsection*{First Direction}
Suppose $N_{\mathbf{C}}$ is nilpotent. 
This implies that $N_{\mathbf{C}}^{\dim V}=0$. 
Via Definition 9.5 (`complexification of $T$'), we can express $N_{\mathbf{C}}^{\dim V}$ as
\[N_{\mathbf{C}}^{\dim V}(u+iv)=N^{\dim V}u+iN^{\dim V}v\]
for all $u,v\in V$. 
Since the left side of our expression above equals zero, it follows that
\[N^{\dim V}u+iN^{\dim V}v=0+i0\]
for all $u,v\in V$, implying that $N$ is nilpotent.

\subsubsection*{Second Direction}
Suppose $N$ is nilpotent. 
This implies that $N^{\dim V}=0$. 
Via Definition 9.5, we can express $N_{\mathbf{C}}^{\dim V}$ as
\[N_{\mathbf{C}}^{\dim V}(u+iv)=N^{\dim V}u+iN^{\dim V}v\]
for all $u,v\in V$. 
Since the right side of our expression above equals zero, it follows that
\[N_{\mathbf{C}}^{\dim V}(u+iv)=0+i0\]
for all $u,v\in V$, implying that $N_{\mathbf{C}}$ is nilpotent.

\clearpage

\section{Nonreal eigenvalues of $T_{\mathbf{C}}$}
\subsection*{Problem statement}
Suppose $T\in\mathcal{L}(\mathbf{R}^3)$ and $5,7$ are eigenvalues of $T$. Prove that $T_{\mathbf{C}}$ has no nonreal eigenvalues.

\subsection*{Solution}
Theorem 9.16 states that nonreal eigenvalues of $T_{\mathbf{C}}$ come in pairs. 
Hence, if $T_{\mathbf{C}}$ had a nonreal eigenvalue $\lambda$, then it would also have the eigenvalue of $\bar{\lambda}$ (where $\lambda\neq\bar{\lambda}$). 
However, this would imply that $T_{\mathbf{C}}$ has 4 distinct eigenvalues, more than $\operatorname{dim}\mathbf{C}^3=3$, which is a contradiction of Theorem 5.13 (`Number of eigenvalues'). 
Therefore, if $T\in\mathcal{L}(\mathbf{R}^3)$ and $5,7$ are eigenvalues of $T$, then $T_{\mathbf{C}}$ has no nonreal eigenvalues.

\clearpage

\section{$T\in\mathcal{L}(\mathbf{R}^7)$ such that $T^2+T+I$ is nilpotent}
\subsection*{Problem statement}
Prove there does not exist an operator $T\in\mathcal{L}(\mathbf{R}^7)$ such that $T^2+T+I$ is nilpotent.

\subsection*{Solution}
Suppose there exists some operator $T\in\mathcal{L}(\mathbf{R}^7)$ such that $T^2+T+I$ is nilpotent. 
Via Theorem 8.18 (`Nilpotent operator raised to dimension of domain is $0$'), it follows that $(T^2+T+I)^7=0$ and for the polynomial $q(z)=(z^2+z+1)^7$ that $q(T)=0$. 
Via Theorem 8.46 (`$q(T)=0$ implies $q$ is a multiple of the minimal polynomial'), we can infer that $q$ is a polynomial multiple of the minimal polynomial of $T$. 
Via Theorem 9.19 (`Operator on odd-dimensional vector space has eigenvalue'), it follows that $T$ has a real eigenvalue, and via Theorem 8.49 (`Eigenvalues are the zeros of the minimal polynomial'), that eigenvalue is a zero of the minimal polynomial of $T$.

However, we can see that the polynomial $z^2+z+1$ has no real zeros\footnote{
This follows from Theorem 4.16 (`Factorization of a quadratic polynomial') and the simple observation that $1^2\ngeq 4(1)$
}. 
Thus, the polynomial $q(z)=(z^2+z+1)^7$ has no real zeros and cannot be a polynomial multiple of the minimal polynomial of $T$ since the minimal polynomial of $T$ is guaranteed to have a real zero. 
Therefore, we have a contradiction and it follows that no such operator $T\in\mathcal{L}(\mathbf{R}^7)$ exists such that $T^2+T+I$ is nilpotent.

\clearpage

\section{$T\in\mathcal{L}(\mathbf{C}^7)$ such that $T^2+T+I$ is nilpotent}
\subsection*{Problem statement}
Give an example of an operator $T\in\mathcal{L}(\mathbf{C}^7)$ such that $T^2+T+I$ is nilpotent.

\subsection*{Solution}
Let the operator $T\in\mathcal{L}(\mathbf{C}^7)$ be defined by
\[Tv=\frac{-1+\sqrt{3}i}{2}v.\]
For every $v\in\mathbf{C}^7$, we can write
\begin{align*}
    (T^2+T+I)v&=T^2v+Tv+Iv\\
    &=(\frac{-1+\sqrt{3}i}{2})^2v+\frac{-1+\sqrt{3}i}{2}v+v\\
    &=\frac{-2-2\sqrt{3}i}{4}v+\frac{-2+2\sqrt{3}i}{4}v+v\\
    &=-v+v=0
\end{align*}
and hence, $T^2+T+I=0$. The $0$ operator is clearly nilpotent, so it follows that $T^2+T+I$ is nilpotent.


\end{document}