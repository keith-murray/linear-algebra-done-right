\documentclass{article}
\usepackage{graphicx}
\usepackage{amsmath}
\usepackage{hyperref}
\usepackage{epigraph} 

\title{Linear Algebra Done Right\\Solutions to Exercises 6.A}
\author{}
\date{}

\providecommand{\abs}[1]{\lvert#1\rvert} \providecommand{\norm}[1]{\lVert#1\rVert}

\begin{document}

\maketitle

\epigraph{Algebra is but written geometry and geometry is but figured algebra.}{\textit{Sophie Germain}}

\clearpage

\renewcommand{\thesection}{3}
\section{Zeros of $p$ are eigenvalues of $T$}
\subsection*{Problem statement}
Suppose $\mathbf{F}=\mathbf{R}$ and $V\neq \{0\}$. Replace the positivity condition (which states that $\langle v, v\rangle\geq 0$ and for all $v\in V$) in the definition of an inner product (6.3) with the condition that $\langle v, v\rangle > 0$ for some $v\in V$. Show that this change in the definition does not change the set of functions from $V\times V$ to $\mathbf{R}$ that are inner products on $V$.

\subsection*{Solution}
The set of functions from $V\times V$ to $\mathbf{R}$ that are inner products on $V$ clearly does not grow smaller. Thus we must show that the set does not grow larger.

Suppose $v\in V$ is such that it satisfies the ``new'' condition that $\langle v, v\rangle > 0$. There are two cases to consider: $V=\operatorname{span}(v)$ and $V\neq\operatorname{span}(v)$.
\\
\\
$\mathbf{{V=\operatorname{\textbf{span}}(v)}}$: For any vector $u\in V$ there exists $\lambda\in\mathbf{R}$ such that $u=\lambda v$. Hence 
\[\langle u, u\rangle =\langle \lambda v, \lambda v \rangle = \abs{\lambda}^2\langle v, v\rangle \]
Given $\langle v, v\rangle > 0$ and $|\lambda|^2\geq 0$ for all $\lambda\in\mathbf{R}$, it follows that 
\[\langle u, u\rangle = |\lambda|^2\langle v, v\rangle \geq 0\]
Thus we've shown that the positivity condition is fulfilled and the set of function that are inner products on $V$ clearly does not grow larger.
\\
\\
$\mathbf{V\neq\operatorname{\textbf{span}}(v)}$: Since $V\neq \{0\}$ and $V\neq\operatorname{span}(v)$, there exists $u\in V$ that is not a scalar multiple of $v$. Theorem 6.14 (`An orthogonal decomposition') allows us to decompose $u$ into $u=cv+w$ where $v$ and $w$ are orthogonal. From the Pythagorean theorem, it follows
\[\norm{v+w}^2=\norm{v}^2+\norm{w}^2\]

Let's assume that $||w||^2 < 0$. Now consider the vector $\lambda w$ where $\lambda=\frac{\norm{v}}{\norm{w}}$\footnote{The $\abs{\frac{\norm{v}}{\norm{w}}}^2 \norm{w}^2$ term is quite harry, but note that $\norm{w}^2 < 0$ and $\abs{\norm{w}}^2 > 0$, therefore $\frac{\norm{w}^2}{\abs{\norm{w}}^2} = -1 $. Also note $\abs{\norm{v}}^2 = \norm{v}^2$ since $\langle v, v\rangle > 0$.}
\[\norm{v+\lambda w}^2=\norm{v}^2+\norm{\lambda w}^2 = \norm{v}^2+\abs{\frac{\norm{v}}{\norm{w}}}^2 \norm{w}^2 = \norm{v}^2 - \norm{v}^2 = 0 \]
Thus we have a contradiction since $v+\lambda w\neq0$ yet $||v+\lambda w||^2=0$, defying the condition of definiteness (definition 6.3). Therefore $||w||^2 > 0$.

It then follows that 
\[||u||^2 = ||v+w||^2=||v||^2+||w||^2 \geq 0\]
showing that the positivity condition is fulfilled and the set of function that are inner products on $V$ clearly does not grow larger.

\end{document}