\documentclass{article}
\usepackage{graphicx}
\usepackage{amsmath}
\usepackage{amsfonts}
\usepackage{hyperref}
\usepackage{epigraph} 

\title{Linear Algebra Done Right\\Solutions to Exercises 6.C}
\author{}
\date{}

\providecommand{\abs}[1]{\lvert#1\rvert} \providecommand{\norm}[1]{\lVert#1\rVert}

\begin{document}

\maketitle

\section{Orthogonal complement of subset}
\subsection*{Problem statement}
Suppose $v_1,\ldots,v_m\in V$. 
Prove that 
\[\{v_1,\ldots,v_m\}^\bot = \big( \operatorname{span}(v_1,\ldots,v_m) \big)^\bot.\]

\subsection*{Solution}
To prove our desired result, let's first show $\{v_1,\ldots,v_m\}^\bot \subset \big( \operatorname{span}(v_1,\ldots,v_m) \big)^\bot$, and then $\big( \operatorname{span}(v_1,\ldots,v_m) \big)^\bot \subset \{v_1,\ldots,v_m\}^\bot$.

\subsubsection*{First Direction}
Suppose $u\in \{v_1,\ldots,v_m\}^\bot$. 
This implies $\langle u, v_j\rangle=0$ for $j=1,\ldots,m$. 
We can write every $v\in\operatorname{span}(v_1,\ldots,v_m)$ as
\[v=a_1 v_1+\cdots + a_m v_m\]
for $a_1,\ldots,a_m\in\mathbf{F}$.
It follows that
\[\langle u,v\rangle = \bar{a}_1\langle u,v_1\rangle+\cdots+ \bar{a}_m\langle u,v_m\rangle=0.\]
Hence, $u\in \big( \operatorname{span}(v_1,\ldots,v_m) \big)^\bot$.

\subsubsection*{Second Direction}
Suppose $u\in \big( \operatorname{span}(v_1,\ldots,v_m) \big)^\bot$. 
Since $v_1,\ldots,v_m\in \operatorname{span}(v_1,\ldots,v_m)$, it follows that $\langle u,v_j\rangle = 0$ for $j=1,\ldots,m$. 
Hence, $u\in \{v_1,\ldots,v_m\}^\bot$.

\clearpage

\section{$U^\bot=\{0\}$ iff $U=V$}
\subsection*{Problem statement}
Suppose $U$ is a finite-dimensional subspace of $V$. 
Prove that $U^\bot=\{0\}$ if and only if $U=V$.

\subsection*{Solution}
\subsubsection*{First Direction}
Suppose $U^\bot=\{0\}$. 
Via Theorem 6.51 (`The orthogonal complement of the orthogonal complement') and Theorem 6.46(b) (`Basic properties of orthogonal complement'), we can write
\[U=(U^\bot)^\bot=\{0\}^\bot=V.\]
Hence, $U=V$.

\subsubsection*{Second Direction}
Suppose $U=V$. 
Via Theorem 6.46(c) (`Basic properties of orthogonal complement'), we can write
\[U^\bot=V^\bot=\{0\}.\]
Hence, $U^\bot=\{0\}$.

\clearpage

\section{$U^\bot=\{0\}$ iff $U=V$}
\subsection*{Problem statement}
Suppose $U$ is a subspace of $V$ with basis $u_1,\ldots,u_m$ and
\[u_1,\ldots,u_m,w_1,\ldots,w_n\]
is a basis of $V$. 
Prove that if the Gram-Schmidt Procedure is applied to the basis of $V$ above, producing a list $e_1,\ldots,e_m,f_1,\ldots,f_n$, then $e_1,\ldots,e_m$ is an orthonormal basis of $U$ and $f_1,\ldots,f_n$ is an orthonormal basis of $U^\bot$.

\subsection*{Solution}
Via the Gram-Schmidt Procedure (Theorem 6.31), we can state
\[\operatorname{span}(u_1,\ldots,u_m)=\operatorname{span}(e_1,\ldots,e_m).\]
Thus, $e_1,\ldots,e_m$ is an orthonormal basis of $U$. 

Via Theorem 6.50 (`Dimension of the orthogonal complement'), we have 
\[\dim V= \dim U+\dim U^\bot\]
which implies $\dim U^\bot=n$. 
Given $e_1,\ldots,e_m,f_1,\ldots,f_n$ is an orthonormal list, it follows that $\langle e_j, f_k\rangle=0$ for $j=1,\ldots,m$ and $k=1,\ldots,n$. 
Hence, \newline$f_j\in\{ e_1,\ldots, e_m\}^\bot$ and, via Exercise 6.C(1), $f_j\in\big(\operatorname{span}(e_1,\ldots,e_m)\big)^\bot$.
Therefore, $f_j\in U^\bot$ for $j=1,\ldots,n$.

Given $\dim U^\bot=n$ and $f_1,\ldots,f_n$ is an orthonormal list of length $n$ in $U^\bot$, it follows that $f_1,\ldots,f_n$ is an orthonormal basis of $U^\bot$.


\end{document}