\documentclass{article}
\usepackage{graphicx}
\usepackage{amsmath}
\usepackage{hyperref}
\usepackage{epigraph} 

\title{Linear Algebra Done Right\\Solutions to Exercises 6.B}
\author{}
\date{}

\providecommand{\abs}[1]{\lvert#1\rvert} \providecommand{\norm}[1]{\lVert#1\rVert}

\begin{document}

\maketitle

\section{Unit vectors in $\mathbf{R}^2$}
\subsection*{Problem statement}
\begin{itemize}
    \item[(a)] Suppose $\theta\in\mathbf{R}$. Show that $(\cos\theta,\sin\theta),(-\sin\theta,\cos\theta)$ and \newline$(\cos\theta,\sin\theta),(\sin\theta,-\cos\theta)$ are orthonormal bases of $\mathbf{R}^2$.
    \item[(b)] Show that each orthonormal basis of $\mathbf{R}^2$ is of the form given by one of the two possibilities of part (a).
\end{itemize}

\subsection*{Solution}
\subsubsection*{Part (a)}
Suppose the inner product is the Euclidean dot product (Example 6.4(a)). 
We can write
\[\langle (\cos\theta,\sin\theta), (-\sin\theta,\cos\theta)\rangle=-(\cos\theta)(\sin\theta)+(\sin\theta)(\cos\theta)=0\]
to show that $(\cos\theta,\sin\theta),(-\sin\theta,\cos\theta)$ are orthogonal. 
To show that \newline$\norm{(\cos\theta,\sin\theta)}=1$, we can write
\[\langle (\cos\theta,\sin\theta), (\cos\theta,\sin\theta)\rangle=\cos^2\theta+\sin^2\theta=1\]
and to show that $\norm{(-\sin\theta,\cos\theta)}=1$, we can write
\[\langle (-\sin\theta,\cos\theta), (-\sin\theta,\cos\theta)\rangle=\sin^2\theta+\cos^2\theta=1.\]
The same reasoning can be applied for $(\cos\theta,\sin\theta),(\sin\theta,-\cos\theta)$.

\subsubsection*{Part (b)}
If we think of vectors in $\mathbf{R}^2$ as arrows, then for $v\in\mathbf{R}^2$, the vector $v/\norm{v}$ is a vector on the unit circle and can be expressed as $(\cos\theta,\sin\theta)$ for some $\theta\in\mathbf{R}$. 
The space of vectors orthogonal to $(\cos\theta,\sin\theta)$ is the line expressed by $\operatorname{span}((-\sin\theta,\cos\theta))$, where the vector $(-\sin\theta,\cos\theta)$ was shown to be orthogonal to $(\cos\theta,\sin\theta)$ in part (a). 
This line intersects with the unit circle at two and only two points, namely $(\cos\theta,\sin\theta)$ and $(\sin\theta,-\cos\theta)$.

\clearpage

\section{Properties of vectors $v\in\operatorname{span}(e_1,\ldots,e_m)$}
\subsection*{Problem statement}
Suppose $e_1,\ldots,e_m$ is an orthonormal list of vectors in $V$. 


\end{document}