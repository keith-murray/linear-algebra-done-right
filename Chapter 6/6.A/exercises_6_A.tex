\documentclass{article}
\usepackage{graphicx}
\usepackage{amsmath}
\usepackage{hyperref}
\usepackage{epigraph} 

\title{Linear Algebra Done Right\\Solutions to Exercises 6.A}
\author{}
\date{}

\providecommand{\abs}[1]{\lvert#1\rvert} \providecommand{\norm}[1]{\lVert#1\rVert}

\begin{document}

\maketitle

\section{Inhomogeneous inner products}
\subsection*{Problem statement}
Show that the function that takes $\big((x_1,x_2),(y_1,y_2)\big)\in \mathbf{R}^2\times\mathbf{R}^2$ to $|x_1y_1|+|x_2y_2|$ is not an inner product on $\mathbf{R}^2$.

\subsection*{Solution}
The inner product is not homogeneous (Definition 6.3). 
Consider $\lambda=-2$. 
\begin{align*} 
\langle(-2x_1,-2x_2),(y_1,y_2)\rangle &= |-2x_1y_1|+|-2x_2y_2| \\ 
 &= 2|x_1y_1| + 2|x_2y_2|\\
 &= 2\langle(x_1,x_2),(y_1,y_2)\rangle\\
 &\neq -2\langle(x_1,x_2),(y_1,y_2)\rangle
\end{align*}

\clearpage

\section{Indefinite inner products}
\subsection*{Problem statement}
Show that the function that takes $\big((x_1,x_2,x_3),(y_1,y_2,y_3)\big)\in \mathbf{R}^3\times\mathbf{R}^3$\newline to $x_1y_1+x_3y_3$ is not an inner product on $\mathbf{R}^3$.

\subsection*{Solution}
The inner product does not obey definiteness (Definition 6.3). 
Consider\newline $(x_1,x_2,x_3)=(y_1,y_2,y_3)=(0,1,0)$.
\[\langle(x_1,x_2,x_3),(y_1,y_2,y_3)\rangle=\langle(0,1,0),(0,1,0)\rangle=0(0)+0(0)=0\]

\clearpage

\section{Replacing the positivity condition}
\subsection*{Problem statement}
Suppose $\mathbf{F}=\mathbf{R}$ and $V\neq \{0\}$. 
Replace the positivity condition (which states that $\langle v, v\rangle\geq 0$ and for all $v\in V$) in the definition of an inner product (6.3) with the condition that $\langle v, v\rangle > 0$ for some $v\in V$. 
Show that this change in the definition does not change the set of functions from $V\times V$ to $\mathbf{R}$ that are inner products on $V$.

\subsection*{Solution}
The set of functions from $V\times V$ to $\mathbf{R}$ that are inner products on $V$ clearly does not grow smaller. 
Thus we must show that the set does not grow larger.

Suppose $v\in V$ is such that it satisfies the ``new'' condition that $\langle v, v\rangle > 0$. 
There are two cases to consider: $V=\operatorname{span}(v)$ and $V\neq\operatorname{span}(v)$.
\\
\\
$\mathbf{{V=\operatorname{\textbf{span}}(v)}}$: For any vector $u\in V$ there exists $\lambda\in\mathbf{R}$ such that $u=\lambda v$. 
Hence 
\[\langle u, u\rangle =\langle \lambda v, \lambda v \rangle = \abs{\lambda}^2\langle v, v\rangle \]
Given $\langle v, v\rangle > 0$ and $|\lambda|^2\geq 0$ for all $\lambda\in\mathbf{R}$, it follows that 
\[\langle u, u\rangle = |\lambda|^2\langle v, v\rangle \geq 0\]
Thus we've shown that the positivity condition is fulfilled and the set of function that are inner products on $V$ clearly does not grow larger.
\\
\\
$\mathbf{V\neq\operatorname{\textbf{span}}(v)}$: Since $V\neq \{0\}$ and $V\neq\operatorname{span}(v)$, there exists $u\in V$ that is not a scalar multiple of $v$. 
Theorem 6.14 (`An orthogonal decomposition') allows us to decompose $u$ into $u=cv+w$ where $v$ and $w$ are orthogonal. 
From the Pythagorean theorem, it follows
\[\norm{v+w}^2=\norm{v}^2+\norm{w}^2\]

Let's assume that $||w||^2 < 0$. 
Now consider the vector $\lambda w$ where $\lambda=\frac{\norm{v}}{\norm{w}}$\footnote{
The $\abs{\frac{\norm{v}}{\norm{w}}}^2 \norm{w}^2$ term is quite harry, but note that $\norm{w}^2 < 0$ and $\abs{\norm{w}}^2 > 0$, therefore $\frac{\norm{w}^2}{\abs{\norm{w}}^2} = -1 $. Also note $\abs{\norm{v}}^2 = \norm{v}^2$ since $\langle v, v\rangle > 0$.
}
\[\norm{v+\lambda w}^2=\norm{v}^2+\norm{\lambda w}^2 = \norm{v}^2+\abs{\frac{\norm{v}}{\norm{w}}}^2 \norm{w}^2 = \norm{v}^2 - \norm{v}^2 = 0 \]
Thus we have a contradiction since $v+\lambda w\neq0$ yet $||v+\lambda w||^2=0$, defying the condition of definiteness (definition 6.3). Therefore $||w||^2 > 0$.

It then follows that 
\[||u||^2 = ||v+w||^2=||v||^2+||w||^2 \geq 0\]
showing that the positivity condition is fulfilled and the set of function that are inner products on $V$ clearly does not grow larger.

\clearpage

\section{Diagonals of a rhombus are perpendicular}
\subsection*{Problem statement}
Suppose $V$ is a real inner product space.
\begin{itemize}
    \item[(a)] Show that $\langle u+v,u-v\rangle=\norm{u}^2 - \norm{v}^2$ for every $u,v\in V$.
    \item[(b)] Show that if $u,v\in V$ have the same norm, then $u+v$ is orthogonal to $u-v$.
    \item[(c)] Use part (b) to show that the diagonals of a rhombus are perpendicular to each other.
\end{itemize}

\subsection*{Solution}

\subsubsection*{a}
Via the definition of the inner product (Definition 6.3) and Theorem 6.7 (`Basic properties of an inner product'), we can expand $\langle u+v,u-v\rangle$ to
\[\langle u+v,u-v\rangle=\langle u,u\rangle+\langle v,u\rangle + \langle u,-v\rangle + \langle v,-v\rangle.\]

Since $V$ is a real inner product space, we have
\[\langle u,-v\rangle=-1\langle u,v\rangle \;\;\;\text{and}\;\;\; \langle v,-v\rangle=-\norm{v}^2\]
and since $\langle u,v\rangle=\overline{\langle v,u\rangle}=\langle v,u \rangle$ for reals, we can write
\[\langle u+v,u-v\rangle=\norm{u}^2+\langle v,u\rangle -\langle v,u \rangle - \norm{v}^2=\norm{u}^2 - \norm{v}^2.\]

\subsubsection*{b}
If $\norm{u} = \norm{v}$, then $\norm{u}^2 = \norm{v}^2$. 
Hence, via part (a), we can write
\[\langle u+v,u-v\rangle=\norm{u}^2 - \norm{v}^2=\norm{u}^2 -\norm{u}^2=0\]
and $u+v$ is orthogonal to $u-v$.

\subsubsection*{c}
In a rhombus, all sides are the same length, and hence, have the same norm. 
Suppose $u,v\in V$ are the sides of the rhombus and $\norm{u} = \norm{v}$. 
We can directly apply our result from part (b) to state $\langle u+v,u-v\rangle=0$, implying that the diagonals of the rhombus are orthogonal, and thus, perpendicular to each other.

\end{document}