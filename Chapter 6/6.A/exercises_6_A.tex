\documentclass{article}
\usepackage{graphicx}
\usepackage{amsmath}
\usepackage{hyperref}
\usepackage{epigraph} 

\title{Linear Algebra Done Right\\Solutions to Exercises 6.A}
\author{}
\date{}

\providecommand{\abs}[1]{\lvert#1\rvert} \providecommand{\norm}[1]{\lVert#1\rVert}

\begin{document}

\maketitle

\section{Inhomogeneous inner products}
\subsection*{Problem statement}
Show that the function that takes $\big((x_1,x_2),(y_1,y_2)\big)\in \mathbf{R}^2\times\mathbf{R}^2$ to $|x_1y_1|+|x_2y_2|$ is not an inner product on $\mathbf{R}^2$.

\subsection*{Solution}
The inner product is not homogeneous (Definition 6.3). 
Consider $\lambda=-2$. 
\begin{align*} 
\langle(-2x_1,-2x_2),(y_1,y_2)\rangle &= |-2x_1y_1|+|-2x_2y_2| \\ 
 &= 2|x_1y_1| + 2|x_2y_2|\\
 &= 2\langle(x_1,x_2),(y_1,y_2)\rangle\\
 &\neq -2\langle(x_1,x_2),(y_1,y_2)\rangle
\end{align*}

\clearpage

\section{Indefinite inner products}
\subsection*{Problem statement}
Show that the function that takes $\big((x_1,x_2,x_3),(y_1,y_2,y_3)\big)\in \mathbf{R}^3\times\mathbf{R}^3$\newline to $x_1y_1+x_3y_3$ is not an inner product on $\mathbf{R}^3$.

\subsection*{Solution}
The inner product does not obey definiteness (Definition 6.3). 
Consider\newline $(x_1,x_2,x_3)=(y_1,y_2,y_3)=(0,1,0)$.
\[\langle(x_1,x_2,x_3),(y_1,y_2,y_3)\rangle=\langle(0,1,0),(0,1,0)\rangle=0\cdot 0+0\cdot 0=0\]

\clearpage

\section{Replacing the positivity condition}
\subsection*{Problem statement}
Suppose $\mathbf{F}=\mathbf{R}$ and $V\neq \{0\}$. 
Replace the positivity condition (which states that $\langle v, v\rangle\geq 0$ and for all $v\in V$) in the definition of an inner product (6.3) with the condition that $\langle v, v\rangle > 0$ for some $v\in V$. 
Show that this change in the definition does not change the set of functions from $V\times V$ to $\mathbf{R}$ that are inner products on $V$.

\subsection*{Solution}
The set of functions from $V\times V$ to $\mathbf{R}$ that are inner products on $V$ clearly does not grow smaller. 
Thus we must show that the set does not grow larger.

Suppose $v\in V$ is such that it satisfies the ``new'' condition that $\langle v, v\rangle > 0$. 
There are two cases to consider: $V=\operatorname{span}(v)$ and $V\neq\operatorname{span}(v)$.
\\
\\
$\mathbf{{V=\operatorname{\textbf{span}}(v)}}$: For any vector $u\in V$ there exists $\lambda\in\mathbf{R}$ such that $u=\lambda v$. 
Hence 
\[\langle u, u\rangle =\langle \lambda v, \lambda v \rangle = \abs{\lambda}^2\langle v, v\rangle \]
Given $\langle v, v\rangle > 0$ and $|\lambda|^2\geq 0$ for all $\lambda\in\mathbf{R}$, it follows that 
\[\langle u, u\rangle = |\lambda|^2\langle v, v\rangle \geq 0\]
Thus we've shown that the positivity condition is fulfilled and the set of function that are inner products on $V$ clearly does not grow larger.
\\
\\
$\mathbf{V\neq\operatorname{\textbf{span}}(v)}$: Since $V\neq \{0\}$ and $V\neq\operatorname{span}(v)$, there exists $u\in V$ that is not a scalar multiple of $v$. 
Theorem 6.14 (`An orthogonal decomposition') allows us to decompose $u$ into $u=cv+w$ where $v$ and $w$ are orthogonal. 
From the Pythagorean theorem, it follows
\[\norm{v+w}^2=\norm{v}^2+\norm{w}^2\]

Let's assume that $||w||^2 < 0$. 
Now consider the vector $\lambda w$ where $\lambda=\frac{\norm{v}}{\norm{w}}$\footnote{
The $\abs{\frac{\norm{v}}{\norm{w}}}^2 \norm{w}^2$ term is quite harry, but note that $\norm{w}^2 < 0$ and $\abs{\norm{w}}^2 > 0$, therefore $\frac{\norm{w}^2}{\abs{\norm{w}}^2} = -1 $. Also note $\abs{\norm{v}}^2 = \norm{v}^2$ since $\langle v, v\rangle > 0$.
}
\[\norm{v+\lambda w}^2=\norm{v}^2+\norm{\lambda w}^2 = \norm{v}^2+\abs{\frac{\norm{v}}{\norm{w}}}^2 \norm{w}^2 = \norm{v}^2 - \norm{v}^2 = 0 \]
Thus we have a contradiction since $v+\lambda w\neq0$ yet $||v+\lambda w||^2=0$, defying the property of definiteness (Definition 6.3). Therefore $||w||^2 \geq 0$.

It then follows that 
\[||u||^2 = ||v+w||^2=||v||^2+||w||^2 \geq 0\]
showing that the positivity condition is fulfilled and the set of function that are inner products on $V$ does not grow larger.

\clearpage

\section{Diagonals of a rhombus are perpendicular}
\subsection*{Problem statement}
Suppose $V$ is a real inner product space.
\begin{itemize}
    \item[(a)] Show that $\langle u+v,u-v\rangle=\norm{u}^2 - \norm{v}^2$ for every $u,v\in V$.
    \item[(b)] Show that if $u,v\in V$ have the same norm, then $u+v$ is orthogonal to $u-v$.
    \item[(c)] Use part (b) to show that the diagonals of a rhombus are perpendicular to each other.
\end{itemize}

\subsection*{Solution}

\subsubsection*{a}
Via the definition of the inner product (Definition 6.3) and Theorem 6.7 (`Basic properties of an inner product'), we can expand $\langle u+v,u-v\rangle$ to
\[\langle u+v,u-v\rangle=\langle u,u\rangle+\langle v,u\rangle + \langle u,-v\rangle + \langle v,-v\rangle.\]

Since $V$ is a real inner product space, we have
\[\langle u,-v\rangle=-1\langle u,v\rangle \;\;\;\text{and}\;\;\; \langle v,-v\rangle=-\norm{v}^2\]
and since $\langle u,v\rangle=\overline{\langle v,u\rangle}=\langle v,u \rangle$ for reals, we can write
\[\langle u+v,u-v\rangle=\norm{u}^2+\langle v,u\rangle -\langle v,u \rangle - \norm{v}^2=\norm{u}^2 - \norm{v}^2.\]

\subsubsection*{b}
If $\norm{u} = \norm{v}$, then $\norm{u}^2 = \norm{v}^2$. 
Hence, via part (a), we can write
\[\langle u+v,u-v\rangle=\norm{u}^2 - \norm{v}^2=\norm{u}^2 -\norm{u}^2=0\]
and $u+v$ is orthogonal to $u-v$.

\subsubsection*{c}
In a rhombus, all sides are the same length, and hence, have the same norm. 
Suppose $u,v\in V$ are the sides of the rhombus and $\norm{u} = \norm{v}$. 
We can directly apply our result from part (b) to state $\langle u+v,u-v\rangle=0$, implying that the diagonals of the rhombus are orthogonal, and thus, perpendicular to each other.

\clearpage

\section{No eigenvalues greater than 1}
\subsection*{Problem statement}
Suppose $T\in \mathcal{L}(V)$ is such that $\norm{Tv}\leq\norm{v}$ for every $v\in V$. 
Prove that $T-\sqrt{2}I$ is invertible.

\subsection*{Solution}
Theorem 5.6 (`Equivalent conditions to be an eigenvalue') implies that ``$T-\sqrt{2}I$ is invertible'' means that $\sqrt{2}$ is not an eigenvalue of $T$. 
Let's first note that $1<\sqrt{2}$. 

Suppose $v$ is an eigenvector of $T$. 
The condition that $\norm{Tv}\leq\norm{v}$ implies that the eigenvalue corresponding to $v$, let's say $\lambda$, we have
\[\abs{\lambda}\norm{v}\leq\norm{v},\]
and thus
\[\lambda\leq 1.\]
Hence, $\sqrt{2}$ cannot be an eigenvalue and $T-\sqrt{2}I$ is invertible.

\clearpage

\section{Condition equivalent to orthogonality}
\subsection*{Problem statement}
Suppose $u,v\in V$. 
Prove that $\langle u,v\rangle=0$ if and only if
\[\norm{u}\leq\norm{u+av}\]
for all $a\in\mathbf{F}$.

\subsection*{Solution}
\subsubsection*{First Direction}
Suppose $\langle u,v\rangle=0$. 
It follows that for all $a\in\mathbf{F}$, we have
\[\langle u,av\rangle=0.\]
Via the Pythagorean Theorem (Theorem 6.13), we can write
\[\norm{u}^2+\norm{av}^2=\norm{u+av}^2.\]
Given the positivity condition of inner products (Definition 6.3), it follows that
\[\norm{u}^2\leq\norm{u+av}^2.\]

\subsubsection*{Second Direction}
Suppose $\norm{u}\leq\norm{u+av}$ for all $a\in\mathbf{F}$. 
Via Theorem 6.14 (`An orthogonal decomposition'), we can decompose $u$ into $cv$ and $w$ where $c=\frac{\langle u,v\rangle}{\norm{v}^2}$, $w=u-cv$, and $\langle cv,w\rangle=0$. 
Invoking the Pythagorean Theorem (Theorem 6.13), we can write
\[\norm{u}^2=\norm{cv+w}^2=\norm{cv}^2+\norm{w}^2=\norm{cv}^2+\norm{u-cv}^2,\]
and via the positivity condition of inner products (Definition 6.3), it follows that
\[\norm{u}^2\geq\norm{u-cv}^2\]
and by taking the square root of both sides
\[\norm{u}\geq\norm{u-cv}.\]
In order to avoid a contradiction with our hypothesis, the expression above must be an equality. 
The expression above can only be an equality if $c=0$ or $v=0$. 
Both cases imply that $\langle u,v\rangle=0$.

\clearpage

\section{$\norm{au+bv}=\norm{bu+av}$ iff $\norm{u}=\norm{v}$}
\subsection*{Problem statement}
Suppose $u,v\in V$. 
Prove that $\norm{au+bv}=\norm{bu+av}$ for all $a,b\in \mathbf{R}$ if and only if $\norm{u}=\norm{v}$.

\subsection*{Solution}
As an organizational note, it's easier to work with squared norms, and the results directly translate to norms. 

\subsubsection*{First Direction}
Suppose $\norm{au+bv}^2=\norm{bu+av}^2$ for all $a,b\in \mathbf{R}$. 
Expanding out $\norm{au+bv}^2=\norm{bu+av}^2$, we have
\[\langle au,au\rangle+\langle bv,au\rangle + \langle au,bv\rangle + \langle bv,bv\rangle=\langle bu,bu\rangle+\langle av,bu\rangle + \langle bu,av\rangle + \langle av,av\rangle.\]
Since $a,b\in\mathbf{R}$, it follows that
\[\langle bv,au\rangle=b\bar{a}\langle v,u\rangle=ab\langle v,u\rangle,\]
and a similar result follows for $\langle au,bv\rangle,\langle av,bu\rangle,\langle bu,av\rangle$.
Hence, we can write
\[a^2\norm{v}^2+ab\langle v,u\rangle + ab\langle u,v\rangle + b^2\norm{u}^2=b^2\norm{v}^2+ab\langle v,u\rangle + ab\langle u,v\rangle + a^2\norm{u}^2,\]
and by subtracting out the similar terms, we have
\[a^2\norm{v}^2 + b^2\norm{u}^2=b^2\norm{v}^2 + a^2\norm{u}^2,\]
By setting $b=0$, it follows that
\[a^2\norm{v}^2=a^2\norm{u}^2,\]
which implies $\norm{u}^2=\norm{v}^2$ and $\norm{u}=\norm{v}$.

\subsubsection*{First Direction}
Suppose $\norm{u}^2=\norm{v}^2$. 
It follows that for all $a,b\in \mathbf{R}$, 
\[a^2\norm{v}^2=a^2\norm{u}^2 \;\;\;\text{and}\;\;\; b^2\norm{v}^2=b^2\norm{u}^2.\]
By adding the two expressions together, we have
\[a^2\norm{v}^2 + b^2\norm{u}^2=b^2\norm{v}^2 + a^2\norm{u}^2.\]
To construct $\norm{au+bv}^2=\norm{bu+av}^2$, one only needs to reverse our work in the \textbf{First Direction}.

\clearpage

\section{Properties of $u=v$}
\subsection*{Problem statement}
Suppose $u,v\in V$ and $\norm{u}=\norm{v}=1$ and $\langle u,v\rangle=1$. 
Prove that $u=v$.

\subsection*{Solution}
Via the Cauchy-Schwarz Inequality (Theorem 6.15), we know that
\[\abs{\langle u,v\rangle}\leq\norm{u}\norm{v},\]
where the inequality is an equality if and only if one of $u,v$ is a scalar multiple of the other. 
In our case, we have
\[1=\langle u,v\rangle = \norm{u}\norm{v}=1\cdot 1=1,\]
hence, we have an equality and $u=cv$ for some $c\in\mathbf{F}$. 
To compute the exact $c$, Theorem 6.14 (`An orthogonal decomposition') tells us that 
\[c=\frac{\langle u,v\rangle}{\norm{v}^2}=\frac{1}{1^2}=1.\]
Therefore, $u=v$.

\clearpage

\section{When $\norm{u}\leq 1$ and $\norm{v}\leq 1$}
\subsection*{Problem statement}
Suppose $u,v\in V$ and $\norm{u}\leq 1$ and $\norm{v}\leq 1$. 
Prove that
\[\sqrt{1-\norm{u}^2}\sqrt{1-\norm{v}^2}\leq 1-\abs{\langle u,v \rangle}.\]

\subsection*{Solution}
Via the Cauchy-Schwarz Inequality (Theorem 6.15) and the problem statement, we have
\[\abs{\langle u,v\rangle}\leq \norm{u}\norm{v}\leq 1.\]
By subtracting by $1$ and multiplying by $-1$, we have
\[0\leq 1-\norm{u}\norm{v}\leq 1- \abs{\langle u,v\rangle}.\]
Thus, if we can show that $\sqrt{1-\norm{u}^2}\sqrt{1-\norm{v}^2}\leq1-\norm{u}\norm{v}$, then our desired result immediately follows.

By squaring the expression $\sqrt{1-\norm{u}^2}\sqrt{1-\norm{v}^2}\leq 1-\norm{u}\norm{v}$ on both sides (which is allowed since both terms are real and greater than or equal to $0$), we have
\[(1-\norm{u}^2)(1-\norm{v}^2)\leq (1-\norm{u}\norm{v})^2\]
which can be expanded to
\[1-\norm{u}^2-\norm{v}^2+\norm{u}^2\norm{v}^2\leq 1-2\norm{u}\norm{v}+\norm{u}^2\norm{v}^2\]
and then reduced to
\[-\norm{u}^2-\norm{v}^2 \leq -2\norm{u}\norm{v}.\]
By rearranging terms, we can write
\[0\leq \norm{u}^2 - 2\norm{u}\norm{v} + \norm{v}^2,\]
and it follows that
\[0\leq (\norm{u} - \norm{v})^2\]
which, given the positivity property of inner products (Definition 6.3), is clearly true. 
Therefore, we can state that 
\[\sqrt{1-\norm{u}^2}\sqrt{1-\norm{v}^2}\leq 1-\norm{u}\norm{v}\]
and our desired result immediately follows.

\clearpage

\section{Searching for vectors in $\mathbf{R}^2$}
\subsection*{Problem statement}
Find vectors $u,v\in \mathbf{R}^2$ such that $u$ is a scalar multiple of $(1,3)$, $v$ is orthogonal to $(1,3)$, and $(1,2)=u+v$.

\subsection*{Solution}
First, let's find a vector orthogonal to $(1,3)$. 
By inspection, the vector $(-3,1)$ will work. 
Now we need to find $a,b\in\mathbf{R}$ such that $u=a(1,3)$, $v=b(-3,1)$, and $(1,2)=u+v$. 
Using the constraint $(1,2)=u+v$, we get the following equations
\begin{align*} 
1=a-3b &\Rightarrow 3a=9b+3, \\ 
2=3a+b &\Rightarrow 3a=2-b,
\end{align*}
and through combining them, we have
\[9b+3=2-b\]
and $b=-\frac{1}{10}$. 
Plugging in $b$, we have $a=\frac{7}{10}$. 
Hence, our vectors are
\[u=(\frac{7}{10},\frac{21}{10}) \;\;\;\text{and}\;\;\; v=(\frac{3}{10},-\frac{1}{10}).\]

\clearpage

\section{Using the Cauchy-Schwarz Inequality: Act 1}
\subsection*{Problem statement}
Prove that
\[16\leq (a+b+c+d)(\frac{1}{a}+\frac{1}{b}+\frac{1}{c}+\frac{1}{d})\]
for all positive numbers $a,b,c,d$.

\subsection*{Solution}
First notice that for \textit{positive} numbers $a,b,c,d$ there exist positive numbers $x,y,z,w$ such that
\[x^2=a,\quad y^2=b,\quad z^2=c,\quad w^2=d.\]
Now we can use Example 6.17(a) of Cauchy-Schwarz Inequalities that states if $x_1,\ldots,x_n,y_1,\ldots,y_n\in\mathbf{R}$, then
\[\abs{x_1y_1+\cdots+x_ny_n}^2\leq (x_1^2+\cdots+x_n^2)(y_1^2+\cdots+y_n^2).\]
Given the expression above, we can write
\[\abs{x(\frac{1}{x})+y(\frac{1}{y})+z(\frac{1}{z})+w(\frac{1}{w})}^2\leq (x^2+y^2+z^2+w^2)(\frac{1}{x^2}+\frac{1}{y^2}+\frac{1}{z^2}+\frac{1}{w^2})\]
where $\frac{1}{x},\frac{1}{y},\frac{1}{z},\frac{1}{w}$ are well defined since $a,b,c,d$ are \textit{positive} numbers\footnote{0 is not a positive number.}. Applying our substitution and reducing the left-hand side of the equation, we have
\[\abs{1+1+1+1}^2=16\leq (a+b+c+d)(\frac{1}{a}+\frac{1}{b}+\frac{1}{c}+\frac{1}{d}),\]
which was to be shown.

\clearpage

\section{Using the Cauchy-Schwarz Inequality: Act 2}
\subsection*{Problem statement}
Prove that
\[(x_1+\cdots+x_n)^2\leq n(x_1^{\:2}+\cdots+x_n^{\:2})\]
for all positive integers $n$ and all real numbers $x_1,\ldots,x_n$.

\subsection*{Solution}
Let's prove this by using the Cauchy-Schwarz inequality on the Euclidean inner product as specified by Example 6.4(a). 
Therefore, if $x_1,\ldots,x_n,y_1,\ldots,y_n\in\mathbf{R}$, then
\[\abs{x_1y_1+\cdots+x_ny_n}^2\leq (x_1^2+\cdots+x_n^2)(y_1^2+\cdots+y_n^2).\]
Now, suppose $(y_1,\ldots,y_n)=(1,\ldots,1)$. 
Then we have
\[\abs{x_1(1)+\cdots+x_n(1)}^2\leq (x_1^2+\cdots+x_n^2)(1^2+\cdots+1^2),\]
which is equivalently expressed as
\[(x_1+\cdots+x_n)^2\leq n(x_1^2+\cdots+x_n^2),\]
which was to be shown.

\clearpage

\section{Law of cosines computes inner products}
\subsection*{Problem statement}
Suppose $u,v$ are nonzero vectors in $\mathbf{R}^2$.
Prove that
\[\langle u,v\rangle=\norm{u}\norm{v}\cos\theta,\]
where $\theta$ is the angle between $u$ and $v$ (thinking of $u$ and $v$ as arrows with initial point at the origin).

\subsection*{Solution}
Via the law of cosines, we can write
\begin{equation}\label{eq:law_cosines}
    \norm{u-v}^2=\norm{u}^2+\norm{v}^2-2\norm{u}\norm{v}\cos\theta.
\end{equation}
By expanding the left-hand side of the above expression, we have
\[\norm{u-v}^2=\langle u-v,u-v \rangle=\norm{u}^2+\langle-v,u\rangle + \langle u,-v \rangle +\norm{v}^2,\]
and since $u,v\in\mathbf{R}^2$, it follows that
\begin{equation}\label{eq:inner_prod_relation}
    \norm{u-v}^2=\norm{u}^2-2\langle u,v \rangle +\norm{v}^2.
\end{equation}
Combining (\ref{eq:law_cosines}) and (\ref{eq:inner_prod_relation}), we have
\[\norm{u}^2-2\langle u,v \rangle +\norm{v}^2=\norm{u}^2+\norm{v}^2-2\norm{u}\norm{v}\cos\theta,\]
and through rearranging terms, we can write
\[\langle u,v\rangle=\norm{u}\norm{v}\cos\theta,\]
which was to be shown.

\clearpage

\section{Angles and the Cauchy-Schwarz Inequality}
\subsection*{Problem statement}
The angle between two vectors (thought of as arrows with initial point at the origin) in $\mathbf{R}^2$ or $\mathbf{R}^3$ can be defined geometrically. 
However, geometry is not as clear in $\mathbf{R}^n$ for $n>3$. 
Thus the angle between two nonzero vectors $x,y\in\mathbf{R}^n$ is defined to be
\[\arccos\frac{\langle x,y\rangle}{\norm{x}\norm{y}},\]
where the motivation for this definition comes from the previous exercise. 
Explain why the Cauchy-Schwarz Inequality is needed to show that this definition makes sense.

\subsection*{Solution}
The $\arccos$ function is only defined for inputs in the range $[-1,1]$. 
With the Cauchy-Schwarz Inequality, the fraction $\frac{\langle x,y\rangle}{\norm{x}\norm{y}}$ is guaranteed to be within $[-1,1]$ since
\[\abs{\langle x,y\rangle} \leq \norm{x}\norm{y}\]
implies
\[\frac{\abs{\langle x,y\rangle}}{\norm{x}\norm{y}}\leq 1,\]
and thus,
\[-1\leq \frac{\langle x,y\rangle}{\norm{x}\norm{y}} \leq 1.\]

\clearpage

\section{Using the Cauchy-Schwarz Inequality: Act 3}
\subsection*{Problem statement}
Prove that
\[(\sum^{n}_{j=1}a_jb_j)^2\leq (\sum^{n}_{j=1}ja_j^{\:2})(\sum^{n}_{j=1}\frac{b_j^{\:2}}{j})\]
for all real numbers $a_1,\ldots,a_n$ and $b_1,\ldots,b_n$.

\subsection*{Solution}
Following from Example 6.17(a), we know that the expression
\[(\sum^{n}_{j=1}x_jy_j)^2\leq (\sum^{n}_{j=1}x_j^{\:2})(\sum^{n}_{j=1}y_j^{\:2})\]
is a valid instance of the Cauchy-Schwarz Inequality (Theorem 6.15) for real numbers $x_1,\ldots,x_n,y_1,\ldots,y_n$. 
By defining $x_j=\sqrt{j}a_j$ and $y_j=\frac{b_j}{\sqrt{j}}$, we get
\[x_jy_j=\sqrt{j}a_j\frac{b_j}{\sqrt{j}}=a_jb_j,\quad x_j^{\:2}=ja_j^{\:2},\quad y_j^{\:2}=\frac{b_j}{j}.\]
Substituting our expressions above into Example 6.17(a), we can write
\[(\sum^{n}_{j=1}a_jb_j)^2\leq (\sum^{n}_{j=1}ja_j^{\:2})(\sum^{n}_{j=1}\frac{b_j^{\:2}}{j})\]
which was to be shown.

\clearpage

\section{Employing the parallelogram equality}
\subsection*{Problem statement}
Suppose $u,v\in V$ are such that 
\[\norm{u}=3,\quad \norm{u+v}=4,\quad \norm{u-v}=6.\]
What number does $\norm{v}$ equal?

\subsection*{Solution}
Via the parallelogram equality (Theorem 6.22), we have
\[\norm{u+v}^2+\norm{u-v}^2=2(\norm{u}^2+\norm{v}^2),\]
and by plugging in numbers, it follows that
\begin{gather*}
    4^2+6^2=2(3^2+\norm{v}^2)\\
    16+36=18+2\norm{v}^2\\
    17=\norm{v}^2
\end{gather*}
and $\norm{v}=\sqrt{17}$.

\clearpage

\section{The $\max$ function isn't a norm}
\subsection*{Problem statement}
Prove or disprove: there is an inner product on $\mathbf{R}^2$ such that the associated norm is given by
\[\norm{(x,y)}=\max\{x,y\}\]
for all $(x,y)\in\mathbf{R}^2$.

\subsection*{Solution}
Let's disprove that such an inner product exists. 
Suppose $(x,y)=(-1,0)$. 
The associated norm is
\[\norm{(-1,0)}=\max\{-1,0\}=0\]
which violates Theorem 6.10 (`Basic properties of the norm') since only the zero vector can have a norm of $0$. 
Thus, $\norm{(x,y)}=\max\{x,y\}$ is not an inner product.

\clearpage

\section{$L^p$-norm is tricky in $\mathbf{R}^2$}
\subsection*{Problem statement}
Suppose $p>0$. 
Prove that there is an inner product on $\mathbf{R}^2$ such that the associated norm is given by
\[\norm{(x,y)}=(x^p+y^p)^{1/p}\]
for all $(x,y)\in \mathbf{R}^2$ if and only if $p=2$.

\subsection*{Solution}
I'll provide two proofs. 
One explicit and correct, and one slick and presumptuous.
\subsubsection*{First Direction (Explicit)}
There are two cases to consider: \textbf{$p$ is odd} and \textbf{$p$ is even}.

\textbf{$p$ is odd}: 
If $p$ is odd, then there does not exist a norm given by
\[\norm{(x,y)}=(x^p+y^p)^{1/p}\]
for all $(x,y)\in \mathbf{R}^2$ because it would violate Theorem 6.10(b) (`Basic properties of the norm') which states
\[\norm{\lambda v}=\abs{\lambda}\norm{v}\]
for all $\lambda\in\mathbf{F}$. 
More explicitly, if $\lambda<0$ and $p$ is odd, then it follows that
\begin{align*}
    \norm{\lambda(x,y)}&=\norm{(\lambda x,\lambda y)}\\
    &=((\lambda x)^p+(\lambda y)^p)^{1/p}\\
    &=((\lambda)^p(x^p+ y^p))^{1/p}\\
    &=\lambda(x^p+y^p)^{1/p}\\
    &=\lambda\norm{(x,y)}\neq\abs{\lambda}\norm{(x,y)}.
\end{align*}

\textbf{$p$ is even}: 
Suppose $p$ is even. 
The squared norm is given by
\[\norm{(x,y)}^2=(x^p+y^p)^{2/p}.\]
Now suppose $u=(1,0)$ and $v=(0,1)$. 
The associated squared norms of $u,v,u+v,u-v$ are
\begin{gather*}
    \norm{u}^2=(1^p+0^p)^{2/p}=1^{2/p}=1,\\
    \norm{v}^2=(0^p+1^p)^{2/p}=1^{2/p}=1,\\
    \norm{u+v}^2=(1^p+1^p)^{2/p}=2^{2/p},\\
    \norm{u-v}^2=(1^p+(-1)^p)^{2/p}=(1^p+1^p)^{2/p}=2^{2/p},
\end{gather*}
where $(-1)^p=1^p$ because $p$ is even. 
Here comes the fun part. 

The squared norms must obey the Parallelogram Equality (Theorem 6.22) which states
\[\norm{u+v}^2+\norm{u-v}^2=2(\norm{u}^2+\norm{v}^2).\]
Plugging in the associated squared norms, we have
\begin{gather*}
    2^{2/p}+2^{2/p}=2(1+1),\\
    2(2^{2/p})=2\cdot 2,\\
    2^1\cdot 2^{2/p}=4,\\
    2^{1+2/p}=2^2.
\end{gather*}
Since the bases are equal and positive, the exponents must also be equal. 
Hence, to solve for $p$, we can write
\begin{gather*}
    1+\frac{2}{p}=2,\\
    \frac{2}{p}=1,\\
    2=p.
\end{gather*}
Therefore, it must be the case that $p=2$.

\subsubsection*{Second Direction (Explicit)}
If $p=2$, then the associated norm 
\[\norm{(x,y)}=(x^2+y^2)^{1/2}\]
is simply an instance of the norm associated with the Euclidean inner product.

\subsubsection*{Slick City}
The squared norm is given by 
\[\norm{(x,y)}^2=(x^p+y^p)^{2/p}.\]
Suppose $u=(1,0)$, $v=(0,1)$, and $u+v=(1,1)$. 
The associated squared norms are
\begin{gather*}
    \norm{u}^2=(1^p+0^p)^{2/p}=1^{2/p}=1,\\
    \norm{v}^2=(0^p+1^p)^{2/p}=1^{2/p}=1,\\
    \norm{u+v}^2=(1^p+1^p)^{2/p}=2^{2/p}.
\end{gather*}
These squared norms must obey the Pythagorean Theorem (Theorem 6.13) which states that for orthogonal vectors, we can write
\[\norm{u+v}^2=\norm{u}^2+\norm{v}^2.\]
Plugging in our values for the associated squared norms, we have
\begin{gather*}
    2^{2/p}=1+1,\\
    2^{2/p}=2,\\
    2^{2/p}=2^1.
\end{gather*}
Since the bases are equal and positive, the exponents must also be equal. 
Therefore, it must be the case that $p=2$.

\textbf{Note}: While this is a ``slick proof'', it assumes that $u$ and $v$ are orthogonal, which we would have to justify. 
We could use Exercise 6.A(19), but then we'd have to calculate $\norm{u-v}^2$, which requires that we specify whether $p$ is positive or negative. 
In that case, we're better off sticking with our explicit proof and using the Parallelogram Equality. 
Interestingly, the Parallelogram Equality is quite powerful because it's a necessary and sufficient condition for norms to have associated inner products (Jordan–von Neumann theorem).

\clearpage

\renewcommand{\thesection}{24}
\section{Injective operators in inner products}
\subsection*{Problem statement}
Suppose $S\in\mathcal{L}(V)$ is an injective operator on $V$. 
Define $\langle\cdot,\cdot\rangle_1$ by
\[\langle u,v\rangle_1=\langle Su,Sv\rangle\]
for $u,v\in V$. 
Show that $\langle\cdot,\cdot\rangle_1$ is an inner product on $V$.

\subsection*{Solution}
Following the definition of an inner product (Definition 6.3), we need to show that $\langle\cdot,\cdot\rangle_1$ satisfies \textbf{positivity}, \textbf{definiteness}, \textbf{additivity in first slot}, \textbf{homogeneity in first slot}, and \textbf{conjugate symmetry}.

\subsubsection*{Positivity}
Positivity of $\langle\cdot,\cdot\rangle_1$ follows from positivity of $\langle Su,Sv\rangle$.

\subsubsection*{Definiteness}
Given the definiteness of $\langle \cdot,\cdot\rangle$,
\[\langle v,v\rangle_1=\langle Sv,Sv\rangle=0,\]
if and only if $Sv=0$. 
Given $S$ is an injective operator, $Sv=0$ if and only if $v=0$. 
Hence, $\langle\cdot,\cdot\rangle_1$ obeys definiteness.

\subsubsection*{Additivity in first slot}
For $u,v,w\in V$, we have
\begin{align*}
    \langle u+v,w\rangle_1&=\langle S(u+v),Sw\rangle\\
    &=\langle Su+Sv,Sw\rangle\\
    &=\langle Su,Sw\rangle+\langle Sv,Sw\rangle\\
    &=\langle u,w\rangle_1 + \langle v,w\rangle_1.
\end{align*}

\subsubsection*{Homogeneity in first slot}
For $\lambda\in\mathbf{F}$ and $u,v\in V$, we have
\begin{align*}
    \langle \lambda v,w\rangle_1&=\langle S( \lambda u),Sw\rangle\\
    &=\langle \lambda Sv,Sw\rangle\\
    &=\lambda \langle Sv,Sw\rangle\\
    &=\lambda\langle u,w\rangle_1.
\end{align*}

\subsubsection*{Conjugate symmetry}
For $u,v\in V$, we have
\begin{align*}
    \langle v,w\rangle_1&=\langle Su,Sw\rangle\\
    &=\overline{\langle Sw,Su\rangle}\\
    &=\overline{\langle w,u\rangle}_1.
\end{align*}

\clearpage

\renewcommand{\thesection}{25}
\section{Non-injective operators in inner products}
\subsection*{Problem statement}
Suppose $S\in\mathcal{L}(V)$ is not injective. 
Define $\langle\cdot,\cdot\rangle_1$ as in the exercise above. 
Explain why $\langle\cdot,\cdot\rangle_1$ is not an inner product on $V$.

\subsection*{Solution}
Since $S$ is not injective, there exists some vector $v\in V$ such that $v\neq0$ and $Sv=0$. 
Hence, it follows that
\[\langle v,v\rangle_1=\langle Sv,Sv\rangle=0,\]
meaning that $\langle\cdot,\cdot\rangle_1$ does not obey the property of definiteness (Definition 6.3) and $\langle\cdot,\cdot\rangle_1$ is not an inner product.

\clearpage

\renewcommand{\thesection}{31}
\section{Apollonius's Identity}
\subsection*{Problem statement}
Use inner products to prove Apollonius's Identity: In a triangle with sides of length $a$, $b$, and $c$, let $d$ be the length of the line segment from the midpoint of the side of length $c$ to the opposite vertex. 
Then
\[a^2+b^2=\frac{1}{2}c^2+2d^2.\]

\subsection*{Solution}
First, we can write the squared norms of $a$ and $b$ as
\[\norm{a}^2=\norm{\frac{1}{2}c+d}^2\quad\text{and}\quad\norm{b}^2=\norm{-\frac{1}{2}c+d}^2.\]
Summing our expressions together, we have
\begin{align*}
    \norm{a}^2+\norm{b}^2&=\norm{\frac{1}{2}c+d}^2+\norm{-\frac{1}{2}c+d}^2\\
    &=\norm{\frac{1}{2}c}^2+\langle \frac{1}{2}c,d\rangle +\langle d,\frac{1}{2}c\rangle+\norm{d}^2\\
    &+\norm{-\frac{1}{2}c}^2+\langle -\frac{1}{2}c,d\rangle +\langle d,-\frac{1}{2}c\rangle+\norm{d}^2.
\end{align*}
Via Theorem 6.7 (`Basic properties of an inner product'), it follows that
\begin{gather*}
    \langle \frac{1}{2}c,d\rangle+\langle -\frac{1}{2}c,d\rangle=\langle\frac{1}{2}c-\frac{1}{2}c,d\rangle=\langle 0,d\rangle=0,\\
    \langle d,\frac{1}{2}c\rangle+\langle d,-\frac{1}{2}c\rangle=\langle d,\frac{1}{2}c-\frac{1}{2}c\rangle=\langle d,0\rangle=0.
\end{gather*}
Via Theorem 6.10(b) (`Basic properties of the norm'), we can write
\begin{gather*}
    \norm{\frac{1}{2}c}^2=\abs{\frac{1}{2}}^2\norm{c}^2=\frac{1}{4}\norm{c}^2,\\
    \norm{-\frac{1}{2}c}^2=\abs{-\frac{1}{2}}^2\norm{c}^2=\frac{1}{4}\norm{c}^2.
\end{gather*}
Hence, putting everything all together, we have
\begin{align*}
    \norm{a}^2+\norm{b}^2&=\norm{\frac{1}{2}c}^2+\langle \frac{1}{2}c,d\rangle +\langle d,\frac{1}{2}c\rangle+\norm{d}^2\\
    &\qquad +\norm{-\frac{1}{2}c}^2+\langle -\frac{1}{2}c,d\rangle +\langle d,-\frac{1}{2}c\rangle+\norm{d}^2\\
    &=\frac{1}{4}\norm{c}^2+ 0 + 0 +\norm{d}^2\\
    &\qquad +\frac{1}{4}\norm{c}^2+ 0 + 0 +\norm{d}^2\\
    &=\frac{1}{2}\norm{c}^2 + 2\norm{d}^2.
\end{align*}

\end{document}