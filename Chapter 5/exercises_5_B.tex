\documentclass{article}
\usepackage{graphicx}
\usepackage{amsmath}
\usepackage{hyperref}
\usepackage{epigraph} 

\title{Linear Algebra Done Right\\Solutions to Exercises 5.B}
\author{}
\date{}

\begin{document}

\maketitle

\epigraph{Algebra is but written geometry and geometry is but figured algebra.}{\textit{Sophie Germain}}

\renewcommand{\thesection}{9}
\section{Zeros of $p$ are eigenvalues of $T$}
\subsection*{Problem statement}
Suppose $V$ is finite-dimensional, $T\in \mathcal{L}(V)$, and $v\in V$ with $v\neq 0$. Let $p$ be a nonzero polynomial of smallest degree such that $p(T)v=0$. Prove that every zero of $p$ is an eigenvalue of $T$.

\subsection*{Solution}
Suppose $\lambda$ is a zero of $p$. We can use theorem 4.11 to write $p$ as 
\[p(z)=(z-\lambda)q(z).\] 
It follows that $p(T)=(T-\lambda I)q(T)$. Given that $p$ is the smallest degree polynomial such that $p(T)v=0$ and $\operatorname{deg}p>\operatorname{deg}q$, it follows that $q(T)v\neq 0$. 

Hence, we have
\[p(T)v=(T-\lambda I)q(T)v=0\]
which implies $T(q(T)v)=\lambda q(T)v$. Therefore $\lambda$ is an eigenvalue of $T$ with $q(T)v$ as the corresponding eigenvector.\footnote{Answer came from \href{https://math.stackexchange.com/questions/4695496/linear-algebra-done-right-exercise-5-b-9-solution-verification}{math.stackexchang.com}.}

\end{document}