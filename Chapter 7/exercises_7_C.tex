\documentclass{article}
\usepackage{graphicx}
\usepackage{amsmath}
\usepackage{hyperref}
\usepackage{epigraph}
\usepackage{csquotes}

\title{Linear Algebra Done Right\\Solutions to Exercises 7.C}
\author{}
\date{}

\begin{document}

\maketitle

\epigraph{Algebra is but written geometry and geometry is but figured algebra.}{\textit{Sophie Germain}}

\section{Self-adjoint on an orthonormal basis}
\subsection*{Problem statement}
Prove or give a counterexample: If $T\in\mathcal{L}(V)$ is self-adjoint and there exists an orthonormal basis $e_1,\ldots,e_n$ of $V$ such that $\langle Te_j,e_j\rangle \geq 0$ for each $j$, then $T$ is a positive operator.

\subsection*{Solution}
Let's give a counterexample. Let's define the operator $T\in\mathcal{L}(\mathbf{R}^2)$ by the matrix
\[M(T) = \begin{pmatrix}4 & 0 \\0 & -2\end{pmatrix}.\]
Now consider the orthonormal basis of $(\frac{1}{\sqrt{2}},\frac{1}{\sqrt{2}})$,$(\frac{1}{\sqrt{2}},-\frac{1}{\sqrt{2}})$. It follows that 
\[\langle T(\frac{1}{\sqrt{2}},\frac{1}{\sqrt{2}}),(\frac{1}{\sqrt{2}},\frac{1}{\sqrt{2}})\rangle=\langle (\frac{4}{\sqrt{2}},-\frac{2}{\sqrt{2}}),(\frac{1}{\sqrt{2}},\frac{1}{\sqrt{2}}) \rangle = \frac{4}{2} -\frac{2}{2}=1 \]
and
\[\langle T(\frac{1}{\sqrt{2}},-\frac{1}{\sqrt{2}}),(\frac{1}{\sqrt{2}},-\frac{1}{\sqrt{2}})\rangle=\langle (\frac{4}{\sqrt{2}},\frac{2}{\sqrt{2}}),(\frac{1}{\sqrt{2}},-\frac{1}{\sqrt{2}}) \rangle = \frac{4}{2} -\frac{2}{2}=1 \]
Hence $T$ satisfies the conditions in the problem statement.

Following Theorem 7.35(b) (`Characterization of positive operators'), $T$ is self-adjoint since it is equal to its conjugate transpose but is not positive because it has a negative eigenvalue of $-2$. Thus it is a counterexample.

\clearpage

\section{Positive operators with $T^2=T$}
\subsection*{Problem statement}
Suppose $T$ is a positive operator on $V$. Suppose $v,w\in V$ are such that 
\[Tv=w\;\;\;\text{and}\;\;\;Tw=v.\]
Prove the $v=w$.

\subsection*{Solution}
Notice that we can nest the two expressions to write
\[T^2v=v\;\;\;\text{and}\;\;\;T^2w=w.\]
Now let's use these expressions and $T$ being positive to deduce that all the eigenvalues of $T$ are either $-1$ or $+1$. Then since $T$ is positive, via Theorem 7.35(b) (`Characterization of positive operators') it follows that all the eigenvalues are $+1$ and $v=w$.

Since $T$ is positive, by definition $T$ is self-adjoint and the Spectral Theorem states that $V$ has an orthonormal basis consisting of eigenvectors of $T$. Let $e_1,\ldots,e_n$ be this orthonormal basis of eigenvectors of $T$ with $\lambda_1,\ldots,\lambda_n$ as the corresponding eigenvectors. Thus we can write $v$ and $w$ as
\[v=a_1e_1+\cdots+a_ne_n\;\;\;\text{and}\;\;\;w=b_1e_1+\cdots+b_ne_n.\]

Given $T^2v=v$ and $T^2w=w$, we can write
\begin{align*}
    T^2(a_1e_1+\cdots+a_ne_n)=a_1\lambda_1^2e_1+&\cdots+a_n\lambda_n^2e_n=a_1e_1+\cdots+a_ne_n\\
    T^2(b_1e_1+\cdots+b_ne_n)=b_1\lambda_1^2e_1+&\cdots+b_n\lambda_n^2e_n=b_1e_1+\cdots+b_ne_n
\end{align*}
Hence it follows that $\lambda_j=\pm 1$. As we've reasoned above, $T$ is positive so all eigenvalues are nonnegative, implying $\lambda_j=1$. Therefore we can write
\begin{align*}
    Tv&=w \\
    T(a_1e_1+\cdots+a_ne_n)&=b_1e_1+\cdots+b_ne_n \\
    a_1\lambda_1e_1+\cdots+a_n\lambda_ne_n&=b_1e_1+\cdots+b_ne_n \\
    a_1e_1+\cdots+a_ne_n&=b_1e_1+\cdots+b_ne_n
\end{align*}
Hence $a_j=b_j$ and $v=w$.

\clearpage

\section{$T|_U\in\mathcal{L}(U)$ is a positive operator on $U$}
\subsection*{Problem statement}
Suppose $T$ is a positive operator on $V$ and $U$ is a subspace of $V$ invariant under $T$. Prove that $T|_U\in\mathcal{L}(U)$ is a positive operator on $U$.

\subsection*{Solution}
Via Theorem 7.28(b) (`Self-adjoint operators and invariant subspaces'), it follows immediately that $T|_U\in\mathcal{L}(U)$ is self-adjoint. To finish the proof, we can write for $u\in U$
\[\langle (T|_U)u,u\rangle=\langle Tu,u\rangle\geq 0\]
where the equality follows from the definition of the \textbf{restriction operator} (Definition 5.14) and the inequality follows from $T$ being positive.

\clearpage

\section{For $T\in\mathcal{L}(V,W)$, $T^*T$ and $TT^*$ are positive}
\subsection*{Problem statement}
Suppose $T\in\mathcal{L}(V,W)$. Prove that $T^*T$ is a positive operator on $V$ and $TT^*$ is a positive operator on $W$.

\subsection*{Solution}
Clearly $T^*T$ is an operator on $V$ and $TT^*$ is an operator on $W$. To show that they are self-adjoint, we can write for $v,u\in V$ and $w,x\in W$
\[\langle T^*Tv,u\rangle=\langle v,(T^*T)^*u\rangle=\langle v,T^*(T^*)^*u\rangle=\langle v,T^*Tu\rangle\]
and
\[\langle TT^*w,x\rangle=\langle w,(TT^*)^*x\rangle=\langle w,(T^*)^*T^*x\rangle=\langle w,TT^*x\rangle.\]
To show that they are positive, we can write for $v\in V$ and $w\in W$
\[\langle T^*Tv,v\rangle=\langle Tv,Tv\rangle\geq 0\]
and
\[\langle TT^*w,w\rangle=\langle T^*v,T^*v\rangle\geq 0\]
where the last inequalities follow from the \textbf{positivity} condition in the definition of an inner product (Definition 6.3).

\clearpage

\section{Sum of two positive operators is positive}
\subsection*{Problem statement}
Prove that the sum of two positive operators on $V$ is positive.

\subsection*{Solution}
Suppose $S,T\in\mathcal{L}(V)$ are positive. We will first prove $S+T$ is self-adjoint and then is positive. For $v,u\in V$, we can write
\begin{align*}
    \langle (S+T)(v),u\rangle&=\langle v,(S+T)^*(u)\rangle=\langle v,S^*u+T^*u\rangle\\
    &=\langle v,Su+Tu\rangle=\langle v,(S+T)(u)\rangle
\end{align*}
where the second equality follows from Theorem 7.6(a) (`Properties of the adjoint') and the third equality comes from the $S$ and $T$ being self-adjoint. Hence $S+T$ is self-adjoint.

To show positivity, for $v\in V$ we can write
\begin{align*}
    \langle (S+T)(v),v\rangle = \langle Sv,v\rangle+\langle Tv,v\rangle\geq 0
\end{align*}
where the inequality follows from the positivity of $S$ and $T$.

\clearpage

\renewcommand{\thesection}{7}
\section{For positive $T$, $T$ invertible iff $\langle Tv,v\rangle > 0$}
\subsection*{Problem statement}
Suppose $T$ is a positive operator on $V$. Prove that $T$ is invertible if and only if
\[\langle Tv,v\rangle > 0\]
for every $v\in V$ with $v\neq 0$.

\subsection*{Solution}
This solution relies on Theorem 7.35(a,e), which essentially states 
\begin{displayquote}
$T$ is positive if and only if there exists an operator $R\in\mathcal{L}(V)$ such that $T=R^*R$.
\end{displayquote}
\subsubsection*{First Direction}
Suppose $T$ is a positive operator and invertible. Following Theorem 7.35(a,e), there exists an operator $R\in\mathcal{L}(V)$ such that $T=R^*R$. Following Exercise 3.D(9)\footnote{Exercise 3.D(9) states that $ST$ is invertible if and only if both $S$ and $T$ are invertible.}, given that $T$ is invertible, then both $R^*$ and $R$ are invertible. Thus we can write for $v\in V$ with $v\neq 0$
\[\langle Tv,v\rangle=\langle R^*Rv,v\rangle=\langle Rv,Rv\rangle >0.\]
To understand the last inequality, since $R$ is invertible, it follows that $\operatorname{null}R=\{0\}$. Hence, for $v\neq 0$ we have $Rv\neq 0$. Invoking the \textbf{positivity} and \textbf{definiteness} properties of inner products, $\langle Rv,Rv\rangle \geq0$ and $\langle Rv,Rv\rangle =0$ if and only if $Rv=0$. Hence it follows that $\langle Rv,Rv\rangle >0$ for $v\in V$ with $v\neq 0$.

\subsubsection*{Second Direction}
Suppose $T$ is a positive operator and $\langle Tv,v\rangle > 0$ for every $v\in V$ with $v\neq 0$. Following Theorem 7.35(a,e), there exists an operator $R\in\mathcal{L}(V)$ such that $T=R^*R$. Thus, for $v\in V$ with $v\neq 0$, we can write
\[0<\langle Tv,v\rangle=\langle R^*Rv,v\rangle=\langle Rv,Rv\rangle\]
implying that $R$ is invertible using the logic of the \textbf{First Direction}.

If we can prove that $R^*$ is also invertible, then Exercise 3.D(9) implies that $T$ is invertible. As a quick \textbf{proof by contradiction}, suppose $R^*$ is not injective. It follows that there exists $u\in\operatorname{range}R=V$ such that $R^*u =0$. Since $R$ is invertible, there exists $w\in V$ such that $Rw=u$. Hence we have
\[\langle Tw,w\rangle=\langle R^*Rw,w\rangle=\langle R^*u,w\rangle=\langle 0,w\rangle=0.\]
Thus it follows that $R^*$ is injective and $R^*$ is invertible.

\clearpage

\renewcommand{\thesection}{10}
\section{More characterization of isometries}
\subsection*{Problem statement}
Suppose $S\in\mathcal{L}(V)$. Prove that the following are equivalent:
\begin{enumerate}
    \item[(a)] $S$ is an isometry;
    \item[(b)] $\langle S^*u,S^*v\rangle=\langle u,v\rangle$ for all $u,v\in V$;
    \item[(c)] $S^*e_1,\ldots,S^*e_m$ is an orthonormal list for every orthonormal list of vectors $e_1,\ldots,e_m$ in $V$;
    \item[(d)] $S^*e_1,\ldots,S^*e_n$ is an orthonormal basis for some orthonormal basis\\$e_1,\ldots,e_n$ in $V$.
\end{enumerate}

\subsection*{Solution}
These statements are essentially just statements in Theorem 7.42 (`Characterization of isometries') with $S^*$ in place of $S$. But let's explicitly show them anyways.

Suppose (a) holds. Via Theorem 7.42(a,g), (a) implies that $S^*$ is an isometry. Via Theorem 7.42(a,b), $S^*$ being an isometry implies that $\langle S^*u,S^*v\rangle=\langle u,v\rangle$ for all $u,v\in V$, and hence (b) holds. 

Now suppose (b) holds. Suppose that $e_1,\ldots,e_m$ is an orthonormal list of vectors in $V$. Then we see that the list $S^*e_1,\ldots,S^*e_m$ is orthonormal because $\langle S^*e_j,S^*e_k\rangle=\langle e_j,e_k\rangle$. Thus (c) holds.

Clearly (c) implies (d). In fact, we could go further by saying $S^*e_1,\ldots,S^*e_n$ is an orthonormal basis for all orthonormal bases $e_1,\ldots,e_n$ in $V$ because (c) is so general.

Now suppose (d) holds. Via Theorem 7.42(d,g), (d) implies that $(S^*)^*$ is an isometry. Via Theorem 7.6(c) (`Properties of the adjoint'), $(S^*)^*=S$ and $S$ is an isometry. Hence, (a) holds.

\end{document}