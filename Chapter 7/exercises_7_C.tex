\documentclass{article}
\usepackage{graphicx}
\usepackage{amsmath}
\usepackage{hyperref}
\usepackage{epigraph} 

\title{Linear Algebra Done Right\\Solutions to Exercises 7.C}
\author{}
\date{}

\begin{document}

\maketitle

\epigraph{Algebra is but written geometry and geometry is but figured algebra.}{\textit{Sophie Germain}}

\section{Self-adjoint on an orthonormal basis}
\subsection*{Problem statement}
Prove or give a counterexample: If $T\in\mathcal{L}(V)$ is self-adjoint and there exists an orthonormal basis $e_1,\ldots,e_n$ of $V$ such that $\langle Te_j,e_j\rangle \geq 0$ for each $j$, then $T$ is a positive operator.

\subsection*{Solution}
Let's give a counterexample. Let's define the operator $T\in\mathcal{L}(\mathbf{R}^2)$ by the matrix
\[M(T) = \begin{pmatrix}4 & 0 \\0 & -2\end{pmatrix}.\]
Now consider the orthonormal basis of $(\frac{1}{\sqrt{2}},\frac{1}{\sqrt{2}})$,$(\frac{1}{\sqrt{2}},-\frac{1}{\sqrt{2}})$. It follows that 
\[\langle T(\frac{1}{\sqrt{2}},\frac{1}{\sqrt{2}}),(\frac{1}{\sqrt{2}},\frac{1}{\sqrt{2}})\rangle=\langle (\frac{4}{\sqrt{2}},-\frac{2}{\sqrt{2}}),(\frac{1}{\sqrt{2}},\frac{1}{\sqrt{2}}) \rangle = \frac{4}{2} -\frac{2}{2}=1 \]
and
\[\langle T(\frac{1}{\sqrt{2}},-\frac{1}{\sqrt{2}}),(\frac{1}{\sqrt{2}},-\frac{1}{\sqrt{2}})\rangle=\langle (\frac{4}{\sqrt{2}},\frac{2}{\sqrt{2}}),(\frac{1}{\sqrt{2}},-\frac{1}{\sqrt{2}}) \rangle = \frac{4}{2} -\frac{2}{2}=1 \]
Hence $T$ satisfies the conditions in the problem statement.

Following Theorem 7.35(b) (`Characterization of positive operators'), $T$ is self-adjoint since it is equal to its conjugate transpose but is not positive because it has a negative eigenvalue of $-2$. Thus it is a counterexample.

\clearpage

\section{Positive operators with $T^2=T$}
\subsection*{Problem statement}
Suppose $T$ is a positive operator on $V$. Suppose $v,w\in V$ are such that 
\[Tv=w\;\;\;\text{and}\;\;\;Tw=v.\]
Prove the $v=w$.

\subsection*{Solution}
Notice that we can nest the two expressions to write
\[T^2v=v\;\;\;\text{and}\;\;\;T^2w=w.\]
Now let's use these expressions and $T$ being positive to deduce that all the eigenvalues of $T$ are either $-1$ or $+1$. Then since $T$ is positive, via Theorem 7.35(b) (`Characterization of positive operators') it follows that all the eigenvalues are $+1$ and $v=w$.

Since $T$ is positive, by definition $T$ is self-adjoint and the Spectral Theorem states that $V$ has an orthonormal basis consisting of eigenvectors of $T$. Let $e_1,\ldots,e_n$ be this orthonormal basis of eigenvectors of $T$ with $\lambda_1,\ldots,\lambda_n$ as the corresponding eigenvectors. Thus we can write $v$ and $w$ as
\[v=a_1e_1+\cdots+a_ne_n\;\;\;\text{and}\;\;\;w=b_1e_1+\cdots+b_ne_n.\]

Given $T^2v=v$ and $T^2w=w$, we can write
\begin{align*}
    T^2(a_1e_1+\cdots+a_ne_n)=a_1\lambda_1^2e_1+&\cdots+a_n\lambda_n^2e_n=a_1e_1+\cdots+a_ne_n\\
    T^2(b_1e_1+\cdots+b_ne_n)=b_1\lambda_1^2e_1+&\cdots+b_n\lambda_n^2e_n=b_1e_1+\cdots+b_ne_n
\end{align*}
Hence it follows that $\lambda_j=\pm 1$. As we've reasoned above, $T$ is positive so all eigenvalues are nonnegative, implying $\lambda_j=1$. Therefore we can write
\begin{align*}
    Tv&=w \\
    T(a_1e_1+\cdots+a_ne_n)&=b_1e_1+\cdots+b_ne_n \\
    a_1\lambda_1e_1+\cdots+a_n\lambda_ne_n&=b_1e_1+\cdots+b_ne_n \\
    a_1e_1+\cdots+a_ne_n&=b_1e_1+\cdots+b_ne_n
\end{align*}
Hence $a_j=b_j$ and $v=w$.

\clearpage

\section{$T|_U\in\mathcal{L}(U)$ is a positive operator on $U$}
\subsection*{Problem statement}
Suppose $T$ is a positive operator on $V$ and $U$ is a subspace of $V$ invariant under $T$. Prove that $T|_U\in\mathcal{L}(U)$ is a positive operator on $U$.

\subsection*{Solution}
Following Theorem 7.28 (`Self-adjoint operators and invariant subspaces'), it follows immediately that $T|_U\in\mathcal{L}(U)$ is self-adjoint. To finish the proof, we can write for $u\in U$
\[\langle (T|_U)u,u\rangle=\langle Tu,u\rangle\geq 0\]
where the equality follows from the definition of the \textbf{restriction operator} (Definition 5.14) and the inequality follows from $T$ being positive.

\clearpage



\end{document}