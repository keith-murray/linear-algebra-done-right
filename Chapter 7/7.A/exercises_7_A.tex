\documentclass{article}
\usepackage{graphicx}
\usepackage{amsmath}
\usepackage{hyperref}
\usepackage{epigraph} 

\title{Linear Algebra Done Right\\Solutions to Exercises 7.A}
\author{}
\date{}

\begin{document}

\maketitle

\section{Adjoint of the forward shift operator}
\subsection*{Problem statement}
Suppose $n$ is a positive integer. Define $T\in\mathcal{L}(\mathbf{F}^{n})$ by
\[T(z_1,\ldots,z_n)=(0,z_1,\ldots,z_{n-1}).\]
Find a formula for $T^*(z_1,\ldots,z_n)$.

\subsection*{Solution}
Following Examples 7.3 and 7.4, we can write
\begin{align*} 
\langle(x_1,\ldots,x_n),T^*(z_1,\ldots,z_n)\rangle &= \langle T(x_1,\ldots,x_n),(z_1,\ldots,z_n)\rangle \\ 
 &= \langle (0,x_1,\ldots,x_{n-1}),(z_1,\ldots,z_n)\rangle \\
 &= 0(z_1) + x_1(z_2) + \cdots + x_{n-1}(z_n) \\
 &= x_1(z_2) + \cdots + x_{n-1}(z_n) + 0(x_nz_1) \\
 &= \langle (x_1,\ldots,x_n),(z_2,\ldots,z_n,0)\rangle.
\end{align*}
Thus,
\[T^*(z_1,\ldots,z_n)=(z_2,\ldots,z_n,0)\].

\clearpage

\section{Complex conjugates of eigenvalues for adjoints}
\subsection*{Problem statement}
Suppose $T\in\mathcal{L}(V)$ and $\lambda\in\mathbf{F}$. Prove that $\lambda$ is an eigenvalue of $T$ if and only if $\bar{\lambda}$ is an eigenvalue of $T^*$.

\subsection*{Solution}
First, let's write 
\[\operatorname{null}(T^*-\bar{\lambda}I)=\operatorname{null}(T-\lambda I)^*=(\operatorname{range}(T-\lambda I))^\bot\]
where the last equality comes from Theorem 7.7(a) (`Null space and range of $T^*$'). 
Thus it follows that 
\[\operatorname{dim}\operatorname{null}(T^*-\bar{\lambda}I)=\operatorname{dim}(\operatorname{range}(T-\lambda I))^\bot.\]
To understand $\operatorname{dim}(\operatorname{range}(T-\lambda I))^\bot$, we can use the Fundamental Theorem of Linear Maps to write
\[\operatorname{dim}\operatorname{range}(T-\lambda I)=\operatorname{dim}V-\operatorname{dim}\operatorname{null}(T-\lambda I)\]
and it follows from Theorem 6.50 (`Dimension of the orthogonal complement') that
\begin{align*}
    \operatorname{dim}(\operatorname{range}(T-\lambda I))^\bot&=\operatorname{dim}V-(\operatorname{dim}V-\operatorname{dim}\operatorname{null}(T-\lambda I))\\
    &=\operatorname{dim}\operatorname{null}(T-\lambda I).
\end{align*}
Hence, we have
\begin{equation}\label{eq:dim_T_lambda}
    \operatorname{dim}\operatorname{null}(T^*-\bar{\lambda}I)=\operatorname{dim}\operatorname{null}(T-\lambda I).
\end{equation}

\subsubsection*{First Direction}
Suppose $\lambda$ is an eigenvalue of $T$. 
Following from (\ref{eq:dim_T_lambda}), this implies that 
\[\operatorname{dim}\operatorname{null}(T^*-\bar{\lambda}I)=\operatorname{dim}\operatorname{null}(T-\lambda I)>0\]
and $\bar{\lambda}$ is an eigenvalue of $T^*$.

\subsubsection*{Second Direction}
Suppose $\bar{\lambda}$ is an eigenvalue of $T^*$. 
Following from (\ref{eq:dim_T_lambda}), this implies that 
\[\operatorname{dim}\operatorname{null}(T-\lambda I)=\operatorname{dim}\operatorname{null}(T^*-\bar{\lambda}I)>0\]
and $\lambda$ is an eigenvalue of $T$.

\clearpage

\section{$U$ invariant under $T$ iff $U^\bot$ invariant under $T^*$}
\subsection*{Problem statement}
Suppose $T\in\mathcal{L}(V)$ and $U$ is a subspace of $V$. Prove that $U$ is invariant under $T$ if and only if $U^\bot$ is invariant under $T^*$.

\subsection*{Solution}
\subsubsection*{First Direction}
Suppose $U$ is invariant under $T$. It follows that for all $v\in U$ and $w\in U^\bot$, we can write $\langle Tv,w\rangle=0$ since $Tv\in U$. Given the definition of the adjoint, we can write $0=\langle Tv,w\rangle=\langle v,T^*w\rangle$ for all $v\in U$ and $w\in U^\bot$. Hence this implies $U^\bot$ is invariant under $T^*$.

\subsubsection*{Second Direction}
Suppose $U^\bot$ is invariant under $T^*$. It follows that for all $v\in U$ and $w\in U^\bot$, we can write $\langle T^* w,v\rangle=0$ since $T^* w\in U^\bot$. Given the definition of the adjoint, we can write $0=\langle T^* w,v\rangle=\langle w,Tv\rangle$ for all $v\in U$ and $w\in U^\bot$. Hence this implies $U$ is invariant under $T$.

\clearpage

\section{$T$ is injective/surjective iff $T^*$ is surjective/injective}
\subsection*{Problem statement}
Suppose $T\in\mathcal{L}(V,W)$. Prove that
\begin{enumerate}
  \item[(a)] $T$ is injective if and only if $T^*$ is surjective;
  \item[(b)] $T$ is surjective if and only if $T^*$ is injective
\end{enumerate}
\textbf{Note:} For Chapter 7, we can assume $V$ and $W$ are finite-dimensional.

\subsection*{Solution for (a)}
\subsubsection*{First Direction}
Suppose $T$ is injective. Then $\operatorname{null}T=\{0\}$. Via Theorem 7.7(c) (`Null space and range of $T^*$'), 
\[\{0\}=\operatorname{null}T=(\operatorname{range}T^*)^\bot.\]
Following Theorem 6.51 (`The orthogonal complement of the orthogonal complement') and Theorem 6.46(b) (`Basic properties of orthogonal complement'), 
\[\operatorname{range}T^*=((\operatorname{range}T^*)^\bot)^\bot=\{0\}^\bot=V.\]
Hence, given $T^*\in\mathcal{L}(W,V)$, it follows that $T^*$ is surjective.

\subsubsection*{Second Direction}
Suppose $T^*$ is surjective. Then $\operatorname{range}T^*=V$. Via Theorem 7.7(b) (`Null space and range of $T^*$'), 
\[V=\operatorname{range}T^*=(\operatorname{null}T)^\bot.\]
Following Theorem 6.51 (`The orthogonal complement of the orthogonal complement') and Theorem 6.46(c) (`Basic properties of orthogonal complement'), 
\[\operatorname{null}T=((\operatorname{null}T)^\bot)^\bot=V^\bot=\{0\}.\]
Hence, it follows that $T$ is injective.

\subsection*{Solution for (b)}
\textbf{Solution for (b)} follows a similar pattern as \textbf{solution for (a)}, except for changing injective to surjective and surjective to injective, and reasoning over $W$ instead of $V$.

\clearpage

\section{$\operatorname{dim}\operatorname{range}T^*=\operatorname{dim}\operatorname{range}T$}
\subsection*{Problem statement}
Prove that
\[\operatorname{dim}\operatorname{null}T^*=\operatorname{dim}\operatorname{null}T+\operatorname{dim}W-\operatorname{dim}V\]
and
\[\operatorname{dim}\operatorname{range}T^*=\operatorname{dim}\operatorname{range}T\]
for every $T\in\mathcal{L}(V,W)$.

\subsection*{Solution for (a)}
Let's first prove $\operatorname{dim}\operatorname{range}T^*=\operatorname{dim}\operatorname{range}T$ and the other proof will easily follow from the Fundamental Theorem of Linear Maps.

Via Theorem 7.7(d) (`Null space and range of $T^*$'), we can write
\[\operatorname{dim}\operatorname{range}T=\operatorname{dim}(\operatorname{null}T^*)^\bot\]
and via Theorem 6.50 (`Dimension of the orthogonal complement'), we can expand on our previous statement to write
\begin{equation}\label{eq:five}
    \operatorname{dim}\operatorname{range}T=\operatorname{dim}W-\operatorname{dim}\operatorname{null}T^*.
\end{equation}
Via the Fundamental Theorem of Linear Maps, we have
\[\operatorname{dim}W=\operatorname{dim}\operatorname{null}T^*+\operatorname{dim}\operatorname{range}T^*\]
and thus we can expand on (\ref{eq:five}) to write
\[\operatorname{dim}\operatorname{range}T=\operatorname{dim}W-\operatorname{dim}W+\operatorname{dim}\operatorname{range}T^*=\operatorname{dim}\operatorname{range}T^*.\]

Wonderful. Now via a simple application of the Fundamental Theorem of Linear Maps, we have
\[\operatorname{dim}V-\operatorname{dim}\operatorname{null}T=\operatorname{dim}W-\operatorname{dim}\operatorname{null}T^*\]
which is only a couple of rearrangements away from 
\[\operatorname{dim}\operatorname{null}T^*=\operatorname{dim}\operatorname{null}T+\operatorname{dim}W-\operatorname{dim}V.\]

\clearpage

\renewcommand{\thesection}{7}
\section{$ST$ is self-adjoint iff $ST=TS$}
\subsection*{Problem statement}
Suppose $S,\,T\in\mathcal{L}(V)$ are self-adjoint. Prove that $ST$ is self-adjoint if and only if $ST=TS$.

\subsection*{Solution}
\subsubsection*{First Direction}
Suppose $ST$ is self-adjoint. Following Theorem 7.6(e) (`Properties of the adjoint'), we have $ST=(ST)^*=T^*S^*$. However, since $S$ and $T$ are self-adjoint, it follows that $ST=S^*T^*$. \href{https://www.youtube.com/watch?v=TCm9788Tb5g}{Now watch this drive}
\[ST=(ST)^*=(S^*T^*)^*=(T^*)^*(S^*)^*=TS\]

\subsubsection*{Second Direction}
Suppose $ST=TS$ (remember that $S$ and $T$ are self-adjoint). For all $v,\,w\in V$, we can write
\[\langle STv,w\rangle=\langle v, T^*S^*w\rangle=\langle v,TSw\rangle=\langle v,STw\rangle\]
where the second equality comes from $S=S^*$ and $T=T^*$, and the third equality comes from $ST=TS$.

\clearpage

\renewcommand{\thesection}{8}
\section{Over $\mathbf{R}$, self-adjoint operators make subspace}
\subsection*{Problem statement}
Suppose $V$ is a real inner product space. Show that the set of self-adjoint operators on $V$ is a subspace of $\mathcal{L}(V)$.

\subsection*{Solution}
Let's call $\mathcal{A}(V)$ the set of self-adjoint operators on $V$. To show that it's a subspace, we need to show it contains the \textbf{additive identity}, is \textbf{closed under addition}, and is \textbf{closed under scalar multiplication}.

\subsubsection*{Additive Identity}
Suppose $T=0$. It follows that for all $v,\,w\in V$ we have $\langle Tv,w\rangle=0=\langle v, T^*w\rangle$, implying that $T^*=0$. Hence, $0$ is self-adjoint and $0\in\mathcal{A}(V)$.

\subsubsection*{Closed under addition}
Suppose $S,\,T$ are self-adjoint ($S,\,T\in\mathcal{A}(V)$). For all $v,\,w\in V$, we can write
\[\langle (S+T)(v),w\rangle=\langle Sv,w\rangle+\langle Tv,w\rangle=\langle v,Sw\rangle+\langle v,Tw\rangle=\langle v,(S+T)(w)\rangle\]
Hence, $S+T$ is self-adjoint and $S+T\in\mathcal{A}(V)$.

\subsubsection*{Closed under scalar multiplication}
Suppose $T$ is self-adjoint and $\lambda\in\mathbf{R}$ (since $V$ is a real inner product space). It follows that for all $v,\,w\in V$, we can write
\[\langle (\lambda T)(v),w\rangle=\lambda\langle Tv,w\rangle=\lambda\langle v,Tw\rangle=\langle v,(\bar{\lambda}\,T)(w)\rangle\]
Given $\lambda=\bar{\lambda}$, we have
\[\langle (\lambda T)(v),w\rangle=\langle v,(\bar{\lambda}\,T)(w)\rangle=\langle v,(\lambda\,T)(w)\rangle\]
and $\lambda\,T$ is self-adjoint and $\lambda\,T\in\mathcal{A}(V)$.

\clearpage

\renewcommand{\thesection}{9}
\section{Over $\mathbf{C}$, self-adjoint operators isn't subspace}
\subsection*{Problem statement}
Suppose $V$ is a complex inner product space with $V\neq\{0\}$. Show that the set of self-adjoint operators on $V$ is not a subspace of $\mathcal{L}(V)$.

\subsection*{Solution}
Refer to our answer for the previous exercise, exercise 7.A(8). The set of self-adjoint operators on $V$ was a subspace because for $\lambda\in\mathbf{R}$ we could write $\lambda=\bar{\lambda}$. Sadly, we cannot write such things for complex innder product spaces. Therefore the set of self-adjoints operators is not closed under scalar multiplication.

\end{document}